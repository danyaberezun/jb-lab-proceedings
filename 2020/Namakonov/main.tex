\newcommand{\valuecom}[1]{\textcolor{green!60!black}{/\!/ #1}}
\newcommand{\textdom}[1]{\mathsf{#1}}
\newcommand{\textcode}[1]{\texorpdfstring{\texttt{#1}}{#1}}
\newcommand{\kw}[1]{\textbf{\textcode{#1}}}
\newcommand{\skipc}{\kw{skip}}
\newcommand{\ite}[3]{\kw{if}\;#1\:\kw{then}\;#2\;\\ \kw{else}\;#3}
\newcommand{\iteml}[3]{
  \kw{if} \; #1\\
  \begin{array}[t]{@{}l@{}l}
    \kw{then}& \begin{array}[t]{l} #2 \end{array} \\
    \kw{else}& \begin{array}[t]{l} #3 \end{array} \\
  \end{array}
}

\newcommand{\set}[1]{\{{#1}\}}
\newcommand{\defeq}{\triangleq}
\renewcommand{\implies}{\Rightarrow}

\colorlet{colorPO}{gray!60!black}
\colorlet{colorRF}{green!60!black}
\colorlet{colorMO}{orange}
\colorlet{colorFR}{purple}
\colorlet{colorECO}{red!80!black}
\colorlet{colorSYN}{green!40!black}
\colorlet{colorHB}{blue}
\colorlet{colorPPO}{magenta}
\colorlet{colorPB}{olive}
\colorlet{colorSBRF}{olive}
\colorlet{colorRMW}{olive!70!black}
\colorlet{colorRSEQ}{blue}
\colorlet{colorSC}{violet}
\colorlet{colorPSC}{violet}
\colorlet{colorREL}{olive}
\colorlet{colorCONFLICT}{olive}
\colorlet{colorRACE}{olive}
\colorlet{colorWB}{orange!70!black}
\colorlet{colorPSC}{violet}
\colorlet{colorSCB}{violet}
\colorlet{colorDEPS}{violet}
\colorlet{colorAR}{black}

\tikzset{
   every path/.style={>=stealth},
   po/.style={->,color=colorPO,thin,shorten >=-0.5mm,shorten <=-0.5mm},
   sw/.style={->,color=colorSYN,shorten >=-0.5mm,shorten <=-0.5mm},
   rf/.style={->,color=colorRF,dashed,,shorten >=-0.5mm,shorten <=-0.5mm},
   hb/.style={->,color=colorHB,thick,shorten >=-0.5mm,shorten <=-0.5mm},
   co/.style={->,color=colorMO,dotted,very thick,shorten >=-0.5mm,shorten <=-0.5mm},
   no/.style={->,dotted,thick,shorten >=-0.5mm,shorten <=-0.5mm},
   fr/.style={->,color=colorFR,dotted,thick,shorten >=-0.5mm,shorten <=-0.5mm},
   deps/.style={->,color=colorDEPS,dotted,thick,shorten >=-0.5mm,shorten <=-0.5mm},
   rmw/.style={->,color=colorRMW,thick,shorten >=-0.5mm,shorten <=-0.5mm},
   pngexport/.style={
     external/system call/.add=
     {}
     {; convert -density 300 -transparent white "\image.pdf" "\image.png"},
     % 
     /pgf/images/external info,
     /pgf/images/include external/.code={%
       \includegraphics
       [width=\pgfexternalwidth,height=\pgfexternalheight]
       {##1.png}%
     },
   }
 }
\newcommand{\dpo}[2]{\draw[po] (#1) edge (#2)}
\newcommand{\dpopo}[2]{\draw[po] (#1) edge node[right] {$\lPO$} (#2)}
\newcommand{\drmw}[2]{\draw[rmw, bend right=20] (#1) edge node[left] {$\lRMW$} (#2)}
\newcommand{\dco}[2]{\draw[co] (#1) edge node[right] {$\lCO$} (#2)}
\newcommand{\dcoext}[4]{\draw[co, #3] (#1) edge node[#4] {$\lCO$} (#2)}
\newcommand{\drf}[2]{\draw[rf] (#1) edge node[left] {$\lRF$} (#2)}
\newcommand{\dfr}[3]{\draw[fr, #3] (#1) edge node[left] {$\lFR$}(#2)}
\newcommand{\dhb}[3]{\draw[hb, #3] (#1) edge node[right] {$\lHB$} (#2)}


%% Orders
\newcommand{\pln}{\mathtt{pln}}
\newcommand{\rlx}{\mathtt{rlx}}
\newcommand{\rel}{{\mathtt{rel}}}
\newcommand{\acq}{{\mathtt{acq}}}
\newcommand{\acqrel}{{\mathtt{acqrel}}}
\newcommand{\sco}{{\mathtt{sc}}}
% omm modes
\newcommand{\na}{{\mathtt{na}}}
\newcommand{\at}{{\mathtt{at}}}

%% Event labels
\newcommand{\rlab}[3]{{\lR}^{#1}({#2},{#3})}
\newcommand{\wlab}[3]{{\lW}^{#1}({#2},{#3})}
\newcommand{\flab}[1]{{\lF}(#1)}

\newcommand{\lR}{{\mathtt{R}}}
\newcommand{\lW}{{\mathtt{W}}}
\newcommand{\lF}{{\mathtt{F}}}
\newcommand{\lE}{{\mathtt{E}}}

%% Relations
\newcommand{\lPO}{{\color{colorPO}\mathtt{po}}}
\newcommand{\lRF}{{\color{colorRF} \mathtt{rf}}}
\newcommand{\lRMW}{{\color{colorRMW} \mathtt{rmw}}}
\newcommand{\lMO}{{\color{colorMO} \mathtt{mo}}}
\newcommand{\lMOx}{{\color{colorMO} \mathtt{mo}}_x}
\newcommand{\lMOy}{{\color{colorMO} \mathtt{mo}}_y}
\newcommand{\lCO}{{\color{colorMO} \mathtt{co}}}
\newcommand{\lCOx}{{\color{colorMO} \mathtt{co}}_x}
\newcommand{\lCOy}{{\color{colorMO} \mathtt{co}}_y}
\newcommand{\lFR}{{\color{colorFR} \mathtt{fr}}}
\newcommand{\lFRx}{{\color{colorFR} \mathtt{rb}}_x}
\newcommand{\lFRy}{{\color{colorFR} \mathtt{rb}}_y}
\newcommand{\lECO}{{\color{colorECO} \mathtt{eco}}}
\newcommand{\lSW}{{\color{colorSYN}\mathtt{sw}}}
\newcommand{\lHB}{{\color{colorHB}\mathtt{hb}}}
\newcommand{\lHBO}{{\color{olive}\mathtt{hbo}}}
\newcommand{\lDOB}{{\mathtt{dob}}}
\newcommand{\lBOB}{{\mathtt{bob}}}
\newcommand{\lAOB}{{\mathtt{aob}}}
\newcommand{\lOBS}{{\mathtt{obs}}}
\newcommand{\lEORD}{{\mathtt{eord}}}
\newcommand{\lTORD}{{\mathtt{tord}}}
\newcommand{\lSC}{{\mathtt{sc}}}
\newcommand{\lAR}{{\color{colorAR} \mathtt{ar}}}

\newcommand{\lSCB}{{\color{colorSCB} \mathtt{scb}}}
\newcommand{\lPSC}{{\color{colorPSC} \mathtt{psc}}}
\newcommand{\lPSCB}{\lPSC_{\rm base}}
\newcommand{\lPSCF}{\lPSC_\lF}

\newcommand{\lDEPS}{{{\color{colorDEPS}\mathtt{deps}}}}
\newcommand{\lCTRL}{{{\color{colorDEPS}\mathtt{ctrl}}}}
\newcommand{\lCTRLISYNC}{{{\color{colorDEPS}\mathtt{ctrl_{isync}}}}}
\newcommand{\lDATA}{{{\color{colorDEPS}\mathtt{data}}}}
\newcommand{\lADDR}{{{\color{colorDEPS}\mathtt{addr}}}}
\newcommand{\lCASDEP}{{{\color{colorDEPS}\mathtt{casdep}}}}

\newcommand{\lmakeE}[1]{#1\mathtt{e}}
\newcommand{\lRFE}{\lmakeE{\lRF}}
\newcommand{\lCOE}{\lmakeE{\lCO}}
\newcommand{\lFRE}{\lmakeE{\lFR}}
\newcommand{\lMOE}{\lmakeE{\lMO}}
\newcommand{\lmakeI}[1]{#1\mathtt{i}}
\newcommand{\lRFI}{\lmakeI{\lRF}}
\newcommand{\lCOI}{\lmakeI{\lCO}}
\newcommand{\lFRI}{\lmakeI{\lFR}}

\newcommand{\Tid}{\mathsf{Tid}}
\newcommand{\Loc}{\mathsf{Loc}}
\newcommand{\Val}{\mathsf{Val}}
\newcommand{\Init}{\mathsf{Init}}
\newcommand{\Lab}{\mathsf{Lab}}
\newcommand{\Mod}{\mathsf{Mod}}
\newcommand{\Modr}{\mathsf{Mod}_{\lR}}
\newcommand{\Modw}{\mathsf{Mod}_{\lW}}
\newcommand{\Modf}{\mathsf{Mod}_{\lF}}
\newcommand{\Modrmw}{\mathsf{Mod}_{\lU}}


% example rels
\newcommand{\exX}{\mathtt{x}}
\newcommand{\exY}{\mathtt{y}}

\definecolor{StringRed}{rgb}{.637,0.082,0.082}
\definecolor{CommentGreen}{rgb}{0.0,0.55,0.3}
\definecolor{KeywordBlue}{rgb}{0.0,0.3,0.55}
\definecolor{LinkColor}{rgb}{0.55,0.0,0.3}
\definecolor{CiteColor}{rgb}{0.55,0.0,0.3}
\definecolor{HighlightColor}{rgb}{0.0,0.0,0.0}

\definecolor{grey}{rgb}{0.5,0.5,0.5}
\definecolor{red}{rgb}{1,0,0}
\definecolor{darkgreen}{rgb}{0.0,0.7,0.0}

\hypersetup{%
  linktocpage=true, pdfstartview=FitV,
  breaklinks=true, pageanchor=true, pdfpagemode=UseOutlines,
  plainpages=false, bookmarksnumbered, bookmarksopen=true, bookmarksopenlevel=3,
  hypertexnames=true, pdfhighlight=/O,
}


\newcommand{\commentNonempty}[1]{
  \ifx\\#1\\
    {}
  \else
    \valuecom{#1}
  \fi
  }
\newcommand{\readInst}[4]{#1 \;:=\;[#2]^{#4}; \commentNonempty{#3}}
\newcommand{\fenceInst}[1]{\mathtt{fence}^{#1};}
\newcommand{\writeInst}[3]{[#1]^{#3}\;:=\;#2;}
\newcommand{\casInstSC}[3]{\mathtt{CAS^{\sco, \sco}_{\sco}(#1, #2, #3)};}
\newcommand{\assignInst}[2]{#1\;:=\;#2;}
\newcommand{\exchangeInstSC}[2]{\mathtt{exchange^{\sco}(#1, #2)};}
\newcommand{\term}[1]{\emph{#1}}



\newcommand{\Wrlx}{\lW^{\rlx}}
\newcommand{\Rrlx}{\lR^{\rlx}}
\newcommand{\Wsc}{\lW^{\sco}}
\newcommand{\Rsc}{\lR^{\sco}}
\newcommand{\Fa}{\lF^{\acq}}
\newcommand{\Fr}{\lF^{\rel}}
\newcommand{\Far}{\lF^{\acqrel}}
\newcommand{\ar}{\ensuremath{ar}}
\newcommand{\IMM}{\mathtt{IMM}}
\newcommand{\OMM}{\mathtt{OCaml}\allowbreak \mathtt{MM}}
\newcommand{\todo}[1]{\textbf{\Large TODO: \textcolor{red}{#1}}}
\newcommand{\strongereq}{\sqsupseteq}
\newcommand{\defin}[1]{\textit{#1}}
\newcommand{\imm}{|_{imm}}

% \declaretheorem[name=Теорема,style=default,numberwithin=section]{thm}
% \declaretheorem[name=Лемма,style=default,numberwithin=section]{lem}
% \declaretheorem[name=Определение,style=default,sibling=thm]{mydefinition}

%%% Local Variables:
%%% mode: latex
%%% TeX-master: t
%%% End:



\hyphenation{%
  не-сог-ла-со-ван-ность
}

% \begin{document}

% \filltitle{ru}{
%     chair              = {Кафедра системного программирования},
%     title              = {Компиляция модели памяти OCaml в Power},
%     type               = {master},
%     position           = {студента},
%     group              = 18.М08-мкн,
%     author             = {Намаконов Егор Сергеевич},
%     supervisor         = {Кознов Д.\,В.},
%     supervisorPosition = {д.ф.-м.н.},
%     reviewer           = {Березун Д. А.},
%     reviewerPosition   = {к.ф.-м.н.},
%     consultant         = {Подкопаев А.В.},
%     consultantPosition = {к.ф.-м.н.},
%   university         = {Санкт-Петербургский государственный университет},
%   faculty            = {Математическое обеспечение и администрирование информационных систем},
%   city               = {Санкт-Петербург},
%   year               = {2020}
% }
\title{Компиляция модели памяти OCaml в Power}
\titlerunning{Компиляция модели памяти OCaml в Power}

\author{Намаконов Егор Сергеевич}
\authorrunning{Намаконов Е.С.}

\tocauthor{Намаконов Егор Сергеевич}
\institute{Санкт-Петербургский государственный университет
  \email{e.namakonov@gmail.com}}


\maketitle

\selectlanguage{russian}

\begin{abstract}
  В настоящее время для языков программирования и процессоров активно разрабатываются модели памяти, направленные на решение различных проблем многопоточного программирования.
  Так, модель памяти языка OCaml  позволяет избежать неопределённого поведения, вызванного гонками по данным.
  Для применения этой модели на практике необходимо доказать корректность её компиляции в распространённые архитектуры процессоров. На данный момент это выполнено для x86-TSO и ARMv8, но не для Power.

  В данной работе предложена схема компиляции модели OCaml в промежуточную модель и доказана её корректность.
  Так как для промежуточной модели уже доказана корректность компиляции в наиболее распространённые архитектуры, полученная схема компиляции даёт корректную схему компиляции модели OCaml в архитектуру Power, а также во все остальные распространённые архитектуры.
\end{abstract}
{\bf Ключевые слова:} слабые модели памяти, корректность компиляции, многопоточность.
% \newgeometry{a4paper,top=20mm,bottom=20mm,left=30mm,right=15mm,nohead,includeheadfoot} % for some reason it's ignored when placed in cls file

% \tableofcontents



\section{Введение}

Результат исполнения многопоточной программы, как правило, является недетерминированным. Конкретное множество допустимых результатов многопоточной программы определяется \defin{моделью памяти} языка программирования. Наиболее известной является   \defin{модель последовательной согласованности} (\foreignlanguage{english}{sequential consistency}, SC \cite{sc}). Она предполагает, что любой результат исполнения программы может быть получен путём попеременного исполнения инструкций отдельных потоков согласно программному порядку в них. Однако из-за оптимизаций, выполняемых современными компиляторами и процессорами, могут наблюдаться сценарии поведения, невозможные в такой модели. Так, на архитектуре x86 чтение по адресу в памяти может вернуть не самое последнее записанное значение, так как операция записи может быть буферизована. 

Отказ от подобных оптимизаций нежелателен, поэтому современные модели памяти допускают некоторые сценарии поведения, невозможные в модели SC. Такие модели памяти называются \defin{слабыми}. Например, слабыми являются модели памяти языков C++ \cite{cpp}, JavaScript \cite{js-mm} и Java \cite{jmm}, а также архитектур Power \cite{power}, x86 \cite{x86} и ARM \cite{arm}.

%Как правило, слабые модели памяти дают более сильные гарантии на поведение программ, в которых конфликтующие обращения по одному и тому же адресу должным образом синхронизированы. В противном случае такие обращения образуют \defin{гонку по данным}, и в этом случае модель памяти может ослабить гарантии на поведение программы. Так, в модели C++ к гонке по данным приводят конфликтующие неатомарные обращения, и в этом случае поведение всей программы объявляется неопределённым \cite{cpp}.
%В модели памяти Java для предотвращения гонки используются встроенные в язык средства синхронизации \cite{jmm}; в случае же возникновения гонки по некоторому адресу допускается чтение произвольных значений по нему в будущем \cite{omm}.

%Для обеспечения баланса между производительностью и предсказуемостью поведения программы современные модели памяти предоставляют программисту гарантии DRF (data race freedom, \textit{свобода от гонок}). Они гарантируют, что поведение программы, не содержащей гонок по данным, будет согласовано с моделью SC. Например, свойство DRF предоставляют модели памяти Java и Promising \cite{promising}. 

Модель памяти OCaml \cite{omm} (далее --- $\OMM$) отличается свойством т.н.  \textit{локальной свободы от гонок по данным} (local data race freedom). Именно, гарантируется, что выполнение конфликтующих обращений по выбранному адресу в памяти (т.е. ситуация гонки по данным) не влияет на обращения к другим адресам, а также на последующие обращения по тому же адресу. Благодаря этому даже при возникновении гонки по данным в некоторый момент исполнения следующие участки программы будут исполнены согласно модели SC. 

% Для того, чтобы использовать $\OMM$ на практике, необходимо доказать её реализуемость на распространённых архитектурах процессоров.
Чтобы гарантировать выполнение этого свойства, при компиляции нужно запретить некоторые оптимизации в зависимости от целевой архитектуры.
% Этого можно достичь, если в зависимости от типа инструкции при компиляции выбирать подходящие режимы доступа или добавлять барьеры в ассемблерный код.
Для этого может понадобиться, например, добавить в ассемблерный код инструкции-\defin{барьеры}, которые запрещают нежелательные оптимизации на уровне процессора. 
Набор таких правил, покрывающий все возможные типы инструкций, называется \defin{схемой компиляции}. Схема компиляции должна быть \defin{корректной} --- при исполнении любой программы, полученной при компиляции согласно этой схеме, должно наблюдаться только сценарии поведения, разрешённые $\OMM$ для исходной программы.

Авторы $\OMM$ разработали схемы компиляции $\OMM$ в модели x86-TSO и ARMv8 \cite{omm} и доказали их корректность. При этом отсутствует схема компиляции в модель архитектуры Power. А между тем данная архитектура часто используется в современном серверном оборудовании \cite{power-servers}. Задача построения такой схемы осложнена тем, что модель Power, в отличие от моделей x86-TSO, ARMv8 и $\OMM$, не обладает т.н. свойством \defin{multicopy atomicity}. Такое свойство означает, что записанные в память значения становятся доступны всем потокам в одном и том же порядке \cite{arm}. Из-за отсутствия этого свойства корректная схема компиляции $\OMM$ в Power должна расставлять барьеры в результирующей программе так, чтобы запретить нежелательные сценарии поведения. 

В рамках данной работы была поставлена задача разработать схему компиляции $\OMM$ в модель Power и доказать её корректность. Для этого было решено использовать промежуточную модель памяти (Intermediate Memory Model, далее — $\IMM$) \cite{imm}, для которой уже доказана корректность компиляции в модель Power. Использование $\IMM$ как промежуточного этапа компиляции позволяет разбить доказательство корректности  на два, которые впоследствии можно использовать в других доказательствах. Таким образом, построение схемы компиляции $\OMM$ в $\IMM$ даёт схемы компиляции $\OMM$ не только в Power, но и другие архитектуры, в которые компилируется $\IMM$ (на данный момент --- x86 и ARM). 


\section{Постановка задачи}

Целью данной работы является доказательство корректности компиляции модели памяти OCaml ($\OMM$) в модель памяти Power \cite{power}. 

В работе были поставлены следующие задачи:

\begin{itemize}
\item построение схемы компиляции $\OMM$ в $\IMM$ (для которой корректность компиляции в Power уже доказана);
\item доказательство корректности полученной схемы;
\item формализация доказательства в системе интерактивного доказательства теорем Coq \cite{coq-description}.
\end{itemize}

\section{Обзор}

В этом разделе приводится пример слабого поведения программы и объясняются его причины. Затем на примерах рассматривается понятие корректности компиляции для моделей памяти. Далее формально описывается декларативный способ задания модели памяти \cite{power}, основанный на понятии графов исполнения. Наконец, формально описываются модели памяти $\OMM$ и $\IMM$, используемые далее в работе.

\subsection{Пример исполнения в слабой модели памяти}

Рассмотрим программу, представленную на \cref{fig:store-buffering}. Здесь и далее используется упрощённый синтаксис программ: $x$ и $y$ обозначают адреса в памяти, $a$ и $b$ — локальные переменные (регистры), $\rlx$ – режим доступа (это понятие будет рассмотрено ниже). Сверху указаны изначальные значения в памяти. В комментариях указаны наблюдаемые при чтении значения. Согласно модели SC, в зависимости от порядка исполнения инструкций в $a$ и $b$ могут быть записаны значения $(1, 1)$, $(1, 0)$ или $(0, 1)$. Однако после компиляции C++-аналога этой программы с помощью компилятора gcc и исполнения на архитектуре x86 в переменные $a$ и $b$ могут быть записаны нули, что не допускается моделью SC. У такого сценария поведения могут быть две причины. Во-первых, gcc может поменять местами обращения по разным адресам во время компиляции. Во-вторых, при исполнении на x86 возможна буферизация записи: в целях оптимизации обращений к памяти запись может быть отложена. 

\begin{figure}[h]
  \centering
  \begin{tabular}{l || l}
    \multicolumn{2}{c}{$x = 0,\ y = 0$} \\
    \hline
    $\writeInst{x}{1}{\rlx}$ & $\writeInst{y}{1}{\rlx}$ \\
    $\readInst{a}{y}{0}{\rlx}$ & $\readInst{b}{x}{0}{\rlx}$ \\
  \end{tabular}
  \caption{Пример программы и её исполнения при буферизации записи}
  \label{fig:store-buffering}
\end{figure}

% Стоит отметить, что после оптимизации некоторых обращений к памяти поведение программы может перестать соответствовать спецификации.
Новые сценарии поведения программы, возникающие в результате оптимизаций, могут быть некорректными с точки зрения требований к программе. Поэтому слабая модель памяти должна предоставлять возможность отменить оптимизации для отдельных инструкций. Для этого используются \defin{режимы доступа}, различные по степени строгости.
% В зависимости от режима доступа инструкции при её компиляции могут быть отменены оптимизации, а также добавлены дополнительные инструкции-\defin{барьеры}, которые запрещают оптимизации на уровне процессора.
% Компилятор по-разному обрабатывает инструкции с разными режимами доступа; в частности, перед отдельными инструкциями могут быть
При компиляции инструкций с более строгими режимами компилятор может отменить некоторые оптимизации, а также добавить в результирующую программу инструкции-барьеры. 

Так, в программе на \cref{fig:store-buffering} был использован режим доступа $\rlx$, который не ограничивает оптимизации соответствующих инструкций. Модель памяти C++ гарантирует, что если в этой программе для всех инструкций установить режим доступа $\sco$ вместо $\rlx$, то поведение полученной программы будет согласовано с моделью SC. Это справедливо в силу того, что  такие обращения будут скомпилированы с использованием инструкции MFENCE \cite{cpp-mappings} --- барьера памяти, запрещающего перестановки SC инструкций.

\subsection{Проблема корректности компиляции на примерах} \label{corr-comp-example}

Модели $\OMM$ и $\IMM$ определены декларативно \cite{power}. Это означает, что каждое возможное поведение программы задаётся в виде \defin{графа исполнения}, а семантика программы определяется как множество графов, удовлетворяющих некоторому условию. Пример программы и одного из графов её исполнения приведён на \cref{fig:example-discriminating}.

%Метки $\na$ и $\at$ в программе на \cref{fig:example-discriminating} обозначают \defin{режимы доступа} соответствующих инструкций. Режимы доступа назначаются программистом для отмены оптимизаций отдельных инструкций. Так, запись в режиме $\at$ на архитектуре x86 компилируется\cite{omm} в инструкцию атомарной замены \texttt{xchg}, перед исполнением которой буфер записей очищается, что ограничивает множество возможных поведений программы.

Вершины графа соответствуют событиям --- операциям над разделяемой памятью, которые производятся при выполнении инструкций программы. Так, событие $\lW^{\at}(x, 1)$ соответствует записи по адресу $x$ значения $1$ в режиме $\at$; другими типами событий являются чтение и барьер памяти (обозначаются $\lR$ и $\lF$ соответственно). Кроме того, в графе выделяются инициализирующие события, которые соответствуют инициализирующей записи нулей в память. На \cref{fig:example-discriminating} все они для краткости обозначены множеством $\mathtt{Init}$; далее в графах мы будем опускать эти события, если это не будет важно для рассуждений. Заметим, что содержимое локальных переменных потока не отражается в графе в явном виде, т.к. взаимодействие между потоками производится только через разделяемую память. 

\newcommand{\discrOffset}{1.5}
\begin{figure}[h]
  \centering
  \begin{minipage}{0.45\textwidth}
    \centering
    \begin{tabular}{l || l || l}
      \multicolumn{3}{c}{$\writeInst{x}{0}{}\ \writeInst{y}{0}{}$} \\
      \hline
      $\writeInst{x}{1}{\at}$ & $\readInst{a}{x}{1}{\at}$ & $\readInst{b}{y}{1}{\na}$ \\
      {}            & $\writeInst{y}{1}{\na}$ & $\readInst{c}{x}{0}{\at}$ \\
    \end{tabular}
  \end{minipage}\hfill
  \begin{minipage}{0.45\textwidth}
    \centering
    \begin{tikzpicture}[yscale=1,xscale=1]
      \node (Winit) at (0,1) {$\mathtt{Init}$};
      
      \node (T1W1) at (-\discrOffset, -1.5) {$\wlab{\at}{x}{1}$};
      \dcoext{Winit}{T1W1}{bend right=20}{left};
      
      \node (T2R1) at (0,0) {$\rlab{\at}{x}{1}$};
      \drf{T1W1}{T2R1};
      \node (T2W1) at (0, -1.5) {$\wlab{\na}{y}{1}$};
      \dpo{T2R1}{T2W1};
      \dcoext{Winit}{T2W1}{bend right=45}{left};
      
      \node (T3R1) at (\discrOffset,0) {$\rlab{\na}{y}{1}$};
      \drf{T2W1}{T3R1};
      \node (T3R2) at (\discrOffset,-1.5) {$\rlab{\at}{x}{0}$};
      \dpo{T3R1}{T3R2}; 
      \drf{Winit}{T3R2};
      \dfr{T3R2}{T1W1}{bend left=20};
    \end{tikzpicture}
  \end{minipage}
  \caption{Пример программы и сценария её исполнения, не согласованного в $\OMM$}
  \label{fig:example-discriminating}
\end{figure}

Рёбра графа задают бинарные отношения между событиями. В данном графе есть четыре
различных отношения: рёбра $\lPO$ соответствуют программному порядку инструкций, $\lRF$ --- чтению записанного ранее значения, $\lCO$ --- порядку выполнения записи по одному адресу, $\lFR$ --- чтению до указанного события записи. Отношения $\lPO$ и $\lCO$ являются транзитивными, поэтому для их задания достаточно указывать только непосредственные рёбра. Кроме того, для краткости будем опускать подпись $\lPO$ рядом с соответствующими рёбрами.

\defin{Согласованными} (допустимыми моделью) называются те сценарии исполнения программы, графы которых удовлетворяют некоторому предикату, заданному моделью. В частности, предикат согласованности $\OMM$ требует, чтобы в графе не было циклов, состоящих только из рёбер $\lCO$ и $\lFR$, проходящих между вершинами с меткой $\at$, а также рёбер $\lPO$ и $\lRF$. Это условие формализует свойство multicopy atomicity, описанное выше.

Граф исполнения на \cref{fig:example-discriminating} не является согласованным по $\OMM$. Действительно, этот сценарий исполнения нарушает свойство multicopy atomicity: второй поток читает записанное в $x$ значение $1$ до записи $1$ в $y$, однако третий поток читает старое значение $0$ из $x$ после чтения $1$ из $y$. Соответствующий граф исполнения не удовлетворяет предикату согласованности $\OMM$, так как между вершинами есть цикл, подходящий под описание выше. Таким образом, в $\OMM$ после исполнения программы на \cref{fig:example-discriminating} переменные $a$, $b$ и $c$ не могут содержать значения $1$, $1$ и $0$ соответственно.

Условие корректности компиляции требует, чтобы сценарии поведения, запрещённые для исходной программы в $\OMM$, также были запрещены для скомпилированной программы в $\IMM$. Для декларативных моделей памяти это означает, что из несогласованности графа исполнения в $\OMM$ должна следовать несогласованность соответствующего ему графа исполнения  в $\IMM$. При этом, как будет показано далее, можно рассматривать вопрос согласованности по $\OMM$ только для графа исполнения скомпилированной программы.

На \cref{fig:example-imm-consistent} приведён результат компиляции программы на \cref{fig:example-discriminating} согласно тривиальной схеме компиляции. 
Такая схема лишь заменяет режимы инструкций на их аналоги в $\IMM$: $\na$ заменяется на $\rlx$, а $\at$ — на $\sco$; дополнительных инструкций не вводится. Соответственно, граф исполнения на \cref{fig:example-imm-consistent} отличается от графа на \cref{fig:example-discriminating} только метками вершин, и в нём сохраняется цикл того же вида. $\IMM$ не гарантирует свойство multicopy atomicity, и потому предикат её согласованности не требует отсутствия таких циклов. Поэтому граф на \cref{fig:example-imm-consistent} согласован, что делает соответствующий ему сценарий поведения разрешается $\IMM$, в отличие от $\OMM$. Поэтому тривиальная схема компиляции не является корректной.

\newcommand{\consOffset}{1.3}
\begin{figure}[h]
  \centering
  \begin{minipage}{0.5\textwidth}
    \centering
    \begin{tabular}{l || l || l}
      \multicolumn{3}{c}{$\writeInst{x}{0}{}\ \writeInst{y}{0}{}$} \\
      \hline
      $\writeInst{x}{1}{\sco}$ & $\readInst{a}{x}{1}{\sco}$ & $\readInst{b}{y}{1}{\rlx}$ \\
      {}            & $\writeInst{y}{1}{\rlx}$ & $\readInst{c}{x}{0}{\sco}$ \\
    \end{tabular}
  \end{minipage}\hfill
  \begin{minipage}{0.4\textwidth}
    \centering
    \begin{tikzpicture}[yscale=1,xscale=1]
      
      \node (T1W1) at (-\consOffset, -1.5) {$\wlab{\sco}{x}{1}$};
      
      \node (T2R1) at (0,0) {$\rlab{\sco}{x}{1}$};
      \drf{T1W1}{T2R1};
      \node (T2W1) at (0, -1.5) {$\wlab{\rlx}{y}{1}$};
      \dpo{T2R1}{T2W1};
      
      \node (T3R1) at (\consOffset,0) {$\rlab{\rlx}{y}{1}$};
      \drf{T2W1}{T3R1};
      \node (T3R2) at (\consOffset,-1.5) {$\rlab{\sco}{x}{0}$};
      \dpo{T3R1}{T3R2}; 
      % \dfr{T3R2}{T1W1}{bend left=20};
      \draw[fr, bend left=30] (T3R2) edge node[left, yshift=1ex] {$\lFR$}(T1W1);
    \end{tikzpicture}
  \end{minipage}
  \caption{Результат компиляции программы на \cref{fig:example-discriminating} с использованием тривиальной схемы компиляции и согласованный по $\IMM$ граф его исполнения}
  \label{fig:example-imm-consistent}
\end{figure}

На \cref{fig:example-imm-inconsistent} приведена программа, полученная в результате компиляции программы на \cref{fig:example-discriminating} согласно схеме компиляции, приведённой в разделе \ref{compilation-scheme}. В результате компиляции в ней появляются инструкции барьеров памяти, запрещающие некоторые оптимизации процессора и компилятора. С этими барьерами граф исполнения перестаёт быть согласованным: из рёбер $\lRF$ между событиями с меткой $\sco$, а также $\lFR$, $\lPO$ и окружённого барьерами $\lRF$ образуется цикл, запрещённый в $\IMM$.


\newcommand{\vsIV}{-1.2}
\newcommand{\hsIV}{2}

\newcommand{\inconsistentExample}[1]{
      \begin{tikzpicture}[yscale=1,xscale=1]
      
      \node (T1F1) at (0, 0) {$\flab{\acq}$};
      \node (T1R1) at (0, \vsIV * 1) {$\rlab{\sco}{x}{0}$};
      \dpo{T1F1}{T1R1};
      \node (T1W1) at (0, \vsIV * 2) {$\wlab{\sco}{x}{1}$};
      \dpo{T1R1}{T1W1};
      \drmw{T1R1}{T1W1};
      
      \node (T2F1) at (\hsIV,0) {$\flab{\acq}$};
      \node (T2R1) at (\hsIV,\vsIV * 1) {$\rlab{\sco}{x}{1}$};
      \dpo{T2F1}{T2R1};
      \drf{T1W1}{T2R1};
      \node (T2F2) at (\hsIV,\vsIV * 2) {$\flab{\acqrel}$};
      \dpo{T2R1}{T2F2};
      \node (T2W1) at (\hsIV,\vsIV * 3) {$\wlab{\rlx}{y}{1}$};
      \dpo{T2F2}{T2W1};
      
      \node (T3R1) at (\hsIV * 2,0) {$\rlab{\rlx}{y}{1}$};
      \drf{T2W1}{T3R1};
      \node (T3F1) at (\hsIV * 2,\vsIV * 1) {$\flab{\acq}$};
      \dpo{T3R1}{T3F1};
      \node (T3R2) at (\hsIV * 2,\vsIV * 2) {$\rlab{\sco}{x}{0}$};
      \dpo{T3F1}{T3R2};
      % \dhb{T2F2}{T3F1}{bend right=5};
      \dfr{T3R2}{T1W1}{bend left=#1};
    \end{tikzpicture}
  }

  \begin{figure}[!h]
    \centering
  % \begin{minipage}{0.6\textwidth}
    \begin{tabular}{l || l || l}
      \multicolumn{3}{c}{$\writeInst{x}{0}{}\ \writeInst{y}{0}{}$} \\
      \hline
      $\fenceInst{\acq}$ & $\fenceInst{\acq}$ & $\readInst{b}{y}{1}{\rlx}$ \\
      $\exchangeInstSC{x}{1}$ & $\readInst{a}{x}{1}{\sco}$ & $\fenceInst{\acq}$\\
      {} & $\fenceInst{\acqrel}$ & $\readInst{c}{x}{0}{\sco}$\\
      {} & $\writeInst{y}{1}{\rlx}$ & {} \\
    \end{tabular}
  % \end{minipage} \hfill
  % \begin{minipage}{0.3\textwidth}
    \inconsistentExample{20}
  % \end{minipage}
  \caption{Результат компиляции программы на \cref{fig:example-discriminating} с использованием схемы компиляции из раздела \ref{compilation-scheme} и его граф исполнения, не согласованный в $\IMM$}
  \label{fig:example-imm-inconsistent}    
\end{figure}

Таким образом, для доказательства корректности компиляции необходимо доказать
следующую теорему\footnote{Формальное понятие соответствие графов вводится в разделе \ref{compilation-scheme}.}.

\begin{restatable}{thrm}{compiletheorem}
  \label{prop:compile-theorem}
  Пусть $PO=||_{\tau\in \Tid} PO_\tau$ и $PI=||_{\tau\in \Tid} PI_\tau$ --- программы для моделей $\OMM$ и $\IMM$ соответственно, причём для любого $\tau \in \Tid$ подпрограмма $PI_\tau$ получена компиляцией $PO_\tau$ с помощью схемы компиляции из \cref{table:scheme}. Пусть $G_I$ --- согласованный по $\IMM$ граф исполнения $PI$. Тогда существует $G_O$ --- согласованный по $\OMM$ граф исполнения $PO$, соответствующий $G_I$. 
\end{restatable}


Ключевой идеей доказательства является то, что согласованность по $\OMM$ можно рассматривать для обоих графов исполнения. Для доказательства теоремы достаточно показать, что $G_I$ является согласованным по $\OMM$. Из этого следует согласованность $G_O$ по $\OMM$, так как он фактически является подграфом $G_I$ , а условия согласованности по $\OMM$ таковы, что выполняются для подграфов. Условие согласованности по $\OMM$ состоит в иррефлексивности одного отношения и ацикличности другого. Для каждого из этих отношений доказывается включение в такое отношение, для которого соответствующее условие выполняется в согласованном по $\IMM$ графе.

\subsection{Графы исполнения}
\label{exec-graphs}

В описаниях декларативных моделей памяти мы будем использовать следующие обозначения отношений между вершинами. Для бинарного отношения $R$ обозначения $R^?$, $R^{+}$, $R^{*}$ соответствуют его рефлексивному, транзитивному и транзитивно-рефлексивному замыканиям соответственно. Обратное отношение записывается как $R^{-1}$. Левая композиция отношений $R_1$ и $R_2$ записывается следующим образом:

$R_1;R_2 \defeq \set{x, y | \exists z. (x, z) \in R_1 \land (z, y) \in R_2}$.

Непосредственные рёбра $R$ обозначаются как $\imm R \defeq R \backslash R;R$. Тождественное отношение на множестве $A$ обозначается как $[A]$; в частности, $[A];R;[B] = R \cap (A \times B)$.

В данном разделе описываются графы исполнения наиболее общего вида, без привязки к конкретным моделям памяти или языкам. 

Считаем, что анализируемая программа $P$ состоит из последовательных подпрограмм отдельных потоков $P_\tau$: $P = ||_{\tau\in \Tid} P_\tau$, где $||$ --- оператор параллельной композиции программ, а $\Tid$ --- конечное множество идентификаторов потоков.

\begin{defn}
  \label{def:graph}
  \emph{Граф исполнения} $G$ задаётся множеством вершин, бинарными отношениями на вершинах, а также функцией, сопоставляющей вершинами \emph{метки}. 
\end{defn}
  
  Множество вершин, обозначаемое как $G.\lE$, делится на инициализирующие события вида $\mathtt{Init}\ loc$ и неинициализирующие события вида $\mathtt{ThreadEvent}\ \tau\ n$. Их компонентами являются:
  \begin{itemize}
  \item $loc \in \Loc$ --- адрес инициализации, где $\Loc$ --- конечное множество адресов;
  \item $\tau \in \Tid$ --- номер потока;
  \item $n\in \mathbb{N}$ --- порядковый номер внутри потока.
  \end{itemize}
    
Функция $G.\Lab$ сопоставляет событиям \term{метки} вида $(type, loc, mode, val)$. Их компонентами являются:
\begin{itemize}
\item $type \in \{\lR, \lW, \lF\}$ --- тип операции (чтение, запись, барьер);
\item $loc \in \Loc$ --- адрес памяти (для барьера не определено);
\item $mode$ --- один из режимов доступа (например, $\rel$), частично упорядоченных отношением ``строже чем'' ($\sqsubset$); конкретное множество режимов и их порядок определяется моделью памяти;
\item $val\in \Val$ --- прочитанное/записанное значение (в случае барьера не определено), где $\Val$ --- множество значений, которые могут храниться в памяти.
\end{itemize}

Следует отметить, что инициализирующие события вида $\mathtt{Init}\ loc$ обрабатываются особым образом. Именно,

$G.\Lab(\mathtt{Init}\ loc) = (\lW, loc, mode_{\Init}, val_{\Init})$, где $mode_{\Init}$ --- выбранный режим доступа для инициализирующих событий записей (например, в $\IMM$ --- $\rlx$), а $val_{\Init}$ --- начальное значение в памяти (как правило, 0).

Для множеств событий с определёнными метками вводятся соответствующие обозначения. Например, события с меткой чтения в режиме $\acq$ или более строгим будем обозначать как $G.\lR^{\acq}$ (или просто $\lR^{\acq}$, если граф очевиден из контекста).

Рёбра графа представляют собой следующие отношения между событиями:
\begin{itemize}
\item программный порядок (program order): $G.\lPO(x, y)\!\! \iff \!\! (x\in \Init \land y\notin \Init) \lor (x.\tau = y.\tau \land x.n<y.n)$;
\item порядок согласованности (coherence order): $G.\lCO = \bigcup_{l\in \Loc} \lCO_l$, где $\lCO_l$ --- тотальный порядок на событиях записи по адресу $l$;
\item наблюдение записанного значения (``читает-из'', reads from):

  $G.\lRF \subseteq \bigcup_{l\in \Loc} G.\lW_l \times G.\lR_l$, где

  $G.\lRF(w, r)\implies G.\Lab(w).val = G.\Lab(r).val$, $codom(G.\lRF) = G.\lR$ и

  $G.\lRF^{-1}$ является функциональным отношением;
\item чтение до указанной записи: $G.\lFR = G.\lRF^{-1}; G.\lCO$ (from-read, ``читает-\allowbreak до'').
\end{itemize}  

Различные модели памяти могут иметь в графе исполнения и другие отношения. Например, в модели $\IMM$ также есть отношение $\lRMW \subseteq \bigcup_{l\in \Loc}[G.\lR_l];\imm\lPO;[G.\lW_l]$, соответствующее паре событий чтения и записи в операции read-modify-write. 

Введём понятия сужения графа на поток $i$: $G_\tau.\lE = \{e\in G.\lE\ |\ e.\tau = i\}$, $G_\tau.\Lab = G.\Lab$.

\begin{defn}
  \label{def:execution}
% \begin{definition}
\term{Графом исполнения программы} $P$ называется такой граф $G$, что его сужение на любой поток $\tau\in \Tid$ является \term{однопоточным графом исполнения} программы $P_\tau$.
% \end{definition}
\end{defn}

Соответствие подпрограммы потока и однопоточного графа исполнения определяется средствами операционной семантики, специфичной для модели памяти \cite{omm}, \cite{imm}. Мы не приводим подробностей здесь, скажем лишь, что такая семантика задаёт соответствие между выполнением инструкций языка и изменением графа исполнения. Так, для $\IMM$ выполнение инструкции $[x]^{\rel}\;:=\;1$ соответствует добавлению в текущий граф вершины с очередным порядковым номером и меткой вида $\lW^\rel(x, 1)$, а также рёбер, отражающих синтаксические зависимости данного события.

\begin{defn}\label{def:outcome}
  \label{definition:outcome-def}
  Граф исполнения определяет \emph{сценарий поведения} программы --- функцию ${f: \Loc \to \Val}$, отображающую адрес в последнее (согласно порядку $\lCO$) записанное по нему значение. 
\end{defn}

Декларативная модель памяти задаётся предикатом согласованности, которому должны удовлетворять графы исполнения программ.
\begin{defn}\label{def:outcome}
  Сценарий поведения программы является \emph{согласованным} по модели памяти $M$, если он задан некоторым графом её исполнения, удовлетворяющим предикату согласованности $M$.
\end{defn}  

\subsection{Описание используемых моделей памяти}
\label{mm-description}

В данном разделе описываются рассматриваемые модели памяти ---  $\OMM$ и $\IMM$, --- а также их предикаты
согласованности.

\subsubsection{Модель памяти OCaml ($\OMM$)}
\label{ocaml-mm}

Модель памяти OCaml ($\OMM$) \cite{omm} задана эквивалентными операционным и декларативным описаниями. Для доказательства корректности компиляции будет использоваться декларативное описание.

$\OMM$ поддерживает два режима доступа: неатомарный $\na$ и атомарный $\at$ (схожи с $\pln$ и $\sco$ в C++). При этом память также разделена на неатомарные и атомарные адреса, и к конкретному адресу можно обратиться только операцией соответствующего режима.

В графе исполнения $\OMM$ есть только операции чтения и записи, барьеры отсутствуют.

Перед рассмотрением предиката согласованности введём ещё несколько обозначений. Для отношения $R$ в графе исполнения будем обозначать $Ri$ рёбра $R$, проходящие между вершинами одного потока, а $Re$ --- между вершинами разных потоков. 

\begin{defn}\label{def:omm-consistent}
% \begin{definition}
  Сценарий исполнения называется \term{согласованным по $\OMM$}, если в соответствующем графе исполнения выполняются следующие аксиомы:

  \begin{enumerate}
  \item последовательная согласованность по отдельным адресам (SC per location, coherence): отношение $\lHBO ; (\lCO \cup \lFR)$ иррефлексивно, где

    $\lHBO \defeq \lPO \cup [\lE^{\at}] ; (\lRF \cup \lCO) ; [\lE^{\at}]$;
    
  \item отсутствие буферизации при чтении (load buffering): отношение $\lPO \cup \lRFE \cup [\lE^{\at}] ; (\lCOE \cup \lFRE) ; [\lE^{\at}]$ ациклично.
  \end{enumerate}
% \end{definition}
\end{defn}

\subsubsection{Промежуточная модель памяти}

Промежуточная модель памяти ($\IMM$) определена декларативно. Полный предикат согласованности $\IMM$ достаточно сложен, поэтому мы рассмотрим лишь часть модели, которая будет необходима для построения схемы компиляции. Перед этим введём ещё несколько обозначений. $R_{loc}$ будем обозначать рёбра $R$, проходящие между вершинами с метками одного и того же адреса, $R_{\neq loc}$ --- между вершинами с метками разных адресов.

Синтаксис программ на $\IMM$ напоминает таковой в C++ --- помимо инструкций атомарного чтения и записи есть инструкции барьеров памяти, а также операций read-modify-write. В граф исполнения программы на $\IMM$, помимо отношений, перечисленных в разделе \ref{exec-graphs}, также входит отношение, связывающее события чтения и записи, которые совершаются при операции read-modify-write:

$G.\lRMW \subseteq ([G.\lR] ;\imm \lPO;[G.\lW])_{loc}$. 

Для построения схемы компиляции мы пользуемся расширением \cite{imm-sc} $\IMM$, которое дополняет оригинальную модель \cite{imm} $\sco$-операциями. 

\begin{defn}\label{def:imm-consistent}
  % \begin{definition}
  Сценарий исполнения называется \term{согласованным по $\IMM$}, если в соответствующем графе исполнения выполняются следующие аксиомы:
  \begin{enumerate}
  \item отношение $\lHB ; (\lRF \cup \lCO \cup \lFR)^{+}$ иррефлексивно, при этом справедливо следующее:
    
  $\lHB \defeq (\lPO \cup \lSW)^{+}$

  $\lSW \defeq \mathtt{\color{blue}release};(\lRFI \cup \lPO^?_{loc};\lRFE);([\lR^{\acq}] \cup \lPO;[\lF^{\acq}])$

  $\mathtt{\color{blue}release} \defeq ([\lW^{\rel}] \cup [\lF^{\rel}];\lPO);\mathtt{\color{blue}rs}$

  $\mathtt{\color{blue}rs} \defeq [\lW];\lPO_{loc};[\lW] \cup [\lW];(\lPO^?_{loc};\lRFE;\lRMW)^{*}$;
\item операции read-modify-write являются атомарными:

  $\lRMW \cap (\lFRE ; \lCOE) = \emptyset$;
  \item отношение $\lAR$ ациклично, при этом справедливо следующее:

    $\lAR \supset \lRFE \cup \lBOB$

    $\lBOB \supset [\lR^{\acq}] ; \lPO \cup \lPO ; [\lF] \cup [\lF] ; \lPO$;
  \item отношение $\lPSCB$ ациклично, при этом справедливо следующее:

    $\lPSCB \defeq ([\lE^\sco] \cup{} [\lF^\sco];\lHB^?);\lSCB;([\lE^\sco] \cup{} \lHB^?;[\lF^\sco])$

    $\lSCB \defeq \lPO \cup \lPO_{\neq loc} ; \lHB ; \lPO_{\neq loc} \cup \lHB_{loc} \cup \lCO \cup \lFR$.

  \end{enumerate}
% \end{definition}
\end{defn}


\section{Обоснование схемы компиляции}
\label{compilation-scheme}

\subsection{Необходимость введения дополнительных инструкций}
\label{extra-instrs}

Ранее в разделе \ref{corr-comp-example} был рассмотрен пример программы, компиляция которой требует вставки дополнительных инструкций. Для удобства граф её исполнения приведён здесь на \cref{fig:example-imm-inconsistent-copy}. Именно, данный граф показывает, что ребро $\lRF e$ должно быть окружено барьерами $\rel$ и $\acq$, поэтому инструкции неатомарной записи и атомарного чтения компилируются с использованием указанных барьеров.


\renewcommand{\hsIV}{2}
\begin{figure}[h]
  \centering
  \begin{minipage}{0.9\textwidth}
    \centering
    \inconsistentExample{20}
  \end{minipage}
  \caption{Ребро $\lRF e$ между вершинами с режимом $\rlx$ должно быть окружено барьерами}
  \label{fig:example-imm-inconsistent-copy}
\end{figure}


Однако видно, что результирующая программа на \cref{fig:example-imm-inconsistent} получена с использованием схемы компиляции, которая вставляет в программу и другие инструкции. Так, барьер перед инструкцией неатомарной записи по адресу $y$ имеет более строгий режим $\acqrel$. Более того, инструкция атомарной записи компилируется в инструкцию read-modify-write, предварённую $\acq$-барьером.

Приведённые ниже примеры демонстрируют, какие нежелательные сценарии поведения запрещают барьеры и инструкции read-modify-write. Для простоты мы рассмотрим эти случаи по отдельности, т.е. не будем вставлять барьеры в граф, содержащий read-modify-write и наоборот. 

Использование барьеров позволяет обеспечить гарантируемое $\OMM$ свойство multicopy atomicity для атомарных адресов. Это означает отсутствие в графе исполнения скомпилированной программы циклов, состоящих из рёбер $\lFR$, проходящих между вершинами с меткой $\sco$, а также рёбер $\lPO$ и $\lRF$. $\IMM$ в общем случае допускает такие циклы. Поскольку отношение $\lAR$ в $\IMM$ не учитывает рёбра $\lFR$, то с целью запрета подобного сценария поведения цикл вида $[\lW^\at];(\lPO; \lRF)^{+};\lPO;[\lR^\at];\lFR$ необходимо представить как $\lHB^{+};\lFR$. Для этого каждое ребро $\lRF$ между неатомарными вершинами необходимо окружить барьерами. Кроме того, так как после подобного ребра может идти $\lRF$ между атомарными вершинами, перед $\lW^\at$ необходимо также расположить $\acq$-барьер. Наконец, $\acq$-барьер необходимо расположить и перед ребром $\lFR$, т.е. перед событием атомарного чтения. \cref{fig:fences-hb} показывает последовательность рёбер $\lRF$ между вершинами различных типов, иллюстрирующую необходимость расположения барьеров. 

\newcommand{\offsetfive}{2.0}
\begin{figure}[h]
  \centering
  \begin{tikzpicture}[yscale=1,xscale=1]
    \node (T11) at (0 * \offsetfive, 0) {$\wlab{\sco}{x}{1}$};
    \node (T12) at (0 * \offsetfive, -1.5) {$\flab{\acqrel}$};
    \node (T13) at (0 * \offsetfive, -3) {$\wlab{\rlx}{y}{1}$};
    \dpo{T11}{T12}; \dpo{T12}{T13};
    
    \node (T21) at (1 * \offsetfive, 0) {$\rlab{\rlx}{y}{1}$};
    \node (T22) at (1 * \offsetfive, -1.5) {$\flab{\acq}$};
    \node (T23) at (1 * \offsetfive, -3) {$\wlab{\sco}{z}{1}$};
    \dpo{T21}{T22}; \dpo{T22}{T23};
    \drfext{T13}{T21}{}{left, yshift=2ex, xshift=2ex};
    \dhbext{T12}{T22}{bend right=20}{yshift=-1ex};
    
    \node (T31) at (2 * \offsetfive, 0) {$\rlab{\sco}{z}{1}$};
    \node (T32) at (2 * \offsetfive, -1.5) {$\flab{\acqrel}$};
    \node (T33) at (2 * \offsetfive, -3) {$\wlab{\rlx}{u}{1}$};
    \dpo{T31}{T32}; \dpo{T32}{T33};
    \drfext{T23}{T31}{}{left, yshift=2ex, xshift=2ex};
    \dhbext{T23}{T31}{bend right=5}{yshift=-3ex};
    
    \node (T41) at (3 * \offsetfive, 0) {$\rlab{\rlx}{u}{1}$};
    \node (T42) at (3 * \offsetfive, -1.5) {$\flab{\acqrel}$};
    \node (T43) at (3 * \offsetfive, -3) {$\wlab{\rlx}{v}{1}$};
    \dpo{T41}{T42}; \dpo{T42}{T43};
    \drfext{T33}{T41}{}{left, yshift=2ex, xshift=2ex};
    \dhbext{T32}{T42}{bend right=20}{yshift=-1ex};
    
    \node (T51) at (4 * \offsetfive, 0) {$\rlab{\rlx}{v}{1}$};
    \node (T52) at (4 * \offsetfive, -1.5) {$\flab{\acq}$};
    \node (T53) at (4 * \offsetfive, -3) {$\rlab{\sco}{x}{1}$};
    \dpo{T51}{T52}; \dpo{T52}{T53};
    \drfext{T43}{T51}{}{left, yshift=2ex, xshift=2ex};
    \dhbext{T42}{T52}{bend right=20}{yshift=-1ex};
    \dfrext{T53}{T11}{bend right=40}{right, xshift=-1ex, yshift=1ex};
  \end{tikzpicture}
  \caption{Вставка барьеров позволяет провести рёбра $\lHB$ вдоль рёбер $\lRF$ и запретить указанный сценарий поведения}
  \label{fig:fences-hb}
\end{figure}

Компиляция инструкций атомарной записи в инструкции $\mathtt{exchange}$ позволяет, согласно требованиям $\OMM$, связать отношением $\lHB$ события атомарной записи, упорядоченные по $\lCO$. Вставка барьеров в этом случае не поможет, т.к. с помощью них можно провести $\lHB$ лишь вдоль рёбер $\lRF$. Однако такое ребро можно ввести искусственно, добавив перед событием записи событие чтения, которое наблюдает $\lCO$-предыдущую запись. Такое условие обеспечивается инструкцией $\mathtt{exchange}$. Она порождает в графе пару вершин, связанных отношением $\lRMW$, что иллюстрируется на \cref{fig:rmw-hb}.

\newcommand{\offsetthree}{6}
\begin{figure}[h]
  \centering
  \begin{minipage}{0.9\textwidth}
    \centering
    \begin{tikzpicture}[yscale=1,xscale=1]
      \node (T11) at (0 * \offsetthree, 0) {$\rlab{\rlx}{x}{1}$};
      \node (T12) at (0 * \offsetthree, -1.5) {$\rlab{\sco}{y}{0}$};
      \node (T13) at (0 * \offsetthree, -3) {$\wlab{\sco}{y}{1}$};
      \dpo{T11}{T12}; \dpo{T12}{T13};
      \drmw{T12}{T13};
      
      \node (T21) at (1 * \offsetthree, 0) {$\rlab{\sco}{y}{1}$};
      \node (T22) at (1 * \offsetthree, -1.5) {$\wlab{\sco}{y}{2}$};
      \node (T23) at (1 * \offsetthree, -3) {$\wlab{\rlx}{x}{1}$};
      \dpo{T21}{T22}; \dpo{T22}{T23};
      \drmw{T21}{T22};
      \dco{T13}{T22};
      \drf{T23}{T11};
      \drf{T13}{T21}; \dhb{T13}{T21}{bend left=30};

      % \dhb{T12}{T22}{bend right=20};
    \end{tikzpicture}
  \end{minipage}
  \caption{Реализация атомарной записи инструкцией read-modify-write позволяет провести ребро $\lHB$ и запретить указанный сценарий поведения}
  \label{fig:rmw-hb}
\end{figure}



\subsection{Схема компиляции}

\cref{table:scheme} содержит предлагаемую схему компиляции $\OMM$ в $\IMM$. За основу взята схема компиляции $\OMM$ в модель $\mathtt{ARMv8}$ из \cite{omm}.

Как описано в разделе \ref{extra-instrs}, дополнительные инструкции используются для того, чтобы запретить для скомпилированной программы сценарии поведения, разрешённые $\IMM$ и запрещённые $\OMM$. 

\begin{table}[h]
  \centering
  \begin{tabular}{ | c | c | c| }
    \hline
    $\OMM$ & $\IMM$ & $\mathtt{ARMv8}$ \\
    \hline
    $\readInst{r}{x}{}{\na}$ & $\readInst{r}{x}{}{\rlx}$ & $\readInst{r}{x}{}{\rlx}$  \\
    $\writeInst{x}{v}{\na}$ & $\fenceInst{\acqrel} \writeInst{x}{v}{\rlx}$ & $\fenceInst{\acq}  \writeInst{x}{v}{\rlx}$\\
    $\readInst{r}{x}{}{\at}$ & $\fenceInst{\acq} \readInst{r}{x}{}{\sco}$ & $\fenceInst{\acq} \readInst{r}{x}{}{\sco}$\\
    $\writeInst{x}{v}{\at}$ & $\fenceInst{\acq} \exchangeInstSC{x}{v}$ & $\exchangeInstSC{x}{v} \fenceInst{\rel}$\\
    \hline
  \end{tabular}
  \caption{Схема компиляции $\OMM$ в $\IMM$ и сравнение её со схемой компиляции $\OMM$ в $\mathtt{ARMv8}$}
  \label{table:scheme}
\end{table}

\section{Доказательство корректности компиляции}
\label{proof}
\compiletheorem*
\begin{proof}
  В теореме \ref{corresponding-existence} доказывается, что существует $G_O$, являющийся графом исполнения $PO$ и соответствующий $G_I$ (формальное понятие соответствия графов рассматривается в разделе \ref{graph-correspondence}).
  
  Далее, в теореме \ref{graph-replacement} показывается, что согласованность по $\OMM$ $G_O$ следует из согласованности $G_I$ по $\OMM$. Затем в теоремах \ref{corr-coherence-thm} и \ref{corr-causality-thm} доказываются два условия согласованности $G_I$ по $\OMM$. 
\end{proof}
% \label{correctness-proof}

% \compiletheorem*

% \begin{proof}
% В \cref{graph-correspondence} показывается, что согласованность по $\OMM$ $G_O$ следует из согласованности $G_I$ по $\OMM$. Затем в \cref{corr-coherence}
% и \cref{corr-causality} доказываются два условия согласованность $G_I$ по $\OMM$. 
% \end{proof}
% Рядом с формулировками утверждений указаны названия их аналогов в доказательстве на Coq \todo{указать}.

\subsection{Соответствие графов исполнения}
\label{graph-correspondence}

% Одно и то же поведение в $\OMM$ и $\IMM$ выражается схожими графами.
Необходимо точно определить соответствие графов исполнения, чтобы один и тот же сценарий исполнения программы можно было представить как в $\IMM$, так и в $\OMM$. Отметим, что графы исполнения в этих моделях различны, т.к. в результате компиляции в программу добавляются новые инструкции, а в граф --- новые вершины. Но эти различия не меняют сценарий поведения программы согласно определению \ref{definition:outcome-def}. 

Чтобы формализовать соответствие графов исполнения, опишем, чем отличаются множества вершин и рёбер в графах исполнения скомпилированной программы и исходной. Вершины графа исполнения в $\OMM$ --- те же, что в графе $\IMM$, за исключением вершин-барьеров, а также операций чтения, выполняемых в ходе операции read-modify-write. При удалении этих вершин также удаляются смежные им рёбра. Метки оставшихся событий в $\OMM$ совпадают с таковыми в $\IMM$ с точностью до переименования режимов доступа. 

\begin{defn}\label{def:compiled}
  % \begin{definition}
  Граф исполнения по $\OMM$ $G_O$ \term{соответствует} графу исполнения по $\IMM$ $G_I$, если выполняются следующие условия:
  \begin{enumerate}
  \item $G_O.\lE = G_I.\lE \cap (G_I.\lR \cup G_I.\lW \setminus dom(G_I.\lRMW))$
  \item $\forall e.\ e \in G_O.\lE \implies G_O.\Lab\ e = \mathtt{renameMode}(G_I.\Lab\ e)$, где $\mathtt{renameMode}$ меняет метку вершины с $\sco$ на $\at$ и с $\rlx$ на $\na$
  \item $G_O.\lCO = [G_O.\lE]; G_I.\lCO; [G_O.\lE]$
  \item $G_O.\lRF = [G_O.\lE]; G_I.\lRF; [G_O.\lE]$
  \end{enumerate}
\end{defn}

\begin{thrm} \label{graph-replacement}
  Пусть $G_O$ и $G_I$ --- пара соответствующих графов. Тогда из согласованности $G_I$ по $\OMM$ следует согласованность $G_O$ по $\OMM$. 
\end{thrm}
\begin{proof}
  Условие согласованности по $\OMM$ требует ацикличности и иррефлексивности отношений, построенных из рёбер $\lPO$, $\lRF$, $\lCO$ и $\lFR$. Заметим, что $G_O.\lE \subseteq G_I.\lE$, так как в $G_O.E$ отсутствуют вершины, соответствующие барьерам и операциям read-modify-write, а новых вершин в $G_O$ не вводится. Кроме того, при удалении вершин из графа удаляются и смежные с ними рёбра, поэтому выполнено следующее: $\forall r \in \set{\lPO, \lRF, \lCO, \lFR}.\ G_O.r \subseteq G_I.r$.

  Если некоторое отношение ациклично (иррефлексивно), то и любое включённое в него отношение также ациклично (иррефлексивно). Поэтому из согласованности $G_I$ по $\OMM$ (при замене режимов доступа $\IMM$ на их аналоги в $\OMM$) следует согласованность $G_O$ по $\OMM$.
\end{proof}

\subsection{Построение графа, соответствующего данному}

Пусть $PO=||_{\tau\in \Tid} PO_\tau$ и $PI=||_{\tau\in \Tid} PI_\tau$ --- программы для моделей $\OMM$ и $\IMM$ соответственно, причём для любого $\tau \in \Tid$ подпрограмма $PI_\tau$ получена компиляцией $PO_\tau$ с помощью схемы компиляции из \cref{table:scheme}. Пусть $G_I$ --- согласованный по $\IMM$ граф исполнения $PI$.

Следующая теорема практически повторяет теорему \ref{prop:compile-theorem}, за исключением требования на согласованность по $\OMM$ искомого графа. 

\begin{thrm} \label{corresponding-existence}
  Существует граф $G_O$, который задаёт сценарий исполнения $PO$ и соответствует $G_I$. 
\end{thrm}
\begin{proof}
  Построим $G_O$ по лемме \ref{corresponding-existence-ext}. 
  
  Как следует из формулировки данной леммы, сужение этого графа на каждый из потоков является однопоточным графом исполнения соответствующей подпрограммы, поэтому $G_O$ является графом исполнения $PO$. Осталось доказать соответствие $G_O$ и $G_I$. 

  Условия на соответствие отношений $\lRF$ и $\lCO$ выполняются непосредственно по формулировке леммы \ref{corresponding-existence-ext}.

  Заметим, что условия на соответствие множеств вершин $G_O.\lE$ и $G_I.\lE$ и функций $G_O.\Lab$ и $G_I.\Lab$ можно доказывать для сужений графов на потоки:  эти условия выполняются для сужения на любой поток тогда и только тогда, когда они выполняются и во всём графе. По формулировке леммы \ref{corresponding-existence-ext} сужения этих графов на любой поток соответствуют друг другу. 
\end{proof}

\begin{lm} \label{corresponding-existence-ext}
  Существует граф $G_O$, для которого справедливы следующие утверждения:
  \begin{itemize}
  \item $G_O.\Init=G_I.\Init$;
  \item $G_O.\lCO = [G_O.\lE]; G_I.\lCO; [G_O.\lE]$;
  \item $G_O.\lRF = [G_O.\lE]; G_I.\lRF; [G_O.\lE]$;
  \item для всех $\tau \in \Tid$ сужение $G_{O\tau}$ соответствует $G_{I\tau}$, а также является графом исполнения подпрограммы $PO_\tau$. 
  \end{itemize}
\end{lm}
\begin{proof}
  Искомый граф построим как объединение однопоточных графов исполнения, полученных по лемме \ref{corresponding-existence-single}. К этому объединению добавим множество инициализирующих вершин $G_I.\Init$. Отношение $\lCO$ и $\lRF$ зададим как таковые в $G_I$, ограничив их на множества вершин в $G_O$. 
\end{proof}

\begin{lm} \label{corresponding-existence-single}
  Для всех $\tau\in Tid, PO_\tau, PI_\tau, G_{I\tau}$, где $G_{I\tau}$ является сужением $G_I$ на поток $\tau$, существует такой граф $G_{O\tau}$, который является графом исполнения для программы $PO_\tau$ и соответствует $G_{I\tau}$. 
\end{lm}
\begin{proof}
  Сужение $G_{I\tau}$ является однопоточным графом исполнения для соответствующего потока, так как $G_{I}$ --- граф исполнения для программы $PI$. Как упоминалось в разделе \ref{exec-graphs}, для задания этого используется операционная семантика, которая при исполнении инструкций потока строит соответствующий граф. Состояние абстрактной машины такой семантики содержит текущий граф, значения регистров и указатель на текущую инструкцию. 

  Для упрощения доказательства сгруппируем переходы абстрактной машины, соответствующие исполнению блоков инструкций, каждый из которых получен компиляцией одной инструкции в $PO_\tau$. Мы получаем блочные состояния и  переходим к т.н.  блочному исполнению, в ходе которого выполнение блока переходов рассматривается как неделимый переход.

  Исходное утверждение доказывается индукцией. Именно, для любого числа блочных переходов и результирующего состояния $\IMM$-машины можно исполнить столько же обычных переходов исполнения $PO_\tau$ и получить состояние $\OMM$-машины, соответствующее заданному.
\end{proof}


\subsection{Доказательство последовательной согласованности по отдельным адресам}
\label{corr-coherence}

\begin{thrm} \label{corr-coherence-thm}
  Отношение $\lHBO ; (\lCO\cup \lFR)$ является иррефлексивным. 
\end{thrm}
Докажем, что для выбранной схемой компиляции справедливо $\lHBO \subseteq \lHB$. С учётом этого доказательство теоремы тривиально: по определению согласованности по $\IMM$, отношение $\lHB ; (\lRF \cup \lCO \cup \lFR)$ иррефлексивно. 

Для этого факта сначала покажем, что последовательность рёбер $\lCO$ между $\sco$-событиями порождает $\lHB$.
% \begin{restatable}{thm}{sc-co-hb}

\begin{lm}
  \label{sc-co-hb}
  Порядок событий $\sco$-записи согласуется с отношением happens-before: $[\lE^{\sco}] ; \lCO ; [\lE^{\sco}] \subseteq \lHB$. 
\end{lm}
%\end{restatable}

\begin{proof}
  Заметим, что  $[\lE^{\sco}] ; \lCO ; [\lE^{\sco}]$ транзитивно. Тогда можно перейти к рассмотрению непосредственных $\lCO$-соседей. 

  Предположим, что $[\lE^{\sco}] ; \lCO_{imm} ; [\lE^{\sco}] \subseteq [\lE^{\sco}] ; \lRF ; [\lE^{\sco}] ; \lPO$. Тогда доказательство тривиально: $\lRF$ по $\sco$ событиям порождает $\lHB$, как и следующий за ним $\lPO$. Значит, остаётся доказать утверждение о включении в $[\lE^{\sco}] ; \lRF ; [\lE^{\sco}] ; \lPO$.

  Рассмотрим два события $w_1, w_2 \in \Wsc$ --- непосредственных $\lCO$-соседей. По схеме компиляции перед $w_2$ следуют $f\in \Fa$ и $r \in \Rsc$, причём $\lRMW(r, w_2)$.

  Покажем, что $\lRF(w_1, r)$. Рассмотрим событие записи $w'$, из которого читает $r$. Обращения по одному и тому же адресу имеют один и тот же режим, поэтому $w' \in \lW^\sco$. 
  Пусть $w_1\neq w'$. Тогда либо $\lCO(w', w_1)$, либо, наоборот, $\lCO(w_1, w')$. В первом случае нарушается атомарность $\lRMW$ между $w_2$ и $r$. Во втором случае получается, что между $\lCO$-соседями $w_1$ и $w_2$ расположен $w'$, что невозможно. 
\end{proof}

\begin{lm}
  \label{hbo-in-hb}
  Отношение happens-before в $\IMM$ содержит happens-before в $\OMM$: $\lHBO \subseteq \lHB$. 
\end{lm}
\begin{proof}
  $\lHBO \defeq \lPO \cup [\lE^{\sco}] ; (\lCO \cup \lRF) ; [\lE^{\sco}]$. По предыдущей лемме $\lCO$, ограниченный на $\sco$, входит в $\lHB$. $\lRF$ по $\sco$ событиями порождает $\lSW$, и, следовательно, $\lHB$. 
\end{proof}

\subsection{Доказательство отсутствия буферизации при чтении}
\label{corr-causality}
Сначала докажем утверждения, которые позволят нам находить барьеры в программном порядке между событиями.

\begin{lm}  \label{sb-w-sync}
  Перед событием записи располагается барьер:

  $[\lE \backslash \lF] ; \lPO ; [\lW] \subseteq \lPO ; ([\Far] ; \lPO ; [\lE^{\rlx}] \cup [\Fa] ; \lPO ; [\lE^{\sco}]) ; [\lW] \cup \lRMW$.
\end{lm}
\begin{proof}
  Согласно схеме компиляции, инструкция барьера располагается либо непосредственно перед инструкцией неатомарной записи, либо перед инструкцией read-modify-write. Таким образом, барьера между событием и $\lPO$-следующим событием записи может не быть, только если это событие чтения в $\lRMW$.
\end{proof}

\begin{lm}  \label{sb-sc-sync}
  В программном порядке между событиями $\rlx$ и $\sco$ располагается барьер:  $[\lE^{\rlx}] ; \lPO ; [\lE^{\sco}] \subseteq \lPO ; [\Fa] ; \lPO$.
\end{lm}
\begin{proof}
  Согласно схеме компиляции, все инструкции в режиме $\sco$ предваряются $\acq$-барьерами. 
\end{proof}

Кроме того, нам понадобятся следующие факты из алгебры \cite{hahn-repo}.

\begin{lm} \label{ct-decompose1}
  Цикл из рёбер двух типов можно представить в виде чередующихся участков рёбер каждого типа: $(\exX \cup \exY)^{+} = \exY^{+} \cup \exY^{*} ; (\exX ; \exY^{*})^{+}$, где $\exX$, $\exY$ --- произвольные отношения.
\end{lm}

\begin{lm} \label{ct-decompose2}
  Отношение $\exX\cup \exY$ является ацикличным, если ацикличны отношения $\exX$, $\exY$ и $\exX^{+};\exY^{+}$.
\end{lm}

\begin{thrm} \label{corr-causality-thm}
  Отношение $\lPO \cup \lRFE \cup [\lE^{\sco}] ; (\lCOE \cup \lFRE) ; [\lE^{\sco}]$ является ацикличным. 
\end{thrm}

Сгруппируем первые два отношения в объединении. Тогда по лемме \ref{ct-decompose2} нужно показать ацикличность следующих отношений:
\begin{itemize}
\item $\lPO \cup \lRFE$;
\item $[\lE^{\sco}] ; (\lCOE \cup \lFRE) ; [\lE^{\sco}]$;
\item $(\lPO \cup \lRFE)^{+} ; ([\lE^{\sco}] ; (\lCOE \cup \lFRE) ; [\lE^{\sco}])^{+}$, и это эквивалентно ацикличности:  $[\lE^{\sco}];(\lPO \cup \lRFE)^{+} ; [\lE^{\sco}] ; ([\lE^{\sco}] ; (\lCOE \cup \lFRE) ; [\lE^{\sco}])^{+}$. 
\end{itemize}

% Второе и третье утверждение докажем, показав, что соответствующие отношения лежат в $([\lE^\sco] ; \lSCB ; [\lE^\sco])^{+}$. Ацикличность этого отношения следует из $\IMM$-согласованности графа: $\lPSCB$ ациклично, где, напомним,
Докажем второе и третье утверждение показав, что соответствующие отношения лежат в $([\lE^\sco] ; \lSCB ; [\lE^\sco])^{+} \subseteq \lPSCB^{+}$. Напомним, что

$\lSCB \defeq \lPO \cup \lPO_{\neq loc} ; \lHB ; \lPO_{\neq loc} \cup \lHB_{loc} \cup \lCO \cup \lFR$ и

$\lPSCB \defeq ([\lE^\sco] \cup [\lF^\sco];\lHB^?);\lSCB;([\lE^\sco] \cup{} \lHB^?;[\lF^\sco])$. В свою очередь, ацикличность $\lPSCB$ следует из согласованности графа по $\IMM$. 

Теперь видно, что второе утверждение верно по определению $\lSCB$. По этой же причине для доказательства третьего утверждения остаётся показать, что $[\lE^{\sco}];(\lPO \cup \lRFE)^{+} ; [\lE^{\sco}] \subseteq ([\lE^\sco] ; \lSCB ; [\lE^\sco])^{*}$. 


\begin{thrm} \label{acyclic-po-rfe}
Отношение $\lPO \cup \lRFE$ ациклично.
\end{thrm}
\begin{proof}
%Вновь воспользуемся леммой \ref{ct-decompose2} и разложим условие ацикличности объединения на ацикличность отношений $\lPO$ (следует из согласованности по $\IMM$) и $\lRFE$ (двух и более таких рёбер подряд идти не может, т.к. их концы имеют разные типы), а также $\lPO^{+} ; \lRFE^{+}$, что эквивалентно ацикличности $\lPO ; \lRFE$.
Вновь воспользуемся леммой \ref{ct-decompose2} и разложим условие ацикличности объединения на ацикличность отношений $\lPO$, $\lRFE$ и $\lPO^{+} ; \lRFE^{+}$. Ацикличность первого следует из согласованности по $\IMM$, двух и более рёбер второго подряд идти не может (т.к. их концы имеют разные типы), а ацикличность третьего эквивалентна ацикличности $\lPO ; \lRFE$. 

  Пусть такой цикл существует. Покажем, что это противоречит условию ацикличности $\lAR$ (следует из согласованности по $\IMM$). Напомним, что $\lAR \supset \lRFE \cup \lBOB$ и $\lBOB \supset [\lR^{\acq}] ; \lPO \cup \lPO ; [\lF] \cup [\lF] ; \lPO$\ .

  По лемме \ref{sb-w-sync} перед событием записи, которой начинается ребро $\lRFE$, есть барьер $\lF^{\sqsupseteq \acq}$, либо весь $\lPO$ является $\lRMW$. В первом случае внутри $\lPO$ есть барьер, а такое отношение лежит в $\lBOB \subseteq \lAR$. Во втором случае $\lRMW$ начинается с $\Rsc$, и такое ребро $\lPO \supseteq \lRMW$ также содержится в $\lBOB$. Наконец, $\lRFE \subseteq \lAR$. 
\end{proof}

Теперь для доказательства второго условия согласованности по $\OMM$ осталось доказать утверждение $[\lE^{\sco}];(\lPO \cup \lRFE)^{+} ; [\lE^{\sco}] \subseteq ([\lE^\sco] ; \lSCB ; [\lE^\sco])^{*}$. 

\begin{thrm} \label{sc-po-rfe-pscb}
  Последовательность из рёбер $\lPO$ и $\lRFE$ между вершинами $\sco$ состоит из рёбер $\lSCB$ между вершинами $\sco$: $[\lE^{\sco}];(\lPO \cup \lRFE)^{+} ; [\lE^{\sco}] \subseteq ([\lE^\sco] ; \lSCB ; [\lE^\sco])^{*}$.
\end{thrm}
\begin{proof}

  Сначала сформулируем утверждение, которое позволит отбрасывать $\lRFE$-рёбра. 

  \begin{lm} \label{sc-rf-scb}
    Рёбра $\lRFE$, у которых один из концов --- $\sco$, входят в $\lSCB$: $[\lE^{\sco}];\lRFE \cup [\lW\backslash \Init];\lRFE;[\lE^{\sco}] \subseteq [\lE^\sco] ; \lSCB ; [\lE^\sco]$. 
  \end{lm}
  \begin{proof}    
    Если один из концов ребра $\lRF$ является $\sco$, таким же является и второй (исключение --- чтение из инициализирующих записей, которые в $\IMM$ являются $\rlx$).
    Тогда $[\lE^{\sco}];\lRF;[\lE^{\sco}] \subseteq [\lE^{\sco}];\lHB_{loc};[\lE^{\sco}] \subseteq [\lE^{\sco}];\lSCB;[\lE^{\sco}]$.
  \end{proof}
  
  По лемме \ref{ct-decompose1} имеем следующее:

  $[\lE^{\sco}];(\lPO \cup \lRFE)^{+};[\lE^{\sco}] = [\lE^{\sco}];(\lRFE^{+} \cup \lRFE^{*} ; (\lPO ; \lRFE^{*})^{+});[\lE^{\sco}]=[\lE^{\sco}];(\lRFE \cup \lRFE^{?} ; (\lPO ; \lRFE^{?})^{+}); [\lE^{\sco}]$.

  Начальные участки $\lRFE$, если они есть, можно отбросить по лемме \ref{sc-rf-scb}. Рассмотрим оставшееся транзитивное замыкание:

  $(\lPO ; \lRFE^{?})^{+} = \lPO; (\lPO ; \lRFE)^{*} ; \lPO^? = \lPO; \lPO^{?} \cup \lPO; (\lPO ; \lRFE)^{+};\lPO^? = \lPO \cup (\lPO ; \lRFE)^{+};\lPO^?$. 

  В первом случае отношение сводится к  $\lPO \subseteq \lSCB$. Во втором случае, если последним ребром является $\lRFE$, то его можно отбросить по лемме \ref{sc-rf-scb}. Остаётся случай $(\lPO ; \lRFE)^{+} ; \lPO$. К каждой паре $\lPO ; \lRFE$ можно применить лемму \ref{sb-w-sync}. В результате нужно доказать следующее утверждение:

  $[(\lW \cup \lR)^{\sco}] ; \quad (\lPO ; ([\Far] ; \lPO ; [\lE^{\rlx}] ; \lRFE \cup [\Fa] ; \lPO ; [\lE^{\sco}] ; \lRFE) \cup \lRMW ; \lRFE)^{+}$
  
  $; \lPO ; [(\lW \cup \lR)^{\sco}] \subseteq  ([\lE^{\sco}] ; \lSCB ;\lE^{\sco}])^{*}$

  Воспользуемся леммой \ref{ct-decompose1}. Имеем, что либо транзитивное замыкание состоит только из $\lRMW ; \lRFE$, либо такие пары рёбер могут следовать после $\lPO ; ([\Far] ; \lPO ; (\lE^{\rlx}) ; \lRFE \cup [\Fa] ; \lPO ; [\lE^{\sco}] ; \lRFE)$. В первом случае замыкание имеет вид $\lHB_{loc} \subseteq \lSCB$, а так как оно заканчивается $\lSC$-событием, оставшееся ребро $\lPO$ также пройдёт по $\lSC$ и образует $\lSCB$. Во втором случае рассмотрим, что именно находится под транзитивным замыканием:

  \newcommand{\CA}{\mathtt{C}}

  \noindent $(\lPO ; ([\Far] ; \lPO ; [\lE^{\rlx}] ; \lRFE \cup [\Fa] ; \lPO ; [\lE^{\sco}] ; \lRFE) ; (\lRMW ; \lRFE)^{*})^{+}=$

  \noindent $(\lPO ; [\Fa] ;([\Far] ; \lPO ; [\lE^{\rlx}] \cup [\Fa] ; \lPO ; [\lE^{\sco}]) ; (\lRFE ; \lRMW)^{*}; \lRFE)^{+}$

  \noindent $=(\lPO ; [\Fa]; \CA; \lRFE)^{+}$, где 

  $\CA=\CA_1 \cup \CA_2 = [\Far] ; \lPO ; [\lE^{\rlx}] ; (\lRFE ; \lRMW)^{*} \cup [\Fa] ; \lPO ; [\lE^{\sco}] ; (\lRFE ; \lRMW)^{*}$.

  \newcommand{\canceloffset}{\hspace{-0.78cm}}
  \canceloffset Заметим, что верно следующее: 
  
  $(\lPO ; [\Fa]; \CA; \lRFE)^{+}=\lPO ; [\Fa]; (\CA; \lRFE; \lPO ; [\Fa])^{*}; \CA; \lRFE$.

  \canceloffset В результате необходимо доказать вот что: 

  $[(\lW \cup \lR)^{\sco}] ; \lPO ; [\Fa]; (\CA; \lRFE; \lPO ; [\Fa];)^{*}; \CA; \lRFE; \lPO ; [(\lW \cup \lR)^{\sco}] $

  $\subseteq  ([\lE^{\sco}] ; \lSCB ; [\lE^{\sco}])^{*}$. 

  \canceloffset Заметим, что выполнено следующее утверждение:

  \noindent $(\CA; \lRFE; \lPO ; [\Fa])^{*} = $

  \noindent $\left(([\Far] ; \lPO ; [\lE^{\rlx}] \cup [\Fa] ; \lPO ; [\lE^{\sco}]) ; (\lRFE ; \lRMW)^{*}; \lRFE; \lPO ; [\Fa] \right)^{*} $

  \noindent $\subseteq \lHB^?.$
  
  
  \canceloffset Это справедливо, поскольку $\lHB$ задано так: 

  $\lHB \defeq (\lPO \cup \lSW)^{+}$,

  $\lSW \supset \mathtt{\color{blue}release};\lRFE;\lPO;[\lF^{\acq}]$,

  $\mathtt{\color{blue}release} \defeq ([\lW^{\rel}] \cup [\lF^{\rel}];\lPO);\mathtt{\color{blue}rs}$,

  $\mathtt{\color{blue}rs} \supset (\lRFE;\lRMW)^{*}$.

  Вспомним, что $\lPO_{\neq loc} ; \lHB ; \lPO_{\neq loc} \subseteq \lSCB$. Воспользуемся тем, что $\lPO$ между событием чтения/записи и барьером образует именно $\lPO_{\neq loc}$. Тогда видно, что $[(\lW \cup \lR)^{\sco}]; \lPO; [\lF^{\acq}] \subseteq [\lE^{\sco}]; \lPO_{\neq loc}$.

  Остаётся доказать следующее утверждение:

  $[\lE^{\sco}]; \lPO_{\neq loc} ; \lHB^?; \CA; \lRFE; \lPO ; [(\lW \cup \lR)^{\sco}] \subseteq  ([\lE^{\sco}] ; \lSCB ; [\lE^{\sco}])^{*}$.

  Для этого перепишем $\CA=\CA_1 \cup \CA_2$ и докажем утверждение для $\CA_1$ и $\CA_2$ по отдельности. 

  
  Случай с $\CA_1$. Покажем, что

  $[\lE^{\sco}]; \lPO_{\neq loc} ; \lHB^?; \CA_1; \lRFE; \lPO ; [(\lW \cup \lR)^{\sco}]\subseteq  ([\lE^{\sco}] ; \lSCB ; [\lE^{\sco}])^{*}$.
Заметим, что по лемме \ref{sb-sc-sync}   в последнем ребре $\lPO$ найдётся $\acq$-барьер, с помощью которого можно будет построить ребро $\lHB$:

\noindent $\CA_1; \lRFE; \lPO ; [(\lW \cup \lR)^{\sco}] =$

\noindent $[\Far] ; \lPO ; [\lE^{\rlx}] ; (\lRFE ; \lRMW)^{*}; \lRFE; \lPO ; [(\lW \cup \lR)^{\sco}]=$

\noindent $[\Far] ; \lPO ; [\lE^{\rlx}] ; (\lRFE ; \lRMW)^{*}; \lRFE; [\lE^{\rlx}]; \lPO; [\Fa]; \lPO ; [(\lW \cup \lR)^{\sco}]$

\noindent $\subseteq \lHB; \lPO_{\neq loc}; [\lE^{\sco}]$.  Тогда $[\lE^{\sco}]; \lPO_{\neq loc} ; \lHB^?; \CA_1; \lRFE; \lPO ; [(\lW \cup \lR)^{\sco}]$

\noindent $\subseteq [\lE^{\sco}]; \lPO_{\neq loc} ; \lHB^?; \lHB; \lPO_{\neq loc}; [\lE^{\sco}]$

\noindent $\subseteq [\lE^{\sco}] ; \lSCB ; [\lE^{\sco}]$.

    \vspace{1em}
    
    Случай с $\CA_2$. Покажем, что выполнено следующее:

    $[\lE^{\sco}]; \lPO_{\neq loc} ; \lHB^?; \CA_2; \lRFE; \lPO ; [(\lW \cup \lR)^{\sco}]\subseteq  ([\lE^{\sco}] ; \lSCB ; [\lE^{\sco}])^{*}$. Заметим, что последовательность пар рёбер $\lRFE ; \lRMW$ входит в $\lSCB$: 

    $\CA_2=[\Fa] ; \lPO ; [\lE^{\sco}] ; (\lRFE ; \lRMW)^{*} $

    $\subseteq \lPO_{\neq loc}; [\lE^{\sco}] ; ([\lE^{\sco}] ; \lRFE; [\lE^{\sco}];  \lRMW; [\lE^{\sco}])^{*}; [\lE^{\sco}]$

    $\subseteq \lPO_{\neq loc}; [\lE^{\sco}] ; ([\lE^{\sco}] ; \lSCB ; [\lE^{\sco}])^{*}; [\lE^{\sco}]$.

    В этом случае имеем:

    \noindent $[\lE^{\sco}]; \lPO_{\neq loc} ; \lHB^?; \CA_2; \lRFE; \lPO ; [(\lW \cup \lR)^{\sco}]\subseteq$

    \noindent $[\lE^{\sco}]; \lPO_{\neq loc} ; \lHB^?; \lPO_{\neq loc}; [\lE^{\sco}] ; ([\lE^{\sco}] ; \lSCB ; [\lE^{\sco}])^{*}; [\lE^{\sco}]; \lRFE; \lPO ; [(\lW \cup \lR)^{\sco}]$

    \noindent $\subseteq ([\lE^{\sco}] ; \lSCB ; [\lE^{\sco}]) ; ([\lE^{\sco}] ; \lSCB ; [\lE^{\sco}])^{*}; ([\lE^{\sco}] ; \lSCB ; [\lE^{\sco}])^2$. \qedhere  
  
\end{proof}

\section{Формализация доказательства в Coq}

В некоторых опубликованных доказательствах корректности компиляции впоследствии были найдены неточности. Например, \cite{rc11} демонстрирует ошибку в схеме компиляции C++ в Power, \cite{imm} --- в схеме компиляции модели Promising в Power.

Чтобы избежать подобных ошибок, доказательство из раздела \ref{proof} было решено формализовать в Coq. Это одна из систем интерактивного доказательства теорем, которые позволяют записать доказательство математического утверждения на функциональном языке программирования и автоматически проверить его корректность. В настоящее время Coq и его аналоги используются как для формализации математики \cite{four-color}, так и для верификации отдельных программных проектов \cite{compcert}. 

Выбор системы Coq обусловлен наличием существующей формализации $\IMM$ и доказательств корректности компиляции для неё \cite{imm-repo}. Данная формализация построена на основе библиотеки \texttt{hahn}, упрощающей работу с бинарными отношениями.

На данный момент формализация $\OMM$ и доказательства корректности компиляции в $\IMM$ занимает порядка 8 тыс. строк кода на Coq. Исходный код доступен на \cite{imm-repo}.

Структура доказательства в Coq повторяет таковую в разделе \ref{proof}. Формализация разбита на следующие файлы:

\begin{itemize}
\item \verb|OCaml.v| --- формализация $\OMM$;
\item \verb|OmmProgram.v| --- условия на программы в $\OMM$ и исполнения в этой модели;  
\item \verb|OmmImmCompScheme.v| --- описание схемы компиляции и её свойств;
\item \verb|OmmImmCompCorrect.v| --- высокоуровневая структура доказательства;
\item \verb|ImmImpliesOmm.v| --- доказательство согласованности по $\OMM$ графа исполнения, согласованного по $\IMM$;  
\item \verb|OmmImmSimulation.v| --- доказательство отношения симуляции между исполнениями в $\OMM$ и $\IMM$;
\item \verb|GraphConstruction.v| --- построение графа исполнения из однопоточных графов исполнения;
\item \verb|BlockSteps.v| --- преобразование скомпилированной программы в последовательность блоков, полученных при компиляции одной исходной инструкции;
\item \verb|CompSchemeGraph.v| --- доказательство утверждения о том, что в графе исполнения скомпилированной программы выполняются предпосылки, необходимые для доказательства его согласованности по $\OMM$;
\item \verb|BoundedRelsProperties.v| --- вспомогательные утверждения о компонентах графа, значения которых устанавливаются пошагово в ходе его построения;
\item \verb|ClosuresProperties.v| --- вспомогательные утверждения про транзитивное и транзитивно-рефлексивное замыкания;
\item \verb|Utils.v| --- вспомогательные утверждения про отношения;
\item \verb|ListHelpers.v| --- вспомогательные утверждения про списки, в том числе те, которые доказаны в Coq 8.10, но не добавлены в стандартную библиотеку Coq 8.09, используемую в проекте.
\end{itemize}

В основе Coq лежит интуиционистская логика, т.е. по умолчанию в нём не используется закон исключённого третьего. Это исключает неконструктивные доказательства, что, в частности, делает возможным экстракцию доказательств на Coq в программы на других языках. Однако, так как экстракция данного доказательства не требуется, а применение закона исключённого третьего значительно упрощает доказательство (и в некоторых случаях является необходимым), то было решено использовать эту аксиому.

Одним из достоинств Coq является возможность написания пользовательских тактик --- инструкций, преобразующих текущую цель доказательства. Такие тактики могут быть частично автоматизированы, т.к. включать перебор различных вариантов доказательства прозрачно от пользователя. В рамках данного проекта было разработано несколько тактик, которые значительно сократили объём доказательств. Подходящей ситуацией для их применения является перебор случаев, выкладки в которых являются схожими, но отличаются в деталях. В этом случае разумным решением является разработка специализированной тактики, которая нивелирует различия разных случаев путём автоматизации.

В рамках данного проекта применение Coq позволило уточнить входные условия на программы, которые до начала работы над проектом явным образом не оговаривались. Так, явным образом выражено условие на разделение атомарных и неатомарных адресов на уровне программы для $\OMM$. Также от этой программы  требуется выделить под инструкции $\mathtt{exchange}$ специальный регистр, который не будет используется в остальной части программы. Наконец, для упрощения части доказательства, касающейся блочных состояний и переходов было решено ограничить целевые адреса в инструкциях $\mathtt{ifgoto}$, запретив переходы за пределы программы. Данное ограничение выглядит разумным, т.к. соответствующее преобразование может быть выполнено реальным компилятором.


\section{Существующие исследования}
\label{related-work}

Проблема корректности компиляции из $\OMM$ и в $\IMM$ рассматривается и в других работах. Так, авторы $\OMM$ \cite{omm} разработали для неё схемы компиляции в архитектуры x86 и ARMv8 \cite{arm}. Модель памяти x86 практически не отличается от SC \cite{x86-reorderings}, поэтому для корректной компиляции $\OMM$ в x86 достаточно лишь реализовать инструкцию атомарной записи с использованием $\mathtt{xchg}$. В схеме компиляции $\OMM$ в ARMv8, в отличие от предложенной нами схемы, при компиляции инструкции неатомарной записи используется барьер $\lF^{\acq}$, а не $\lF^{\acqrel}$. Это объясняется тем, что в модели ARMv8 отношение $\mathtt{ob}$ (аналог $\lAR$ в $\IMM$) включает в себя $\lRFE \cup \lFRE \cup \lCOE$ по неатомарным операциям и $\lPO$ с $\acq$-барьером перед событием записи, поэтому в последовательности рёбер вида $(\lPO;\lRFE)^{+}$ не требуется $\rel$-барьер. Наконец, в \cite{omm-stm} утверждается, что свойство локальной свободы от гонок можно реализовать в программной транзакционной памяти с использованием тех же схем компиляции в x86 и ARMv8.

В \cite{imm} приведены схемы компиляции моделей  Promising \cite{promising} и RC11 \cite{rc11} в $\IMM$. Предикат консистентности RC11 требует ацикличности отношения $\lPO \cup \lRF$, поэтому между каждым событием не-$\sco$ чтения и последующим событием записи необходимо располагать барьер. С этим дополнительным условием доказательство корректности компиляции значительно упрощается. В свою очередь, модель Promising является достаточно слабой, поэтому для неё корректной является тривиальная схема компиляции (при выполнении некоторых условий на инструкции read-modify-write в ней). Однако доказательство этого значительно сложнее, так как в нём применяется метод обхода графа исполнения, который позволяет связать операционную семантику модели Promising и декларативную семантику $\IMM$.

Стоит отметить, что компиляция с использованием промежуточного представления программы может быть менее эффективной, чем компиляция в целевую модель памяти напрямую. Так, при композиции схем компиляции $\OMM$ в $\IMM$ \ref{table:scheme} и $\IMM$ в ARMv8 \cite{imm-sc} инструкция атомарной записи $\OMM$ компилируется с использованием двух барьеров, в то время как схема компиляции $\OMM$ в ARMv8 из \cite{omm} использует только один. 

\section{Заключение}
\label{future-work}

% В данной работе представлена схема компиляции $\OMM$ в промежуточную модель, позволяющая получить схему компиляции $\OMM$ в Power.

% Были выполнены поставленные задачи:
% \begin{itemize}
% \item предложена схема компиляции $\OMM$ в $\IMM$. Так как $\IMM$ является более слабой моделью, чем $\OMM$, в схеме компиляции используются барьеры памяти и инструкции read-modify-write, которые обеспечивают необходимую синхронизацию для неатомарных и атомарных доступов к памяти соответственно;
% \item для доказательства корректности компиляции был рассмотрен граф исполнения скомпилированной программы и доказано, что из согласованности по $\IMM$ такого графа следует его согласованность по $\OMM$;
% \item полученное доказательство было формализовано в Coq. 
% \end{itemize}

В данной работе представлена схема компиляции $\OMM$ в промежуточную модель, позволяющая получить схему компиляции $\OMM$ в Power. При этом были получены следующие результаты.

\begin{itemize}
\item Предложена схема компиляции $\OMM$ в $\IMM$, использующая барьеры памяти и инструкции read-modify-write для необходимой синхронизации доступов к памяти. 
\item Была доказана корректность предложенной схемы. Для этого было введено формальное понятие соответствия графов, которое позволило связать сценарии исполнения исходной и скомпилированной программ. Затем было доказано, что для каждого согласованного по $\IMM$ графа исполнения скомпилированной программы существует соответствующий ему граф исполнения исходной программы, согласованный по $\OMM$. 
\item Полученное доказательство было формализовано в Coq. Без учёта существующей формализации $\IMM$ формализация заняла порядка 8 тыс. строк кода. 
\end{itemize}

Результаты работы были представлены на Открытой конференции ИСП РАН и опубликованы в "Трудах Института системного программирования РАН" \cite{publication}. 

Данное исследование может быть продолжено при реализации свойства локальной свободы от гонок в других моделях памяти. Например, изучается возможность реализовать это свойство в модели памяти C++ \cite{ldrf-c11}. Результаты, полученные в данной работе, могут быть использованы для построения схем компиляции таких моделей. В частности, сравнение схемы компиляции $\OMM$ в ARMv8 \cite{omm} и полученной в данной работе позволяет предположить, что с большой вероятностью для компиляции инструкции атомарной записи в подобных моделях необходимо будет использовать инструкцию атомарной замены.


\bibliographystyle{ugost2008ls}
\bibliography{Namakonov/sources}

%%% Local Variables:
%%% mode: latex
%%% TeX-master: t
%%% End:
