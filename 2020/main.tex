\documentclass[runningheads]{jblab}

\usepackage{etex}
\usepackage[utf8]{inputenc}
\usepackage[english,russian]{babel}
\usepackage[T2A]{fontenc}
\usepackage{amsmath}

\usepackage{tikz}
\usetikzlibrary{arrows, matrix}

\usepackage{indentfirst}
\usepackage{amsfonts}
\usepackage{amssymb}
\usepackage{scalerel}
\usepackage{makecell}
\usepackage{graphicx}
\usepackage{listings}
\usepackage{algpseudocode}
\usepackage{algorithm}

\usepackage[compatibility=false]{caption}
\captionsetup{labelfont=bf}

\DeclareCaptionType{Listing}[Листинг]
\captionsetup[Listing]{within=none,labelsep=period}

\DeclareCaptionType{Figure}[Рисунок]
\captionsetup[Figure]{within=none,labelsep=period}

\usepackage{algorithmicx}
\usepackage{wrapfig}
\usepackage{supertabular}
\usepackage{epigraph}
\usepackage{euscript}
\usepackage{textcomp}
\usepackage{cite}
\usepackage{ucs}
\usepackage{verbments}
\usepackage{xcolor}
\usepackage{subcaption}
\usepackage[inference]{semantic}
\usepackage[pdfencoding=auto]{hyperref}
\usepackage{hyperref}
\usepackage[inline]{enumitem}
\usepackage{cmap}
\usepackage{bm}
\usepackage{cancel}
\usepackage{booktabs}
\usepackage{empheq}
\usepackage{enumitem}
\usepackage{thm-restate}
\usepackage[capitalise, english, russian]{cleveref}

\DeclareMathAlphabet{\mathcal}{OT1}{pzc}{m}{it}

\usepackage{bussproofs}
\usepackage{mathtools}
\usepackage{multirow}

\usepackage{url}
\let\proof\relax
\let\endproof\relax

\usepackage{amsthm}
\usepackage{stmaryrd}
\usepackage{placeins}
\usepackage{changepage}

\definecolor{dkgreen}{rgb}{0,0.6,0}
\definecolor{dred}{rgb}{0.545,0,0}
\definecolor{dblue}{rgb}{0,0,0.545}
\definecolor{lgrey}{rgb}{0.9,0.9,0.9}
\definecolor{gray}{rgb}{0.4,0.4,0.4}
\definecolor{darkblue}{rgb}{0.0,0.0,0.6}

\usepackage{verbatim}
\usepackage{float}
\usepackage{minted}
\usepackage{array}
\usepackage{thmtools}

%Terekhov
\usepackage{adjustbox}

\lstdefinelanguage{sql}{
keywords={MATCH, WHERE, RETURN, PATH, PATTERN, COUNT},
sensitive=true,
commentstyle=\small\itshape\ttfamily,
keywordstyle=\ttfamily\underbar,
identifierstyle=\ttfamily,
basewidth={0.5em,0.5em},
columns=fixed,
fontadjust=true,
literate={->}{{$\to$}}3,
morecomment=[s]{(*}{*)}
}


%Artemeva
\usepackage{listings}
\usepackage{amssymb}
\usepackage{amsthm}
\usepackage{boxedminipage}

% Kuklina
\usepackage{longtable}
\usetikzlibrary{positioning}

\newcommand{\cd}[1]{\texttt{#1}}
\let\emptyset\varnothing

\textwidth=10cm
\textheight=15cm

\oddsidemargin=0pt
\evensidemargin=0pt

\topmargin=0pt

\begin{document}
\newcommand{\Issue}[0]{8}
\newcommand{\Year}[0]{2020}

\sloppy

\begin{titlepage}

\centering

\includegraphics[width=4cm]{logo_JetBrains_1.png}
\vskip 1mm
\mbox{\Large{\textsc{Труды лаборатории}}}
\vskip 0.5cm
\mbox{\Large{\textsc{языковых инструментов}}}
\vskip 2.5cm
\large{Выпуск \Issue}
\vskip 6cm
\large{Санкт-Петербург, \Year}
\end{titlepage}

\thispagestyle{empty}
\phantom{xx}
\pagebreak

\chapter*{Предисловие}

Предисловие

\vskip 2cm
\begin{flushright}
\textit{подпись}
\end{flushright}

{
\tableofcontents
\thispagestyle{plain}
}

% Кастомные леммы и теоремы для удобства перевода на русский
% сделано по аналогии с вариантом 2019 года, только тут это в
% главном файле, а там было у Ю.Сусаниной в её тексте
\newtheorem{defn}{Определение}%[section]
\newtheorem{lm}{Лемма}
\newtheorem{thrm}{Теорема}
\setcounter{lm}{0}
\setcounter{thrm}{0}


\title{Разработка матричного алгоритма поиска путей с контекстно-свободными ограничениями для RedisGraph}
\titlerunning{Поиск путей с КС ограничениями для RedisGraph}

\author{Терехов Арсений Константинович}
\authorrunning{Терехов~А.~К.}

\tocauthor{Терехов~А.~К.}
\institute{Санкт-Петербургский государственный университет\\
	\email{simpletondl@yandex.ru}}

\maketitle

\begin{abstract}
Поиск путей с контекстно-свободными ограничениями подразумевает использование контекстно-свободной грамматики для задания ограничений на множество искомых путей в графе. Данные ограничения используются в таких областях, как статический анализ кода и анализ RDF-данных. Однако на текущий момент ни одна графовая база данных не поддерживает запросы с контекстно-свободными ограничениями, что препятствует развитию прикладных решений. В данной работе представлено решение данной проблемы: реализована поддержка расширенного необходимыми конструкциями языка запросов Cypher для графовой базы данных RedisGraph.
\end{abstract}

\phantomsection
\section*{Введение}

Реляционное программирование~--- это чистая форма логического программирования,
в которой программы представляются как наборы математических отношений~\cite{byrdMK}.
Отношения
не различают входные и выходные параметры, из-за чего одно и то же
отношение может решать несколько связанных проблем. К примеру, отношение, задающее
интерпретатор языка, можно использовать не только для вычисления программ по
заданному входу, но и для генерации возможных входных значений по заданному результату
или самих программ по спецификации входных и выходных значений.

miniKanren~--- это семейство встраиваемых предметно-ориентированных языков программирования~\cite{byrdMK}.
miniKanren был специально сконструирован для поддержки реляционной парадигмы,
опираясь на опыт логических языков, таких как языки семейства Prolog~\cite{logicMJ},
Mercury~\cite{mercury} и Curry~\cite{curry}.

Реляционная парадигма довольно сложна, хотя потенциал её весьма велик.
Часто наиболее естественный способ записи отношения не является эффективным. В
частности, при задании функциональных отношений как сопоставления выходов
входам, как это наблюдается в примере с интерпретатором, поиск входов по выходам практически
всегда работает медленно.

Специализация --- это техника автоматической оптимизации программ,
при которой на основе программы и её частично известного входа
порождается новая, более эффективная программа, которая сохраняет семантику
исходной. Для специализации логических языков используются методы частичной дедукции~\cite{advanced},
самый проработанный из которых --- это \cpd\cite{cpd}. ECCE, реализация \forcpd для Prolog, показывает
хорошие результаты~\cite{controlPoly},
% однако специфика реляционного программирования и его отличия от логических языков подразумевает возможность разработать более подходящий
однако, в силу различий между реляционным и логическим программированием, можно предположить возможность разработать более подходящий
метод специализации. Уже существует адаптация \forcpd для miniKanren~\cite{lozov},
однако её результаты нестабильны: несмотря на то, что в некоторых случаях
производительность программ улучшается, в других -- она может существенно ухудшиться.

Другой подход для специализации --- это суперкомпиляция,
техника автоматической трансформации и анализа программ,
при которой программа символьно исполняется с сохранением истории вычислений,
на основе которой строится оптимизированная версия кода.
Суперкомпиляция успешно применяется к функциональным и императивным языкам,
однако для логических языков не сильно развита. Существуют
работы, посвящённые демонстрации сходства процессов частичной дедукции и суперкомпиляции~\cite{pdAndDriving},
а также предназначенный для Prolog суперкомпилятор APROPOS~\cite{apropos}, который, однако, довольно ограничен
в своих возможностях и требует ручного контроля.

В данной работе предлагается способ адаптации и реализации суперкомпилятора для
реляционного языка miniKanren, а также рассматриваются его возможные вариации, приводящие к
дальнейшему повышению производительности реляционных программ, и производится экспериментальное
исследование результата.

\input{Terekhov/diploma_bib}

\title{Разработка матричного алгоритма поиска путей с контекстно-свободными ограничениями для RedisGraph}
\titlerunning{Поиск путей с КС ограничениями для RedisGraph}

\author{Терехов Арсений Константинович}
\authorrunning{Терехов~А.~К.}

\tocauthor{Терехов~А.~К.}
\institute{Санкт-Петербургский государственный университет\\
	\email{simpletondl@yandex.ru}}

\maketitle

\begin{abstract}
Поиск путей с контекстно-свободными ограничениями подразумевает использование контекстно-свободной грамматики для задания ограничений на множество искомых путей в графе. Данные ограничения используются в таких областях, как статический анализ кода и анализ RDF-данных. Однако на текущий момент ни одна графовая база данных не поддерживает запросы с контекстно-свободными ограничениями, что препятствует развитию прикладных решений. В данной работе представлено решение данной проблемы: реализована поддержка расширенного необходимыми конструкциями языка запросов Cypher для графовой базы данных RedisGraph.
\end{abstract}

\phantomsection
\section*{Введение}

Реляционное программирование~--- это чистая форма логического программирования,
в которой программы представляются как наборы математических отношений~\cite{byrdMK}.
Отношения
не различают входные и выходные параметры, из-за чего одно и то же
отношение может решать несколько связанных проблем. К примеру, отношение, задающее
интерпретатор языка, можно использовать не только для вычисления программ по
заданному входу, но и для генерации возможных входных значений по заданному результату
или самих программ по спецификации входных и выходных значений.

miniKanren~--- это семейство встраиваемых предметно-ориентированных языков программирования~\cite{byrdMK}.
miniKanren был специально сконструирован для поддержки реляционной парадигмы,
опираясь на опыт логических языков, таких как языки семейства Prolog~\cite{logicMJ},
Mercury~\cite{mercury} и Curry~\cite{curry}.

Реляционная парадигма довольно сложна, хотя потенциал её весьма велик.
Часто наиболее естественный способ записи отношения не является эффективным. В
частности, при задании функциональных отношений как сопоставления выходов
входам, как это наблюдается в примере с интерпретатором, поиск входов по выходам практически
всегда работает медленно.

Специализация --- это техника автоматической оптимизации программ,
при которой на основе программы и её частично известного входа
порождается новая, более эффективная программа, которая сохраняет семантику
исходной. Для специализации логических языков используются методы частичной дедукции~\cite{advanced},
самый проработанный из которых --- это \cpd\cite{cpd}. ECCE, реализация \forcpd для Prolog, показывает
хорошие результаты~\cite{controlPoly},
% однако специфика реляционного программирования и его отличия от логических языков подразумевает возможность разработать более подходящий
однако, в силу различий между реляционным и логическим программированием, можно предположить возможность разработать более подходящий
метод специализации. Уже существует адаптация \forcpd для miniKanren~\cite{lozov},
однако её результаты нестабильны: несмотря на то, что в некоторых случаях
производительность программ улучшается, в других -- она может существенно ухудшиться.

Другой подход для специализации --- это суперкомпиляция,
техника автоматической трансформации и анализа программ,
при которой программа символьно исполняется с сохранением истории вычислений,
на основе которой строится оптимизированная версия кода.
Суперкомпиляция успешно применяется к функциональным и императивным языкам,
однако для логических языков не сильно развита. Существуют
работы, посвящённые демонстрации сходства процессов частичной дедукции и суперкомпиляции~\cite{pdAndDriving},
а также предназначенный для Prolog суперкомпилятор APROPOS~\cite{apropos}, который, однако, довольно ограничен
в своих возможностях и требует ручного контроля.

В данной работе предлагается способ адаптации и реализации суперкомпилятора для
реляционного языка miniKanren, а также рассматриваются его возможные вариации, приводящие к
дальнейшему повышению производительности реляционных программ, и производится экспериментальное
исследование результата.

\input{Terekhov/diploma_bib}

\title{Разработка матричного алгоритма поиска путей с контекстно-свободными ограничениями для RedisGraph}
\titlerunning{Поиск путей с КС ограничениями для RedisGraph}

\author{Терехов Арсений Константинович}
\authorrunning{Терехов~А.~К.}

\tocauthor{Терехов~А.~К.}
\institute{Санкт-Петербургский государственный университет\\
	\email{simpletondl@yandex.ru}}

\maketitle

\begin{abstract}
Поиск путей с контекстно-свободными ограничениями подразумевает использование контекстно-свободной грамматики для задания ограничений на множество искомых путей в графе. Данные ограничения используются в таких областях, как статический анализ кода и анализ RDF-данных. Однако на текущий момент ни одна графовая база данных не поддерживает запросы с контекстно-свободными ограничениями, что препятствует развитию прикладных решений. В данной работе представлено решение данной проблемы: реализована поддержка расширенного необходимыми конструкциями языка запросов Cypher для графовой базы данных RedisGraph.
\end{abstract}

\phantomsection
\section*{Введение}

Реляционное программирование~--- это чистая форма логического программирования,
в которой программы представляются как наборы математических отношений~\cite{byrdMK}.
Отношения
не различают входные и выходные параметры, из-за чего одно и то же
отношение может решать несколько связанных проблем. К примеру, отношение, задающее
интерпретатор языка, можно использовать не только для вычисления программ по
заданному входу, но и для генерации возможных входных значений по заданному результату
или самих программ по спецификации входных и выходных значений.

miniKanren~--- это семейство встраиваемых предметно-ориентированных языков программирования~\cite{byrdMK}.
miniKanren был специально сконструирован для поддержки реляционной парадигмы,
опираясь на опыт логических языков, таких как языки семейства Prolog~\cite{logicMJ},
Mercury~\cite{mercury} и Curry~\cite{curry}.

Реляционная парадигма довольно сложна, хотя потенциал её весьма велик.
Часто наиболее естественный способ записи отношения не является эффективным. В
частности, при задании функциональных отношений как сопоставления выходов
входам, как это наблюдается в примере с интерпретатором, поиск входов по выходам практически
всегда работает медленно.

Специализация --- это техника автоматической оптимизации программ,
при которой на основе программы и её частично известного входа
порождается новая, более эффективная программа, которая сохраняет семантику
исходной. Для специализации логических языков используются методы частичной дедукции~\cite{advanced},
самый проработанный из которых --- это \cpd\cite{cpd}. ECCE, реализация \forcpd для Prolog, показывает
хорошие результаты~\cite{controlPoly},
% однако специфика реляционного программирования и его отличия от логических языков подразумевает возможность разработать более подходящий
однако, в силу различий между реляционным и логическим программированием, можно предположить возможность разработать более подходящий
метод специализации. Уже существует адаптация \forcpd для miniKanren~\cite{lozov},
однако её результаты нестабильны: несмотря на то, что в некоторых случаях
производительность программ улучшается, в других -- она может существенно ухудшиться.

Другой подход для специализации --- это суперкомпиляция,
техника автоматической трансформации и анализа программ,
при которой программа символьно исполняется с сохранением истории вычислений,
на основе которой строится оптимизированная версия кода.
Суперкомпиляция успешно применяется к функциональным и императивным языкам,
однако для логических языков не сильно развита. Существуют
работы, посвящённые демонстрации сходства процессов частичной дедукции и суперкомпиляции~\cite{pdAndDriving},
а также предназначенный для Prolog суперкомпилятор APROPOS~\cite{apropos}, который, однако, довольно ограничен
в своих возможностях и требует ручного контроля.

В данной работе предлагается способ адаптации и реализации суперкомпилятора для
реляционного языка miniKanren, а также рассматриваются его возможные вариации, приводящие к
дальнейшему повышению производительности реляционных программ, и производится экспериментальное
исследование результата.

\input{Terekhov/diploma_bib}

\title{Разработка матричного алгоритма поиска путей с контекстно-свободными ограничениями для RedisGraph}
\titlerunning{Поиск путей с КС ограничениями для RedisGraph}

\author{Терехов Арсений Константинович}
\authorrunning{Терехов~А.~К.}

\tocauthor{Терехов~А.~К.}
\institute{Санкт-Петербургский государственный университет\\
	\email{simpletondl@yandex.ru}}

\maketitle

\begin{abstract}
Поиск путей с контекстно-свободными ограничениями подразумевает использование контекстно-свободной грамматики для задания ограничений на множество искомых путей в графе. Данные ограничения используются в таких областях, как статический анализ кода и анализ RDF-данных. Однако на текущий момент ни одна графовая база данных не поддерживает запросы с контекстно-свободными ограничениями, что препятствует развитию прикладных решений. В данной работе представлено решение данной проблемы: реализована поддержка расширенного необходимыми конструкциями языка запросов Cypher для графовой базы данных RedisGraph.
\end{abstract}

\phantomsection
\section*{Введение}

Реляционное программирование~--- это чистая форма логического программирования,
в которой программы представляются как наборы математических отношений~\cite{byrdMK}.
Отношения
не различают входные и выходные параметры, из-за чего одно и то же
отношение может решать несколько связанных проблем. К примеру, отношение, задающее
интерпретатор языка, можно использовать не только для вычисления программ по
заданному входу, но и для генерации возможных входных значений по заданному результату
или самих программ по спецификации входных и выходных значений.

miniKanren~--- это семейство встраиваемых предметно-ориентированных языков программирования~\cite{byrdMK}.
miniKanren был специально сконструирован для поддержки реляционной парадигмы,
опираясь на опыт логических языков, таких как языки семейства Prolog~\cite{logicMJ},
Mercury~\cite{mercury} и Curry~\cite{curry}.

Реляционная парадигма довольно сложна, хотя потенциал её весьма велик.
Часто наиболее естественный способ записи отношения не является эффективным. В
частности, при задании функциональных отношений как сопоставления выходов
входам, как это наблюдается в примере с интерпретатором, поиск входов по выходам практически
всегда работает медленно.

Специализация --- это техника автоматической оптимизации программ,
при которой на основе программы и её частично известного входа
порождается новая, более эффективная программа, которая сохраняет семантику
исходной. Для специализации логических языков используются методы частичной дедукции~\cite{advanced},
самый проработанный из которых --- это \cpd\cite{cpd}. ECCE, реализация \forcpd для Prolog, показывает
хорошие результаты~\cite{controlPoly},
% однако специфика реляционного программирования и его отличия от логических языков подразумевает возможность разработать более подходящий
однако, в силу различий между реляционным и логическим программированием, можно предположить возможность разработать более подходящий
метод специализации. Уже существует адаптация \forcpd для miniKanren~\cite{lozov},
однако её результаты нестабильны: несмотря на то, что в некоторых случаях
производительность программ улучшается, в других -- она может существенно ухудшиться.

Другой подход для специализации --- это суперкомпиляция,
техника автоматической трансформации и анализа программ,
при которой программа символьно исполняется с сохранением истории вычислений,
на основе которой строится оптимизированная версия кода.
Суперкомпиляция успешно применяется к функциональным и императивным языкам,
однако для логических языков не сильно развита. Существуют
работы, посвящённые демонстрации сходства процессов частичной дедукции и суперкомпиляции~\cite{pdAndDriving},
а также предназначенный для Prolog суперкомпилятор APROPOS~\cite{apropos}, который, однако, довольно ограничен
в своих возможностях и требует ручного контроля.

В данной работе предлагается способ адаптации и реализации суперкомпилятора для
реляционного языка miniKanren, а также рассматриваются его возможные вариации, приводящие к
дальнейшему повышению производительности реляционных программ, и производится экспериментальное
исследование результата.

\input{Terekhov/diploma_bib}

% % \title{Разработка матричного алгоритма поиска путей с контекстно-свободными ограничениями для RedisGraph}
\titlerunning{Поиск путей с КС ограничениями для RedisGraph}

\author{Терехов Арсений Константинович}
\authorrunning{Терехов~А.~К.}

\tocauthor{Терехов~А.~К.}
\institute{Санкт-Петербургский государственный университет\\
	\email{simpletondl@yandex.ru}}

\maketitle

\begin{abstract}
Поиск путей с контекстно-свободными ограничениями подразумевает использование контекстно-свободной грамматики для задания ограничений на множество искомых путей в графе. Данные ограничения используются в таких областях, как статический анализ кода и анализ RDF-данных. Однако на текущий момент ни одна графовая база данных не поддерживает запросы с контекстно-свободными ограничениями, что препятствует развитию прикладных решений. В данной работе представлено решение данной проблемы: реализована поддержка расширенного необходимыми конструкциями языка запросов Cypher для графовой базы данных RedisGraph.
\end{abstract}

\phantomsection
\section*{Введение}

Реляционное программирование~--- это чистая форма логического программирования,
в которой программы представляются как наборы математических отношений~\cite{byrdMK}.
Отношения
не различают входные и выходные параметры, из-за чего одно и то же
отношение может решать несколько связанных проблем. К примеру, отношение, задающее
интерпретатор языка, можно использовать не только для вычисления программ по
заданному входу, но и для генерации возможных входных значений по заданному результату
или самих программ по спецификации входных и выходных значений.

miniKanren~--- это семейство встраиваемых предметно-ориентированных языков программирования~\cite{byrdMK}.
miniKanren был специально сконструирован для поддержки реляционной парадигмы,
опираясь на опыт логических языков, таких как языки семейства Prolog~\cite{logicMJ},
Mercury~\cite{mercury} и Curry~\cite{curry}.

Реляционная парадигма довольно сложна, хотя потенциал её весьма велик.
Часто наиболее естественный способ записи отношения не является эффективным. В
частности, при задании функциональных отношений как сопоставления выходов
входам, как это наблюдается в примере с интерпретатором, поиск входов по выходам практически
всегда работает медленно.

Специализация --- это техника автоматической оптимизации программ,
при которой на основе программы и её частично известного входа
порождается новая, более эффективная программа, которая сохраняет семантику
исходной. Для специализации логических языков используются методы частичной дедукции~\cite{advanced},
самый проработанный из которых --- это \cpd\cite{cpd}. ECCE, реализация \forcpd для Prolog, показывает
хорошие результаты~\cite{controlPoly},
% однако специфика реляционного программирования и его отличия от логических языков подразумевает возможность разработать более подходящий
однако, в силу различий между реляционным и логическим программированием, можно предположить возможность разработать более подходящий
метод специализации. Уже существует адаптация \forcpd для miniKanren~\cite{lozov},
однако её результаты нестабильны: несмотря на то, что в некоторых случаях
производительность программ улучшается, в других -- она может существенно ухудшиться.

Другой подход для специализации --- это суперкомпиляция,
техника автоматической трансформации и анализа программ,
при которой программа символьно исполняется с сохранением истории вычислений,
на основе которой строится оптимизированная версия кода.
Суперкомпиляция успешно применяется к функциональным и императивным языкам,
однако для логических языков не сильно развита. Существуют
работы, посвящённые демонстрации сходства процессов частичной дедукции и суперкомпиляции~\cite{pdAndDriving},
а также предназначенный для Prolog суперкомпилятор APROPOS~\cite{apropos}, который, однако, довольно ограничен
в своих возможностях и требует ручного контроля.

В данной работе предлагается способ адаптации и реализации суперкомпилятора для
реляционного языка miniKanren, а также рассматриваются его возможные вариации, приводящие к
дальнейшему повышению производительности реляционных программ, и производится экспериментальное
исследование результата.

\input{Terekhov/diploma_bib}

\title{Разработка матричного алгоритма поиска путей с контекстно-свободными ограничениями для RedisGraph}
\titlerunning{Поиск путей с КС ограничениями для RedisGraph}

\author{Терехов Арсений Константинович}
\authorrunning{Терехов~А.~К.}

\tocauthor{Терехов~А.~К.}
\institute{Санкт-Петербургский государственный университет\\
	\email{simpletondl@yandex.ru}}

\maketitle

\begin{abstract}
Поиск путей с контекстно-свободными ограничениями подразумевает использование контекстно-свободной грамматики для задания ограничений на множество искомых путей в графе. Данные ограничения используются в таких областях, как статический анализ кода и анализ RDF-данных. Однако на текущий момент ни одна графовая база данных не поддерживает запросы с контекстно-свободными ограничениями, что препятствует развитию прикладных решений. В данной работе представлено решение данной проблемы: реализована поддержка расширенного необходимыми конструкциями языка запросов Cypher для графовой базы данных RedisGraph.
\end{abstract}

\phantomsection
\section*{Введение}

Реляционное программирование~--- это чистая форма логического программирования,
в которой программы представляются как наборы математических отношений~\cite{byrdMK}.
Отношения
не различают входные и выходные параметры, из-за чего одно и то же
отношение может решать несколько связанных проблем. К примеру, отношение, задающее
интерпретатор языка, можно использовать не только для вычисления программ по
заданному входу, но и для генерации возможных входных значений по заданному результату
или самих программ по спецификации входных и выходных значений.

miniKanren~--- это семейство встраиваемых предметно-ориентированных языков программирования~\cite{byrdMK}.
miniKanren был специально сконструирован для поддержки реляционной парадигмы,
опираясь на опыт логических языков, таких как языки семейства Prolog~\cite{logicMJ},
Mercury~\cite{mercury} и Curry~\cite{curry}.

Реляционная парадигма довольно сложна, хотя потенциал её весьма велик.
Часто наиболее естественный способ записи отношения не является эффективным. В
частности, при задании функциональных отношений как сопоставления выходов
входам, как это наблюдается в примере с интерпретатором, поиск входов по выходам практически
всегда работает медленно.

Специализация --- это техника автоматической оптимизации программ,
при которой на основе программы и её частично известного входа
порождается новая, более эффективная программа, которая сохраняет семантику
исходной. Для специализации логических языков используются методы частичной дедукции~\cite{advanced},
самый проработанный из которых --- это \cpd\cite{cpd}. ECCE, реализация \forcpd для Prolog, показывает
хорошие результаты~\cite{controlPoly},
% однако специфика реляционного программирования и его отличия от логических языков подразумевает возможность разработать более подходящий
однако, в силу различий между реляционным и логическим программированием, можно предположить возможность разработать более подходящий
метод специализации. Уже существует адаптация \forcpd для miniKanren~\cite{lozov},
однако её результаты нестабильны: несмотря на то, что в некоторых случаях
производительность программ улучшается, в других -- она может существенно ухудшиться.

Другой подход для специализации --- это суперкомпиляция,
техника автоматической трансформации и анализа программ,
при которой программа символьно исполняется с сохранением истории вычислений,
на основе которой строится оптимизированная версия кода.
Суперкомпиляция успешно применяется к функциональным и императивным языкам,
однако для логических языков не сильно развита. Существуют
работы, посвящённые демонстрации сходства процессов частичной дедукции и суперкомпиляции~\cite{pdAndDriving},
а также предназначенный для Prolog суперкомпилятор APROPOS~\cite{apropos}, который, однако, довольно ограничен
в своих возможностях и требует ручного контроля.

В данной работе предлагается способ адаптации и реализации суперкомпилятора для
реляционного языка miniKanren, а также рассматриваются его возможные вариации, приводящие к
дальнейшему повышению производительности реляционных программ, и производится экспериментальное
исследование результата.

\input{Terekhov/diploma_bib}

\title{Разработка матричного алгоритма поиска путей с контекстно-свободными ограничениями для RedisGraph}
\titlerunning{Поиск путей с КС ограничениями для RedisGraph}

\author{Терехов Арсений Константинович}
\authorrunning{Терехов~А.~К.}

\tocauthor{Терехов~А.~К.}
\institute{Санкт-Петербургский государственный университет\\
	\email{simpletondl@yandex.ru}}

\maketitle

\begin{abstract}
Поиск путей с контекстно-свободными ограничениями подразумевает использование контекстно-свободной грамматики для задания ограничений на множество искомых путей в графе. Данные ограничения используются в таких областях, как статический анализ кода и анализ RDF-данных. Однако на текущий момент ни одна графовая база данных не поддерживает запросы с контекстно-свободными ограничениями, что препятствует развитию прикладных решений. В данной работе представлено решение данной проблемы: реализована поддержка расширенного необходимыми конструкциями языка запросов Cypher для графовой базы данных RedisGraph.
\end{abstract}

\phantomsection
\section*{Введение}

Реляционное программирование~--- это чистая форма логического программирования,
в которой программы представляются как наборы математических отношений~\cite{byrdMK}.
Отношения
не различают входные и выходные параметры, из-за чего одно и то же
отношение может решать несколько связанных проблем. К примеру, отношение, задающее
интерпретатор языка, можно использовать не только для вычисления программ по
заданному входу, но и для генерации возможных входных значений по заданному результату
или самих программ по спецификации входных и выходных значений.

miniKanren~--- это семейство встраиваемых предметно-ориентированных языков программирования~\cite{byrdMK}.
miniKanren был специально сконструирован для поддержки реляционной парадигмы,
опираясь на опыт логических языков, таких как языки семейства Prolog~\cite{logicMJ},
Mercury~\cite{mercury} и Curry~\cite{curry}.

Реляционная парадигма довольно сложна, хотя потенциал её весьма велик.
Часто наиболее естественный способ записи отношения не является эффективным. В
частности, при задании функциональных отношений как сопоставления выходов
входам, как это наблюдается в примере с интерпретатором, поиск входов по выходам практически
всегда работает медленно.

Специализация --- это техника автоматической оптимизации программ,
при которой на основе программы и её частично известного входа
порождается новая, более эффективная программа, которая сохраняет семантику
исходной. Для специализации логических языков используются методы частичной дедукции~\cite{advanced},
самый проработанный из которых --- это \cpd\cite{cpd}. ECCE, реализация \forcpd для Prolog, показывает
хорошие результаты~\cite{controlPoly},
% однако специфика реляционного программирования и его отличия от логических языков подразумевает возможность разработать более подходящий
однако, в силу различий между реляционным и логическим программированием, можно предположить возможность разработать более подходящий
метод специализации. Уже существует адаптация \forcpd для miniKanren~\cite{lozov},
однако её результаты нестабильны: несмотря на то, что в некоторых случаях
производительность программ улучшается, в других -- она может существенно ухудшиться.

Другой подход для специализации --- это суперкомпиляция,
техника автоматической трансформации и анализа программ,
при которой программа символьно исполняется с сохранением истории вычислений,
на основе которой строится оптимизированная версия кода.
Суперкомпиляция успешно применяется к функциональным и императивным языкам,
однако для логических языков не сильно развита. Существуют
работы, посвящённые демонстрации сходства процессов частичной дедукции и суперкомпиляции~\cite{pdAndDriving},
а также предназначенный для Prolog суперкомпилятор APROPOS~\cite{apropos}, который, однако, довольно ограничен
в своих возможностях и требует ручного контроля.

В данной работе предлагается способ адаптации и реализации суперкомпилятора для
реляционного языка miniKanren, а также рассматриваются его возможные вариации, приводящие к
дальнейшему повышению производительности реляционных программ, и производится экспериментальное
исследование результата.

\input{Terekhov/diploma_bib}

%\title{Разработка матричного алгоритма поиска путей с контекстно-свободными ограничениями для RedisGraph}
\titlerunning{Поиск путей с КС ограничениями для RedisGraph}

\author{Терехов Арсений Константинович}
\authorrunning{Терехов~А.~К.}

\tocauthor{Терехов~А.~К.}
\institute{Санкт-Петербургский государственный университет\\
	\email{simpletondl@yandex.ru}}

\maketitle

\begin{abstract}
Поиск путей с контекстно-свободными ограничениями подразумевает использование контекстно-свободной грамматики для задания ограничений на множество искомых путей в графе. Данные ограничения используются в таких областях, как статический анализ кода и анализ RDF-данных. Однако на текущий момент ни одна графовая база данных не поддерживает запросы с контекстно-свободными ограничениями, что препятствует развитию прикладных решений. В данной работе представлено решение данной проблемы: реализована поддержка расширенного необходимыми конструкциями языка запросов Cypher для графовой базы данных RedisGraph.
\end{abstract}

\phantomsection
\section*{Введение}

Реляционное программирование~--- это чистая форма логического программирования,
в которой программы представляются как наборы математических отношений~\cite{byrdMK}.
Отношения
не различают входные и выходные параметры, из-за чего одно и то же
отношение может решать несколько связанных проблем. К примеру, отношение, задающее
интерпретатор языка, можно использовать не только для вычисления программ по
заданному входу, но и для генерации возможных входных значений по заданному результату
или самих программ по спецификации входных и выходных значений.

miniKanren~--- это семейство встраиваемых предметно-ориентированных языков программирования~\cite{byrdMK}.
miniKanren был специально сконструирован для поддержки реляционной парадигмы,
опираясь на опыт логических языков, таких как языки семейства Prolog~\cite{logicMJ},
Mercury~\cite{mercury} и Curry~\cite{curry}.

Реляционная парадигма довольно сложна, хотя потенциал её весьма велик.
Часто наиболее естественный способ записи отношения не является эффективным. В
частности, при задании функциональных отношений как сопоставления выходов
входам, как это наблюдается в примере с интерпретатором, поиск входов по выходам практически
всегда работает медленно.

Специализация --- это техника автоматической оптимизации программ,
при которой на основе программы и её частично известного входа
порождается новая, более эффективная программа, которая сохраняет семантику
исходной. Для специализации логических языков используются методы частичной дедукции~\cite{advanced},
самый проработанный из которых --- это \cpd\cite{cpd}. ECCE, реализация \forcpd для Prolog, показывает
хорошие результаты~\cite{controlPoly},
% однако специфика реляционного программирования и его отличия от логических языков подразумевает возможность разработать более подходящий
однако, в силу различий между реляционным и логическим программированием, можно предположить возможность разработать более подходящий
метод специализации. Уже существует адаптация \forcpd для miniKanren~\cite{lozov},
однако её результаты нестабильны: несмотря на то, что в некоторых случаях
производительность программ улучшается, в других -- она может существенно ухудшиться.

Другой подход для специализации --- это суперкомпиляция,
техника автоматической трансформации и анализа программ,
при которой программа символьно исполняется с сохранением истории вычислений,
на основе которой строится оптимизированная версия кода.
Суперкомпиляция успешно применяется к функциональным и императивным языкам,
однако для логических языков не сильно развита. Существуют
работы, посвящённые демонстрации сходства процессов частичной дедукции и суперкомпиляции~\cite{pdAndDriving},
а также предназначенный для Prolog суперкомпилятор APROPOS~\cite{apropos}, который, однако, довольно ограничен
в своих возможностях и требует ручного контроля.

В данной работе предлагается способ адаптации и реализации суперкомпилятора для
реляционного языка miniKanren, а также рассматриваются его возможные вариации, приводящие к
дальнейшему повышению производительности реляционных программ, и производится экспериментальное
исследование результата.

\input{Terekhov/diploma_bib}


\end{document}
