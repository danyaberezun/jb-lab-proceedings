\section*{Приложение А}

%\begin{figure}[h!]
%\begin{lstlisting}
%doubleAppend z1 z2 z3 z4 =
%  (z1 $\equiv$ nil () $\land$ app3 z2 z3 z4) $\lor$
%  fresh (fE fB fA)
%    (z1 $\equiv$ fA :: fB $\land$
%    (z4 $\equiv$ fA :: fE) $\land$
%    appD fB z2 z3 fE)
%
%appD z1 z2 z3 z4 =
%  (z1 $\equiv$ nil ()  $\land$ app3 z2 z3 z4) $\lor$
%  fresh (fE fB fA)
%   (z1 $\equiv$ fA :: fB $\land$
%   (z4 $\equiv$ fA :: fE) $\land$
%    appD fB z2 z3 fE)
%
%app3 z1 z2 z3 =
%  (z1 $\equiv$ nil () $\land$ (z2 $\equiv$ z3)) $\lor$
%  fresh (fD fC fB fA)
%    (z1 $\equiv$ fA :: fB $\land$
%    (z3 $\equiv$ fA :: fD) $\land$
%    app3 fB z2 fD)
%\end{lstlisting}
%\caption{Специализированная системой ECCE программа \lstinline{doubleAppend}}
%\end{figure}


\begin{figure}[h!]
\begin{lstlisting}[basicstyle=\scriptsize]{language=Haskell}
generateGraph :: Int -> Int -> Int -> [(Int, Int)]
generateGraph n m =
   evalRand (do tree <- forM [1..n-1] $\textdollar$ \i ->
                   getEdge i           <$\textdollar$>
                   getRandomR (0, i-1) <*>
                   getRandom
                rest <- replicateM (m - n + 1) $\textdollar$
                   getRandomR (0, n-2) >>=
                     \i -> getEdge i             <$\textdollar$>
                           getRandomR (i+1, n-1) <*>
                           getRandom
                pure $\textdollar$ tree ++ rest) . mkStdGen
    where getEdge :: Int -> Int -> Bool -> (Int, Int)
          getEdge i j True = (i, j)
          getEdge i j False = (j, i)
\end{lstlisting}
\caption{Генератор случайного ориентированного графа с заданными параметрами.}
\label{fig:graphGen}
\end{figure}
