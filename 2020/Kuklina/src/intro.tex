\phantomsection
\section*{Введение}

Реляционное программирование~--- это чистая форма логического программирования,
в которой программы представляются как наборы математических отношений~\cite{byrdMK}.
Отношения
не различают входные и выходные параметры, из-за чего одно и то же
отношение может решать несколько связанных проблем. К примеру, отношение, задающее
интерпретатор языка, можно использовать не только для вычисления программ по
заданному входу, но и для генерации возможных входных значений по заданному результату
или самих программ по спецификации входных и выходных значений.

miniKanren~--- это семейство встраиваемых предметно-ориентированных языков программирования~\cite{byrdMK}.
miniKanren был специально сконструирован для поддержки реляционной парадигмы,
опираясь на опыт логических языков, таких как языки семейства Prolog~\cite{logicMJ},
Mercury~\cite{mercury} и Curry~\cite{curry}.

Реляционная парадигма довольно сложна, хотя потенциал её весьма велик.
Часто наиболее естественный способ записи отношения не является эффективным. В
частности, при задании функциональных отношений как сопоставления выходов
входам, как это наблюдается в примере с интерпретатором, поиск входов по выходам практически
всегда работает медленно.

Специализация --- это техника автоматической оптимизации программ,
при которой на основе программы и её частично известного входа
порождается новая, более эффективная программа, которая сохраняет семантику
исходной. Для специализации логических языков используются методы частичной дедукции~\cite{advanced},
самый проработанный из которых --- это \cpd\cite{cpd}. ECCE, реализация \forcpd для Prolog, показывает
хорошие результаты~\cite{controlPoly},
% однако специфика реляционного программирования и его отличия от логических языков подразумевает возможность разработать более подходящий
однако, в силу различий между реляционным и логическим программированием, можно предположить возможность разработать более подходящий
метод специализации. Уже существует адаптация \forcpd для miniKanren~\cite{lozov},
однако её результаты нестабильны: несмотря на то, что в некоторых случаях
производительность программ улучшается, в других -- она может существенно ухудшиться.

Другой подход для специализации --- это суперкомпиляция,
техника автоматической трансформации и анализа программ,
при которой программа символьно исполняется с сохранением истории вычислений,
на основе которой строится оптимизированная версия кода.
Суперкомпиляция успешно применяется к функциональным и императивным языкам,
однако для логических языков не сильно развита. Существуют
работы, посвящённые демонстрации сходства процессов частичной дедукции и суперкомпиляции~\cite{pdAndDriving},
а также предназначенный для Prolog суперкомпилятор APROPOS~\cite{apropos}, который, однако, довольно ограничен
в своих возможностях и требует ручного контроля.

В данной работе предлагается способ адаптации и реализации суперкомпилятора для
реляционного языка miniKanren, а также рассматриваются его возможные вариации, приводящие к
дальнейшему повышению производительности реляционных программ, и производится экспериментальное
исследование результата.
