
На основе вышесказанного можно построить обобщённый алгоритм суперкомпиляции,
который в псевдокоде представлен на рисунке~\ref{fig:scalgogen}.

\begin{figure}[h!]
% обобщение вверх
% else if $\exists$ parent: parent $\genup$ configuration
% then
%    node $\larrow$ generalize(configuration, parent)
%    addUp(env, tree, parent, node)

\begin{lstlisting}[escapechar=@]
supercomp(program, query):
  env $\larrow$ createEnv program
  configuration $\larrow$ initialize query
  graph $\larrow$ emptyTree
  drive(env, graph, configuration)
  return residualize graph @\label{line:resid}@

drive(env, graph, configuration): @\label{line:env}@
  if configuration is empty  @\label{line:success}@
  then add(env, graph, success node)
  else if $\exists$ parent: configuration $\variant$ parent @\label{line:renaming}@
  then add(env, graph, renaming node)
  else if $\exists$ parent: parent $\embed^+$ configuration @\label{line:abstraction}@
  then
    add(env, graph, abstraction node)
    children $\larrow$ generalize(configuration, parent) @\label{line:gen}@
    $\forall \text{child} \in \text{children}:$
      drive(env, graph, child)
  else @\label{line:unfolding}@
    add(env, graph, unfolding node)
    children $\larrow$ unfold(env, configuration) @\label{line:unfold}@
    $\forall \text{child} \in \text{children}:$
      drive(env, graph, child)
\end{lstlisting}
\caption{Обобщённый алгоритм суперкомпиляции.}
\label{fig:scalgogen}
\end{figure}

Алгоритм суперкомпиляции принимает на себя программу и запрос,
на который необходимо специализировать программу, и после
инициализации начальных значений, включающих в себя некоторое
\emph{окружения программы} (\lstinline{env} на строке~\ref{line:env}),
в котором хранятся все вспомогательные структуры, запускает процесс прогонки.
Прогонка производится до схождения и производит следующие действия в
зависимости от состояния:
\begin{itemize}
\item если конфигурация пустая (строка~\ref{line:success}), это означает, что вычисления
      успешно сошлись в конкретную подстановку. В таком случае происходит добавление
      в граф листового узла с этой подстановкой;
\item если существует такая родительская конфигурация, что она является вариантом текущей (строка~\ref{line:renaming}),
      то происходит свёртка и в граф добавляется листовой узел с ссылкой на родителя;
\item если же среди родителей находится такой, на котором срабатывает свисток (строка~\ref{line:abstraction}),
      тогда производится обобщение, порождающее дочерние конфигурации (строка~\ref{line:gen}),
      на которых продолжается процесс прогонки;
\item иначе происходит шаг символьного вычисления (строка~\ref{line:unfolding}), на котором
      порождаются конфигурации.

\end{itemize}

%%%%%%%%%%%%%%%%%%%%%%%%%%%%%%%%%%%%%%%%%%%%%%%%%%%%%%%%%%%%%%%%%%%
%%%%%%%%%%%%%%%%%%%%%%%%%%%%%%%%%%%%%%%%%%%%%%%%%%%%%%%%%%%%%%%%%%%
% Описание окружения
%%%%%%%%%%%%%%%%%%%%%%%%%%%%%%%%%%%%%%%%%%%%%%%%%%%%%%%%%%%%%%%%%%%
%%%%%%%%%%%%%%%%%%%%%%%%%%%%%%%%%%%%%%%%%%%%%%%%%%%%%%%%%%%%%%%%%%%

\emph{Окружение} для суперкомпиляции должно сохранять следующие объекты:
\begin{itemize}
\item подстановку, в которой содержатся все накопленные непротиворечивые унификации,
      необходимую в процессе прогонки для проверки новых унификаций;
\item первую свободную семантическую переменную, необходимую для генерации свежих переменных,
      к примеру, при абстракции;
\item определение программы, необходимое для замены вызова на его тело при развёртывании.
\end{itemize}

После завершения этапа прогонки из графа процессов извлекается остаточная программа
(строка~\ref{line:resid}).
