\section{Анализ времени связывания}
\label{annotator}

Данная глава посвящена анализу времени связывания для языка \miniKanren{}.
В первом разделе рассказано об адаптации идей анализа времени связывания для определения направления вычислений.
Второй раздел вводит понятие программы в \textit{нормальной форме} --- программы, содержащей меньшее количество конструкций --- необходимой для анализа времени связывания.
Алгоритм аннотирования для программ в нормальной форме описывается в третьем разделе.
Несколько примеров аннотирования --- в четвертом разделе.
Пятый раздел посвящен разрешению проблем алгоритма аннотирования нормализованной программы.
В шестом разделе предлагаются способы нормализации программ с учётом необходимости последующего аннотирования.
Корректность предложенного алгоритма доказана в седьмом разделе.

\subsection{Анализ времени связывания для \miniKanren{}}

Цель анализа времени связывания --- указать порядок, в котором имена связываются со значениями.
Алгоритм принимает на вход программу на \miniKanren{} и данные о том, какие переменные считаются входными. 
В результате работы алгоритма каждой переменной ставится в соответствие положительное число, обозначающее время связывания этой переменной.
Мы будем называть процесс подбора чисел \emph{аннотированием}, а сам алгоритм --- алгоритмом анализа времени связывания или алгоритмом аннотирования.

Если о переменной ничего неизвестно, она аннотируется $Undef$; иначе указывается время связывания: целое положительное число.
В начале работы алгоритма известными являются переменные, указанные как входные --- они аннотируются числом $0$.
Если переменная унифицируется с константой (термом, не содержащим свободных переменных), то мы считаем её временем связывания $1$.
Если переменная унифицируется с термом, каждая свободная переменная которого аннотирована, мы аннотируем эту переменную числом $1+n$, где $n$ --- максимальная аннотация свободных переменных терма. 
Таким образом мы распространяем информацию о времени связывания на непроаннотированные переменные.

На аннотациях имеется порядок --- естественный порядок на положительных числах, при этом $Undef$ считается меньше любой числовой аннотации.
Аннотация никогда не заменяется на меньшую.
Полурешётка на аннотациях изображена на рисунке~\ref{fig:semilattice}.

\begin{figure}[htbp]
  \centering
  \begin{tikzpicture}
    \draw (0,0) node[below] {Undef};
    \draw (0,0) -- (0,.5);
    \draw (0,.5) node[above] {0};
    \draw (0,1.25) -- (0,1.75);
    \draw (0,1.75) node[above] {1};
    \draw (0,2.5) -- (0,3);
    \draw (0,3) node[above] {2};
    \draw[dotted] (0,3.75) -- (0,4.25);
  \end{tikzpicture}
  \caption{Полурешетка на аннотациях}
  \label{fig:semilattice}
\end{figure}

\subsection{Понятие нормальной формы}
\label{lab:normform}

\emph{Нормальная форма отношения на \miniKanren{}} --- представление отношения на \miniKanren{}, в котором все свободные переменные введены в область видимости на самом верхнем уровне; цель --- дизъюнкцию конъюнкций вызовов отношений или унификаций термов; отсутствуют унификации двух конструкторов.
Соответствующий абстрактный синтаксис приведен на рисунке~\ref{fig:normMiniKanren}.

\begin{figure}[h!]
    \begin{center}
    \begin{minipage}{0.5\textwidth}
    \begin{align*}
      Goal  &: \underline{fresh} \ [Name] \ (\bigvee \bigwedge Goal') \\
      Goal' &: \underline{call} \ Name \ [Var] \\
            &\mid Var \equiv Term \\
      Term  &: Var \\ 
            &\mid \underline{cons} \ Name \ [Term]
    \end{align*}
    \end{minipage}
    \end{center}
  \caption{Абстрактный синтаксис нормализованной программы на \miniKanren{}}
  \label{fig:normMiniKanren}
\end{figure}

\emph{Нормальная форма программы на \miniKanren{}} --- представление программы на \miniKanren{}, при котором цель программы является нормализованным отношением; цель каждого определения является нормализованным отношением. 

Важно заметить, что любое отношение \miniKanren{} можно преобразовать в нормальную форму.
Доказательство данного утверждения приведено в разделе~\ref{lab:normProof}.

Рассмотрим отличия нормализованной программы от ненормализованной.
\begin{itemize}
    \item Тело определения находится в дизъюнктивной нормальной форме.
    \item Все свободные переменные введены при помощи $fresh$ на самом верхнем уровне.
    \item Не существует унификаций термов-конструкторов.
    \item Не существует вызовов с аргументами-конструкторами.
\end{itemize}

Такие ограничения вводятся с целью упрощения процесса аннотирования и трансляции в целом.
\begin{itemize}
    \item ДНФ тела позволяет уменьшить глубину вложенности программы.
    \item $fresh$-цель задаёт область видимости вычислений и позволяет использовать одинаковые имена переменных в различных областях видимости --- её наличие только на самом верхнем уровне означает, что все переменные принадлежат одной области видимости.
    \item Отсутствие унификаций термов-конструкторов позволяет не производить очевидной унификации в процессе выполнения алгоритма.
    \item Отсутствие вызовов на термах-конструкторах позволяет избежать неопределённости в процессе аннотирования.
\end{itemize}

\subsection{Алгоритм аннотирования нормализованной программы}
\label{lab:coreAnn}

Ниже описывается базовый алгоритм аннотирования нормализованной программы на \miniKanren{}, псевдокод которого представлен на рисунке~\ref{alg:annotate}.
Вспомогательные функции $annotateDisj$, $annotateUnification$ и $annotateInvoke$ приведены на рисунках~\ref{alg:annotateDisj},~\ref{alg:annotateUnification} и~\ref{alg:annotateInvoke} соответственно.

\begin{figure}[h!]
  \begin{center}
  \begin{minipage}{1.1\textwidth}
\begin{algorithm}[H]
  zdes algorithm
  % \KwIn{($goal$,~$scope$) --- нормализованная программа на \miniKanren{} (цель и список отношений); $inVars$ --- список входных переменных}
  % \KwOut{$goal$ --- проаннотированная цель;~$stack$ --- стек вызовов}
  % $stack \gets []$\;
  % \For {$var~\KwFrom~goal$} {
  %   \eIf {$var \in inVars$} {
  %     $var \gets (var,~0)$
  %   }{
  %     $var \gets (var,~Undef)$
  %   }
  % }
  % \For {$disj~\KwFrom~goal$} {
  %   $disj \gets moveUnifsBeforeInvokes(disj)$\;
  %   $(disj,~stack) \gets annotateDisj(disj,~stack)$
  % }
  % \Return {$(goal,~stack)$}
\end{algorithm}
  \end{minipage}
  \end{center}
  \caption{Алгоритм $annotate$ для аннотирования нормализованной программы на \miniKanren{}}
  \label{alg:annotate}
\end{figure}

Алгоритм аннотирования $annotate$ получает на вход нормализованную программу на \miniKanren{} (цель и список определений), а также список входных переменных.
По окончанию его работы будут получены проаннотированная цель и ассоциативный массив, содержащий проаннотированные определения, требующихся для вычисления цели.
Ассоциативный массив представляет собой отображение пары имя-направление отношения в проаннотированную цель --- тело отношения.
Мы будем называть этот массив \emph{стеком вызовов}, потому что в нем будут находиться вызываемые отношения.

\emph{Успешным результатом аннотирования} назовём ситуацию, когда получившийся по окончании выполнения алгоритма стек вызовов удовлетворяет следующим условиям:
\begin{itemize}
    \item Все отношения, требуемые для вычисления цели программы, присутствуют в стеке;
    \item Все переменные отношений, присутствующих в стеке вызовов, проаннотированы числом.
\end{itemize}

При инициализации алгоритма выполняются следующие действия:
\begin{itemize}
    \item Все входные переменные аннотируются $0$;
    \item Создается пустой стек вызовов.
\end{itemize}

\begin{figure}[h!]
  \begin{center}
  \begin{minipage}{1.1\textwidth}
\begin{algorithm}[H]
  % \KwIn{$disj$ --- дизъюнкт; $stack$ --- стек вызовов}
  % \KwOut{$disj$ --- проаннотированный дизъюнкт; $stack$ --- стек вызовов}
  % \While {$not(isFixedPointReached(disj,~stack))$} {
  %   \For {$conj~\KwFrom~disj$} {
  %     \Switch{$conj$} {
  %       \Case{$unif \gets isUnification(conj)$}{
  %         $(conj,~stack) \gets annotateUnification(unif)$
  %       }
  %       \Case{$invoke \gets isInvoke(conj)$}{
  %         $(conj,~stack) \gets annotateInvoke(invoke,~stack,~scope)$
  %       }
  %     }
  %     \For {$(conjVar,~conjAnn)~\KwFrom~conj$} {
  %       \For {$(disjVar,~disjAnn)~\KwFrom~disj$} {
  %         \If {$disjAnn = Undef~\KwAnd~disjVar = conjVar$} {
  %           $disjAnn \gets conjAnn$
  %         }
  %       }
  %     }
  %   }
  % }
  % \Return {$(disj,~stack)$}
\end{algorithm}
  \end{minipage}
  \end{center}
  \caption{Алгоритм $annotateDisj$ для аннотирования дизъюнкта}
  \label{alg:annotateDisj}
\end{figure}

Для аннотации цели в ДНФ необходимо проаннотировать все её дизъюнкты.
Аннотация дизъюнкта $annotateDisj$ (см. рисунок~\ref{alg:annotateDisj}) осуществляется итеративно, пока не будет достигнута неподвижная точка кода, описывающего шаг аннотирования.
За один шаг аннотируется хотя бы одна конъюнкция (унификация или вызов отношения).
Если в течение шага ни одна новая переменная не была проаннотирована, считается, что достигнута неподвижная точка.

Конъюнкты аннотируются в заранее определенном порядке: cначала мы аннотируем унификации, а затем вызовы отношений.
Данный порядок задает функция $moveUnifsBeforeInvokes$ на рисунке~\ref{alg:annotate}.
Аннотации переменных в дизъюнкте должны согласовываться: одна и та же переменная в конъюнктах одного дизъюнкта должна иметь одну и ту же аннотацию.
Каждый раз при аннотации новой переменной необходимо установить ту же аннотацию всем другим вхождениям этой переменной в дизъюнкте.

Для того, чтобы аннотировать конъюнкцию необходимо аннотировать все ее конъюнкты, то есть унификации и вызовы отношения.
Об этом будет рассказано в следующих разделах.
%%%%%%%%%%%%%%%%%%%%%%%%%%%%%%%%%%%%%%%%%%%%%%%%%%%%%%%%%%%%%%%%%%%%%%%%%%%%%%%%%%%%%%%%%%%%%%%%%%%%%%%%%%%%%%%%%%%%%%%

\subsubsection{Алгоритм аннотирования унификаций}

Псевдокод алгоритма аннотирования унификаций представлен на рисунке~\ref{alg:annotateUnification}.

При аннотировании унификаций возможны следующие случаи (здесь и далее аннотация переменной указывается в верхнем индексе).
\begin{itemize}
    \item Унификация имеет вид $x^{Undef} \equiv t[y_0^{i_0}, \dots, y_k^{i_k}]$, то есть переменная, имеющая аннотацию $Undef$, унифицируется с термом $t$ со свободными переменными $y_j^{i_j}$ с целочисленными аннотациями $i_j$. В таком случае переменной $x$ необходимо присвоить аннотацию $n + 1$, где $n = max \{ i_0, \dots i_k\}$ (в псевдокоде на рисунке~\ref{alg:annotateUnification} --- функция $getMaxAnnotation$).
    \item Переменная, аннотированная числом, унифицируется с термом: $x^{n} \equiv t[y_0^{i_0}, \dots, y_k^{i_k}]$; некоторые свободные переменные терма проаннотированны $Undef$.
    Тогда всем переменным $y_j^{Undef}$ присваивается аннотация $n+1$ при помощи функции $replaceUndefWith$.
    \item Остальные случаи симметричны.
\end{itemize}

\begin{figure}[h!]
  \begin{center}
  \begin{minipage}{1\textwidth}
\begin{algorithm}[H]
  % \KwIn{$unif$ --- унификация}
  % \KwOut{$unif$ --- унификация}
  % $(left,~right) \gets unif$\;
  % \Switch{$(left,~right)$} {
  %   \Case{$((var,~ann) \gets isUndefVariable(left),~\_)$} {
  %     $ann \gets getMaxAnnotation(right) + 1$
  %   }
  %   \Case{$((var,~ann) \gets isVariable(left),~\_)$} {
  %     $right \gets replaceUndefWith(ann + 1,~right)$
  %   }
  %   \Other{
  %     $//~symmetric~cases$\;
  %     $\dots$
  %   }
  % }
  % \Return {$unif$}
\end{algorithm}
  \end{minipage}
  \end{center}
  \caption{Алгоритм $annotateUnification$ для аннотирования унификации}
  \label{alg:annotateUnification}
\end{figure}

%%%%%%%%%%%%%%%%%%%%%%%%%%%%%%%%%%%%%%%%%%%%%%%%%%%%%%%%%%%%%%%%%%%%%%%%%%%%%%%%%%%%%%%%%%%%%%%%%%%%%%%%%%%%%%%%%%%%%%%

\subsubsection{Аннотирование вызовов отношений}

Аннотирование вызовов отношения состоит из двух частей:
\begin{itemize}
    \item аннотирования тела вызываемого отношения в соответствии с направлением вызова (опционально);
    \item аннотирования аргументов самого вызова отношения.
\end{itemize}
Псевдокод алгоритма приведен на рисунке~\ref{alg:annotateInvoke}.

\begin{figure}[h!]
  \begin{center}
  \begin{minipage}{1\textwidth}
\begin{algorithm}[H]
  % \KwIn{$invoke$ --- вызов отношения; $stack$ --- стек вызовов; $scope$ --- список определений}
  % \KwOut{$invoke$ --- вызов отношения; $stack$ --- стек вызовов}
  % $(name,~terms) \gets invoke$\;
  % $invokeDirection \gets makeInvokeDirection(terms)$\;
  % $stackKey \gets (name,~invokeDirection)$\;
  % \If {$NameDirectionAreNotInStack(stackKey,~stack)$} {
  %   $inVars \gets []$\;
  %   \For {$(var,~ann) \gets invokeDirection$} {
  %     \If {$ann~=~0$} {
  %       $inVars \gets var~:~inVars$
  %     }
  %   }
  %   $body \gets getBodyByName(name,~scope)$\;
  %   $stack \gets insert(stack,~stackKey,~null)$\;
  %   $body \gets annotation(body,~inVars)$\;
  %   $program \gets (body,~scope)$\;
  %   $(body,~stack) \gets annotate(program,~inVars)$\;
  %   $stack \gets insert(stack,~stackKey,~body)$
  % }
  % $terms \gets replaceUndefWith(getMaxAnnotation(terms) + 1,~terms)$\;
  % \Return {$(invoke,~stack)$}
\end{algorithm}
  \end{minipage}
  \end{center}
  \caption{Алгоритм $annotateInvoke$ для аннотирования вызова отношения}
  \label{alg:annotateInvoke}
\end{figure}

Запускать алгоритм аннотирования тела вызываемого отношения нужно только в случае, если это ещё не было сделано для данного направления.
Чтобы определить необходимость аннотирования тела вызова, по имени вызова и его направлению проверим наличие согласованного направления в стеке вызовов.
Два направления назовем \emph{согласованными}, если аннотации их аргументов попарно равны.
Если согласованного направления не нашлось, запустим аннотирование тела вызываемого отношения.

Получим направление вызова.
Для этого аннотации аргументов обнуляются: числовые аннотации становятся $0$, а $Undef$ --- $1$.
Для вызываемого отношения не важен момент времени в прошлом, когда его входные переменные стали известны --- для него они все стали известны в момент времени $0$.
В то же время по возвращении из вызова все $Undef$ переменные станут известны --- для вызывающего отношения это следующий момент за моментом вызова.

Aннотирование тела вызываемого отношения состоит из следующих шагов:
\begin{itemize}
    \item получение входных переменных по направлению вызова;
    \item получение тела вызываемого отношения из списка определений программы при помощи функции $getBodyByName$;
    \item вставки имени и направления в стек вызовов (однако, соответствующее им тело отношение отсутствует: оно будет проаннотировано на следующем шаге и будет добавлено в стек вызовов позже);
    \item запуск алгоритма аннотирования $annotate$ (см. рисунок~\ref{alg:annotate}) для тела вызываемого отношения на обновлённом стеке вызовов;
    \item обновление стека вызовов: по имени и направлению в стек вызовов помещается тело после аннотирования.
\end{itemize}

Добавление в стек вызовов информации о ранее проаннотированных в конкретных направлениях отношениях позволяет избежать повторного аннотирования.
В частности, помогает не получить бесконечный цикл при аннотировании рекурсивного вызова.

Для аннотирования аргументов вызова отношения необходимо заменить $Undef$-аннотации переменных на $n+1$, где $n$ --- максимальная аннотация переменных-аргументов вызова.
В псевдокоде на рисунке~\ref{alg:annotateInvoke} для этого используются функции $replaceUndefWith$ и $getMaxAnnotation$.
Это верно, потому что после завершения вызова мы считаем, что все $Undef$-переменные стали известны из вызываемого отношения.
При этом, так как при аннотировании дизъюнкта сначала аннотируются все унификации, а затем --- все вызовы отношений, можно утверждать, что, к моменту аннотирования первого по порядку вызова отношения будут известны все возможные переменные.
Случай нескольких вызовов отношений в одном дизъюнкте рассматривается дополнительно в разделе~\ref{lab:disjPerm}.

\subsection{Примеры аннотирования}

В этом разделе приведено несколько примеров процесса аннотирования отношений.
Числа над переменными обозначают аннотации.

%%%%%%%%%%%%%%%%%%%%%%%%%%%%%%%%%%%%%%%%%%%%%%%%%%%%%%%%%%%%%%%%%%%%%%%%%%%%%%%%

\subsubsection{Отношение $append^o \ x \ y \ ?$}

$append^o$ --- отношение связывающее три списка, первые два из которых являются конкатенацией третьего.
Его аннотирование в направлении $append^o \ x \ y \ ?$ представлено на рисунке~\ref{lst:appendoIIOANN}.

\begin{figure}[h!]
  \begin{center}
  \begin{minipage}{0.5\textwidth}
  \begin{lstlisting}[language=Haskell, frame=single, numbers=left,numberstyle=\small, firstnumber=8, escapechar=|]
 $append^o$ $x^0$ $y^0$ $z^1$ =
   ($x^0$ $\equiv$ [] $\wedge$ $y^0$ $\equiv$ $z^1$) $\vee$ |\label{line:appendoIIOANN2}|
   ($fresh$ [$h$, $t$, $r$] (
       $x^0$ $\equiv$ $h^1$ : $t^1$ $\wedge$ |\label{line:appendoIIOANN4}|
       $z^3$ $\equiv$ $h^1$ : $r^2$ $\wedge$ |\label{line:appendoIIOANN5}|
       $append^o$ $t^1$ $y^0$ $r^2$          |\label{line:appendoIIOANN6}|
   ))
    \end{lstlisting}
  \end{minipage}
  \end{center}
  \caption{Результат аннотирования отношения $append^o \ x \ y \ ?$}
  \label{lst:appendoIIOANN}
\end{figure}

В данном случае переменные $x$ и $y$ являются входными.
При начале работы алгоритма, такого отношения и направления нет в стеке вызовов, поэтому они добавляются в стек и рекурсивно запускается аннотирование цели $append^o$.
Так как $x$ и $y$ --- входные переменные, их аннотации известны и равны $0$.

В начале аннотирования первого дизъюнкта переменные $x$ и $y$ известны --- остаётся определить $z$.
Аннотация $z$ равна $1$, так как $z$ унифицируется с $y$, аннотация которой --- $0$.

Во втором дизъюнкте аннотации $h$ и $t$ в строке~\ref{line:appendoIIOANN4} можно установить, так как известна аннотация $x$.
Аннотация $h$ распространяется на~\ref{line:appendoIIOANN5} строку, а аннотация $t$ --- на~\ref{line:appendoIIOANN6} строку.
Рекурсивный вызов отношения в строке ~\ref{line:appendoIIOANN6} согласован с имеющимся в стеке, поэтому можно проаннотировать переменную $r$.
Распространяем аннотацию $r$ в строке~\ref{line:appendoIIOANN5}.
На последнем шаге аннотируем $z$ в строке~\ref{line:appendoIIOANN4}.

%%%%%%%%%%%%%%%%%%%%%%%%%%%%%%%%%%%%%%%%%%%%%%%%%%%%%%%%%%%%%%%%%%%%%%%%%%%%%%%%

\subsubsection{Отношение $append^o \ ? \ ? \ z$}

В случае аннотирования $append^o \ ? \ ? \ z$ входной считается переменная $z$ (см. рисунок~\ref{lst:appendoOOIANN}).

\begin{figure}[h!]
  \begin{center}
  \begin{minipage}{0.5\textwidth}
  \begin{lstlisting}[language=Haskell, frame=single, numbers=left,numberstyle=\small, firstnumber=15, escapechar=|]
 $append^o$ $x^1$ $y^1$ $z^0$ =
   ($x^1$ $\equiv$ [] $\wedge$ $y^1$ $\equiv$ $z^0$) $\vee$ |\label{line:appendoOOIANN2}|
   ($fresh$ [$h$, $t$, $r$] (
       $x^3$ $\equiv$ $h^1$ : $t^2$ $\wedge$ |\label{line:appendoOOIANN4}|
       $z^0$ $\equiv$ $h^1$ : $r^1$ $\wedge$ |\label{line:appendoOOIANN5}|
       $append^o$ $t^2$ $y^2$ $r^1$          |\label{line:appendoOOIANN6}|
   ))
    \end{lstlisting}
  \end{minipage}
  \end{center}
  \caption{Результат аннотирования отношения $append^o \ ? \ ? \ z$}
  \label{lst:appendoOOIANN}
\end{figure}

Пусть $append^o$ уже в стеке и $z$ проаннотирована.
В первом дизъюнкте $x$ и $y$ имеют аннотацию~$1$: $y$ унифицируется со входной переменной $z$, а $x$ --- с константой.
Во втором дизъюнкте на первом шаге становятся известны аннотации $h$ и $r$ (строка~\ref{line:appendoOOIANN5}).
Аннотация $r$ распространяется на строку~\ref{line:appendoOOIANN6}.
Отношение с согласованным направлением есть в стеке, поэтому можно аннотировать $t$ и $y$.
Далее аннотация $t$ распространяется на строку~\ref{line:appendoOOIANN4}, и на последнем шаге аннотируется $x$.

%%%%%%%%%%%%%%%%%%%%%%%%%%%%%%%%%%%%%%%%%%%%%%%%%%%%%%%%%%%%%%%%%%%%%%%%%%%%%%%%

\subsubsection{Отношение $revers^o \ ? \ y$}

Ещё один пример --- отношение $revers^o$.
Оно связывает два списка, получающиеся переворачиванием друг друга.
Его определение приведено в листинге~\ref{lst:reversoOIANN}.

\begin{figure}[h!]
  \begin{center}
  \begin{minipage}{0.5\textwidth}
  \begin{lstlisting}[language=Haskell, frame=single, numbers=left,numberstyle=\small, firstnumber=22, escapechar=|]
 $revers^o$ $x^1$ $y^0$ =
   ($x^1$ $\equiv$ [] $\wedge$ $y^0$ $\equiv$ []) $\vee$ |\label{line:reversoOIANN2}|
   ($fresh$ [$h$, $t$, $r$] (
       $x^5$ $\equiv$ $h^2$ : $t^4$ $\wedge$ |\label{line:reversoOIANN4}|
       $append^o$ $r^3$ $[h^2]$ $y^0$ |\label{line:reversoOIANN5}|
       $revers^o$ $t^4$ $r^3$ $\wedge$ |\label{line:reversoOIANN6}|
   ))
    \end{lstlisting}
  \end{minipage}
  \end{center}
  \caption{Результат аннотирования отношения $revers^o \ ? \ y$}
  \label{lst:reversoOIANN}
\end{figure}

Вызов $revers^o \ ? \ y$ добавляется в стек вызовов, а переменная $y$ инициализируется как $y$ входная.
При аннотировании второго дизъюнкта на первом шаге можно попытаться проаннотировать только вызов $append^o$ в строке~\ref{line:reversoOIANN5} --- известна $y$.
Такого отношения в стеке вызовов нет, поэтому необходимо добавить его и вызвать аннотирование.
Это и есть вызов $append^o \ ? \ ? \ z$, рассмотренный выше (см. рисунок~\ref{lst:appendoOOIANN}).
Аннотирование $append^o$ позволяет определить аннотации переменных $r$ и $h$ и распространить их по другим конъюнктам.
На следующем шаге происходит вычисление аннотации переменной $t$ рекурсивного вызова $revers^o$, так как он уже есть в стеке (см. строку~\ref{line:reversoOIANN6}).
Распространение аннотации $t$ и аннотирование $x$ происходит на следующем шаге в строке~\ref{line:reversoOIANN4}.

\subsection{Проблемы аннотирования нормализованной программы}
\label{lab:problems}

%%%%%%%%%%%%%%%%%%%%%%%%%%%%%%%%%%%%%%%%%%%%%%%%%%%%%%%%%%%%%%%%%%%%%%%%%%%%%%%%%%%%%%%%%%%%%%%

\subsubsection{Несколько вызовов в одном дизъюнкте}
\label{lab:disjPerm}

Существуют отношения, точная аннотация которых невозможна без полного перебора возникающих вариантов или пользовательских аннотаций.
Один из видов таких отношений --- отношения, содержащие несколько вызовов в одном дизъюнкте.
Пример такого отношения приведен на рисунке~\ref{lst:reloDEF}.

\begin{figure}[h!]
  \begin{center}
  \begin{minipage}{0.25\textwidth}
  \begin{lstlisting}[language=Haskell, frame=single, numbers=left,numberstyle=\small, firstnumber=29, escapechar=|]
 $rel^o$ $x$ $y$ $z$ =
   $f^o$ $x$ $y$ $\vee$
   $h^o$ $z$ $y$ $\vee$
   $g^o$ $x$ $z$
    \end{lstlisting}
  \end{minipage}
  \end{center}
  \caption{Пример программы на \miniKanren{} с несколькими вызовами в одном дизъюнкте}
  \label{lst:reloDEF}
\end{figure}

Пусть $y$ --- входная переменная.
В этом случае порядок вычисления вызовов $f^o$ и $h^o$ не зависит друг от друга, но зависит от направления вычисления $g^o$.
Оно не может вычисляться до вычисления $f^o$ и $h^o$ (неизвестны входные переменные), но может вычисляться между ними ($g^o \ x \ ?$ или $g^0 \ ? \ z$) или после (выполнять роль предиката: $g^o \ x \ y$).

Пример такого поведения можно увидеть при аннотировании $revers^o$.
Аннотирование того же определения $revers^o$, что было рассмотрено на рисунке~\ref{lst:reversoOIANN}, в направлении $revers^o \ x \ ?$, приведено на рисунке~\ref{lst:reversoIOANNfail}).

\begin{figure}[h!]
  \begin{center}
  \begin{minipage}{0.5\textwidth}
  \begin{lstlisting}[language=Haskell, frame=single, numbers=left,numberstyle=\small, firstnumber=33, escapechar=|]
 $revers^o$ $x^0$ $y^1$ =
   ($x^0$ $\equiv$ [] $\wedge$ $y^1$ $\equiv$ []) $\vee$ |\label{line:reversoIOANNfail2}|
   ($fresh$ [$h$, $t$, $r$] (
       $x^0$ $\equiv$ $h^1$ : $t^1$ $\wedge$ |\label{line:reversoIOANNfail4}|
       $append^o$ $r^{2}$ $[h^1]$ $y^{2}$ |\label{line:reversoIOANNfail5}|
       $revers^o$ $t^1$ $r^{2}$ $\wedge$ |\label{line:reversoIOANNfail6}|
   ))
    \end{lstlisting}
  \end{minipage}
  \end{center}
  \caption{Результат аннотирования отношения $revers^o \ x \ ?$ с порядком вызовов $append^o$-$revers^o$}
  \label{lst:reversoIOANNfail}
\end{figure}

Во втором дизъюнкте в начале определяются аннотации переменных $h$ и $t$ в строке~\ref{line:reversoIOANNfail4} --- затем они распространяются на последующие вызовы.
Входным является только второй аргумент, поэтому необходимо проаннотировать тело $append^o \ ? \ y \ ?$ (см. рисунок~\ref{lst:appendoOIOANNfail}).
Аннотация $append^o \ ? \ y \ ?$ завершится неудачией: строках~\ref{line:appendoOIOANNfail4} и~\ref{line:appendoOIOANNfail5} остались переменные, проаннотированные $Undef$.
Причина в том, что при таком направлении не существует возможности узнать значения (а, значит, и время связывания) переменных $h$, $x$ и $z$ (данная проблема обсуждается в следующем подразделе).
Аннотирование $revers^o \ x \ ?$ завершится успешно, однако, при трансляции невозможно получить работающую программу, так как неуспешно завершилось аннотирование $append^o \ ? \ y \ ?$.

\begin{figure}[h!]
  \begin{center}
  \begin{minipage}{0.55\textwidth}
  \begin{lstlisting}[language=Haskell, frame=single, numbers=left,numberstyle=\small, firstnumber=40, escapechar=|]
 $append^o$ $x^1$ $y^0$ $z^1$ =
   ($x^1$ $\equiv$ [] $\wedge$ $y^1$ $\equiv$ $z^0$) $\vee$ |\label{line:appendoOIOANNfail2}|
   ($fresh$ [$h$, $t$, $r$] (
       $x^{Undef}$ $\equiv$ $h^{Undef}$ : $t^1$ $\wedge$ |\label{line:appendoOIOANNfail4}|
       $z^{Undef}$ $\equiv$ $h^{Undef}$ : $r^1$ $\wedge$ |\label{line:appendoOIOANNfail5}|
       $append^o$ $t^1$ $y^0$ $r^1$ |\label{line:appendoOIOANNfail6}|
   ))
    \end{lstlisting}
  \end{minipage}
  \end{center}
  \caption{Результат аннотирования отношения $append^o \ ? \ y \ ?$}
  \label{lst:appendoOIOANNfail}
\end{figure}

На рисунке~\ref{lst:reversoIOANN} приведён результат аннотирования в случае другого порядка вызовов: теперь $revers^o$ происходит перед $append^o$.

\begin{figure}[h!]
  \begin{center}
  \begin{minipage}{0.5\textwidth}
  \begin{lstlisting}[language=Haskell, frame=single, numbers=left,numberstyle=\small, firstnumber=47, escapechar=|]
 $revers^o$ $x^0$ $y^1$ =
   ($x^0$ $\equiv$ [] $\wedge$ $y^1$ $\equiv$ []) $\vee$ |\label{line:reversoIOANN2}|
   ($fresh$ [$h$, $t$, $r$] (
       $x^0$ $\equiv$ $h^1$ : $t^1$ $\wedge$ |\label{line:reversoIOANN4}|
       $revers^o$ $t^4$ $r^3$ $\wedge$ |\label{line:reversoIOANN5}|
       $append^o$ $r^3$ $[h^2]$ $y^0$ |\label{line:reversoIOANN6}|
   ))
    \end{lstlisting}
  \end{minipage}
  \end{center}
  \caption{Результат аннотирования отношения $revers^o \ x \ ?$ с порядком вызовов $revers^o$-$append^o$}
  \label{lst:reversoIOANN}
\end{figure}

Строка~\ref{line:reversoIOANN5} содержит рекурсивный вызов того же направления, что и исходное отношение --- так становится известна переменная $r$.
Аннотирование будет успешно завершено аннотированием вызова $append^o \ ? \ ? \ z$ (см. рисунок~\ref{lst:appendoOOIANN}) в строке~\ref{line:reversoIOANN6}.

Для решения проблемы нескольких вызовов в одном дизъюнкте предложено решение с перестановками конъюнктов.
Для каждого дизъюнкта аннотируего отношения создаётся несколько его версий при помощи функции $makeInvokesPermutations$.
Каждая версия отличается очередной перестановкой вызовов.
Далее происходит запуск аннотирования дизъюнкта на каждой из версий до тех пор, пока либо аннотирование закончится успехом (проверка данного факта осуществляется с помощью функции $isSuccessfulAnnotation$), либо будут просмотрены все возможные версии.
В последнем случае считается, что аннотирование не успешно.
Описанный алгоритм находится на рисунке~\ref{alg:annotateDisjWithSomeInvokes}.
Алгоритм использует функцию $annotateDisj$, приведенную на рисунке~\ref{alg:annotateDisj}.

\begin{figure}[h!]
  \begin{center}
  \begin{minipage}{1\textwidth}
\begin{algorithm}[H]
  % \KwIn{$disj$ --- дизъюнкт; $stack$ --- стек вызовов}
  % \KwOut{$disj$ --- проаннотированный дизъюнкт; $stack$ --- стек вызовов}
  % $backUpStack \gets stack$\;
  % \For {$disj~\KwFrom~makeInvokesPermutations(disj)$} {
  %   $(disj,~stack) \gets annotateDisj(disj,~stack)$\;
  %   \eIf {$isSuccessfulAnnotation(stack)$} {
  %     \Return {$(disj,~stack)$}
  %   } {
  %     $stack \gets backUpStack$
  %   }
  %   \Return {($disj,~backUpStack$)}
  % }
\end{algorithm}
  \end{minipage}
  \end{center}
  \caption{Алгоритм $annotateDisjWithSomeInvokes$ для аннотирования дизъюнкта с несколькими вызовами отношений}
  \label{alg:annotateDisjWithSomeInvokes}
\end{figure}

%%%%%%%%%%%%%%%%%%%%%%%%%%%%%%%%%%%%%%%%%%%%%%%%%%%%%%%%%%%%%%%%%%%%%%%%%%%%%%%%%%%%%%%%%%%%%%%

\subsubsection{Зависимость $fresh$-переменных только друг от друга}
\label{lab:gen}

Причина, по которой невозможно проаннотировать $append^o \ ? \ y \ ?$ (см. рисунок~\ref{lst:appendoOIOANNfail}) --- $fresh$-переменные, которые зависят только друг от друга.
В данном примере это переменные $h$, $x$ и $z$.
$h$ встречается в двух унификациях, но проаннотировать её невозможно.
В строке~\ref{line:appendoOIOANNfail4} она унифицируется с переменной $x$, а в строке~\ref{line:appendoOIOANNfail5} --- с переменной $z$.
$x$, и $z$ являются выходными ($fresh$-переменными) и их значения остаются неизвестными, так как они не присутствуют в последнем конъюнкте (строка~\ref{line:appendoOIOANNfail6}).

Таким образом, не существует возможности проаннотировать $fresh$-переменные, зависящие друг от друга, так как они никогда не станут известны и останутся свободными.
В \miniKanren{} такие переменные могут принимать все допустимые значения.
Чтобы решить проблему зависимости $fresh$-переменных только друг от друга, сымитируем подход \miniKanren{}, добавив генерацию оставшихся свободными переменных.
Под \emph{добавлением генерации} понимается добавление в дизъюнкт нового конъюнкта --- унификации целевой переменной со специальным термом-конструктором вида $C~gen~[]$ (в конкретном синтаксисе $<gen:>$).
Здесь подразуменвается, что слово $gen$ зарезервировано и не используется как название какого-либо конструктора в исходной программе.
В этом случае аннотация такой переменной будет являться аннотацией константы и равняться $1$.

\begin{figure}[h!]
  \begin{center}
  \begin{minipage}{1.1\textwidth}
\begin{algorithm}[H]
  % \KwIn{$stack$ --- стек вызовов}
  % \KwOut{$stack$ --- стек вызовов с добавленной генерацией}
  % \For {$(nameDirection,~goal)~\KwFrom~stack$} {
  %   \For {$unif~\KwFrom~goal$} {
  %     \If {$containsUndefAnnotations(unif)$} {
  %       $(left,~right) \gets unif$\;
  %       $subst \gets isSubsetOf(left,~right)$\;
  %       $termWithVarsForGeneration \gets null$\;
  %       \eIf {$isNull(subst)$} {
  %         $termWithVarsForGeneration \gets right$
  %       } {
  %         $termWithVarsForGeneration \gets left$
  %       }
  %       \For {$(var,~ann)~\KwFrom~termWithVarsForGeneration$} {
  %         \If {$ann~=~Undef$} {
  %           $unif \gets (unif~\wedge~(V~var~:=:~C~gen~[])))$
  %         }
  %       }
  %       \KwBreak
  %     }
  %   }
  % }
\end{algorithm}
  \end{minipage}
  \end{center}
  \caption{Алгоритм $addGeneration$ для добавления генерации зависимых друг от друга $fresh$-переменных}
  \label{alg:addGeneration}
\end{figure}

Псевдокод алгоритма добавления генерации находится на рисунке~\ref{alg:addGeneration}.
Он использует функцию $isSubsetOf$, приведенную на рисунке~\ref{alg:isSubsetOf}.

Первый шаг для генерации --- проаннотировать программу без генерации по алгоритму аннотирования, приведённому на рисукнке~\ref{alg:annotate}, чтобы выяснить, какие переменные проаннотировать не удалось.
Для каждого определения из полученного стека вызовов, содержащего $Undef$-аннотации: для каждой унификации, если она содержит непроаннотированные переменные ($containsUndefAnnotations$), добавить генерацию.

Если унификация содержит непроаннотированные переменные, то она содержит их в обеих частях.
Нет смысла генерировать переменные обеих частей: переменные одной части станут известны, если станут известны переменные другой.
Будем генерировать переменные той части унификации, которая является частным случаем другой.

Рассмотрим, как по двум термам определить, является ли первый из них частным случаем второго (см. псевдокод на рисунке~\ref{alg:isSubsetOf}).

\begin{figure}[h!]
  \begin{center}
  \begin{minipage}{1.1\textwidth}
\begin{algorithm}[H]
  % \KwIn{($left,~right$) --- два терма: левая и правая часть унификации}
  % \KwOut{$subst$ --- список пар переменная-терм, являющийся подстановкой подтермов второго терма вместо переменных первого терма (возвращает $null$, если подстановки не существует)}
  % $subst \gets []$\;
  % \Switch{$(left,~right)$} {
  %   \Case{$((var,~\_) \gets isVariable(left),~\_)$} {
  %     $term \gets getTermByVar(subst,~var)$\;
  %     \eIf {$isNull(term)$} {
  %       \If {$term~\neq~right$} {
  %         \Return {$null$}
  %       }
  %     } {
  %       $subst \gets insert(subst,~var,~right)$\;
  %     }
  %     \Return {$subst$}
  %   }
  %   \Other{
  %     \Return {$null$}
  %   }
  % }
\end{algorithm}
  \end{minipage}
  \end{center}
  \caption{Алгоритм $isSubsetOf$ для определения того, является ли первый входной терм частным случаем второго}
  \label{alg:isSubsetOf}
\end{figure}

Заведём пустую подстановку: список пар переменная-терм.
Если первый терм является частным случаем второго, то результатом алгоритма будет подстановка соответствующих подтермов второго терма вместо переменных первого терма.
Возможным следующие варианты.
\begin{itemize}
    \item Первый терм --- переменная $var$, второй --- терм $right$ (может быть и переменной, и конструктором).
    По переменной $var$ попробуем получить терм подстановки $term$.
    Если это удалось сделать, сравним полученный терм с $term$ --- в случае совпадения вернём текущую подстановку, иначе искомой подстановки не существует и первый подтерм не является подтермом второго.
    Если $right$ в подстановке не нашлось, добавим в неё пару $(var,~right)$.
    \item Случая двух конструкторов для нормализованной программы нет.
    \item В оставшихся случаях искомой подстановки заведомо не существует.
\end{itemize}

После добавления генерации необходимо запустить алгоритм аннотирования~\ref{alg:annotate} снова для того, чтобы проаннотировать сгенерированные переменные и распространить их аннотации на всё отношение.
Шаг ``генерация-аннотация'' нужно повторять до достижения неподвижной точки: после генерации при повторном аннотировании может появиться возможность проаннотировать вызов, который до этого был на полностью неопределённых переменных.
Для аннотирования тела этого вызова также может потребоваться генерация.
Результат аннотирования отношения $append^o \ ? \ y \ ?$ с добавлением генерации приведен на рисунке~\ref{lst:appendoOIOANN}.

\begin{figure}[h!]
  \begin{center}
  \begin{minipage}{0.49\textwidth}
  \begin{lstlisting}[language=Haskell, frame=single, numbers=left,numberstyle=\small, firstnumber=54, escapechar=|]
 $append^o$ $x^1$ $y^0$ $z^1$ =
   ($x^1$ $\equiv$ [] $\wedge$ $y^1$ $\equiv$ $z^0$) $\vee$ |\label{line:appendoOIOANN2}|
   ($fresh$ [$h$, $t$, $r$] (
       $h^1$ $\equiv$ $<gen:>$ $\wedge$ |\label{line:appendoOIOANN4}|
       $x^2$ $\equiv$ $h^1$ : $t^1$ $\wedge$ |\label{line:appendoOIOANN5}|
       $z^2$ $\equiv$ $h^1$ : $r^1$ $\wedge$ |\label{line:appendoOIOANN6}|
       $append^o$ $t^1$ $y^0$ $r^1$ |\label{line:appendoOIOANN7}|
   ))
    \end{lstlisting}
  \end{minipage}
  \end{center}
  \caption{Результат аннотирования отношения $append^o \ ? \ y \ ?$ c добавлением генерации}
  \label{lst:appendoOIOANN}
\end{figure}

Генерация позволила проаннотировать $append^o \ ? \ y \ ?$.
В этом случае аннотирование $revers^o$ с последовательностью вызовов $append^o$-$revers^o$ (рисунок~\ref{lst:reversoIOANNfail}) становится успешным.
В итоге сущесвует два способа трансляции $revers^o \ x \ ?$.
Так, для примера на рисунке~\ref{lst:reversoIOANNfail} из вызовов будут сгенерированны две функции: предикат $revers^o$ и $append^o \ ? \ y \ ?$.
Для отношения на рисунке~\ref{lst:reversoIOANN}) --- только $append^o \ ? \ ? \ z$.

Генерация переменных способна влиять на направления вычислений конъюнктов, поэтому её стоит применять только по необходимости.
Чтобы уменьшить это влияние, будем генерировать переменные только в случае не успешного аннотирования.
Для этого запустим алгоритм генерации после запуска алгоритма перебора перестановок конъюнктов.
На примере $revers^o$: из двух вариантов предпочтительной последовательностью конъюнктов обладает вариант на рисунке~\ref{lst:reversoIOANN} --- $revers^o$ с последовательностью вызовов $revers^o$-$append^o$.
Аннотирование второго дизъюнкта на рисунке~\ref{lst:reversoIOANNfail} завершится неудачей, поэтому произойдёт перестановка вызовов и получим последовательность конъюнктов другого варианта.
Это приведёт к успешному аннотированию и генерация не понадобится.

\subsection{Нормализация программ для аннотирования}
\label{sec:normalize}

Нормализация программ для аннотирования с вызовами на термах-конструкторах рассматривается в последующих разделах.
Доказательство того, что любое отношение \miniKanren{} можно преобразовать в нормальную форму приводится в подразделе~\ref{lab:normProof}.

%%%%%%%%%%%%%%%%%%%%%%%%%%%%%%%%%%%%%%%%%%%%%%%%%%%%%%%%%%%%%%%%%%%%%%%%%%%%%%%%%%%%%%%%%%%%%%%%%%%%%%%%%

\subsubsection{Нерекурсивные вызовы на конструкторах}
\label{lab:non-rec}

К моменту вызова аргумент-конструктор может быть проаннотирован частично.
В этом случае неизвестно является ли переменная, соответствующая данному аргументу, входной или выходной.
Другими словами, невозможно определить направление вызова.

Для решения проблемы нерекурсивных вызовов на конструкторах будем действовать следующим образом.
\begin{itemize}
    \item Сформируем новое отношение, принимающее на вход все переменные аргументов вызова. Его тело --- тело вызываемого отношения с подставленными в него аргументами.
    \item Вызов исходного отношения на аргументах-конструкторах заменим на вызов нового отношения на аргументах-переменных. 
\end{itemize}

Рассмотрим вызов $append^o~(a:as)~ys~z$.
Один из его аргументов --- конструктор списка.
Сформируем новое отношение $append^o1$ (см. рисунок ~\ref{lst:appendo1}), осуществив подстановку $x~\gets~(a:as)$ в тело $append^o$.

\begin{figure}[h!]
  \begin{center}
  \begin{minipage}{0.35\textwidth}
  \begin{lstlisting}[language=Haskell, frame=single, numbers=left,numberstyle=\small, firstnumber=62, escapechar=|]
  $append^o1$ $a$ $as$ $y$ $z$ =
    ($fresh$ [$h$, $t$, $r$] (
        a $\equiv$ h $\wedge$ |\label{line:appendo13}|
        as $\equiv$ t $\wedge$ |\label{line:appendo14}|
        z $\equiv$ h : r $\wedge$ |\label{line:appendo15}|
        $append^o$ t y r |\label{line:appendo16}|
    ))
    \end{lstlisting}
  \end{minipage}
  \end{center}
  \caption{Отношение $append^o1 \ a \ as \ y \ z$, полученное подстановкой $x~\gets~(a:as)$ в $append^o \ x \ y \ z$}
  \label{lst:appendo1}
\end{figure}

Заметим, что первый дизъюнкт $append^o$ отсутствует в $append^o1$.
Он стал заведомо ошибочен: унификация $x \equiv []$ обратилась в $(a:as) \equiv []$.
Во втором дизъюнкте первый конъюнкт обратился в унификацию двух конструкторов и, как следствие, был заменен на конъюнкцию двух унификаций.

Производить замену вызова на аргументах-конструкторах нужно также в теле созданного отношения, поэтому вышеописанные шаги должны выполняться до достижения неподвижной точки.
В связи с этим можем получать зацикливание при работе с рекурсивными вызовами на конструкторах.

%%%%%%%%%%%%%%%%%%%%%%%%%%%%%%%%%%%%%%%%%%%%%%%%%%%%%%%%%%%%%%%%%%%%%%%%%%%%%%%%

\subsubsection{Рекурсивные вызовы на конструкторах}
\label{lab:rec}

Рассмотрим аннотирование рекурсивных вызовов на конструкторах на примере.
Отношение $revacc^o$ связывает три списка: третий получается переворачиванием первого, а второй является аккумулятором.
$revacc^o$ приведено на рисунке~\ref{lst:revacco}.

\begin{figure}[h!]
  \begin{center}
  \begin{minipage}{0.4\textwidth}
  \begin{lstlisting}[language=Haskell, frame=single, numbers=left,numberstyle=\small, firstnumber=69, escapechar=|]
  $revacc^o$ $xs$ $acc$ $sx$ =
    ($xs$ $\equiv$ [] $\wedge$ $sx$ $\equiv$ $acc$) $\vee$ |\label{line:revacco2}|
    ($fresh$ [$h$, $t$] (
        $xs$ $\equiv$ $h$ : $t$ $\wedge$ |\label{line:revacco4}|
        $revacc^o$ $t$ $(h~\%~acc)$ $sx$ |\label{line:revacco5}|
    ))
    \end{lstlisting}
  \end{minipage}
  \end{center}
  \caption{Отношение $revacc^o xs \ acc \ sx$}
  \label{lst:revacco}
\end{figure}

Данное отношение содержит рекурсивный вызов на конструкторе в строке~\ref{line:revacco5}.
Попробуем заменить его на новое отношение в соответствии с шагами, описанными в разделе~\ref{lab:non-rec}.
Подстановка $aсс~\gets~(h:acc)$ в $revacc^o$ представлена на рисунке~\ref{lst:revacco1}.

\begin{figure}[h!]
  \begin{center}
  \begin{minipage}{0.45\textwidth}
  \begin{lstlisting}[language=Haskell, frame=single, numbers=left,numberstyle=\small, firstnumber=75, escapechar=|]
  $revacc^o1$ $xs$ $h$ $acc$ $sx$ =
    ($xs$ $\equiv$ [] $\wedge$ $sx$ $\equiv$ $(h~\%~acc)$) $\vee$ |\label{line:revacco12}|
    ($fresh$ [$h'$, $t$] (
        $xs$ $\equiv$ $h'$ : $t$ $\wedge$        |\label{line:revacco14}|
        $revacc^o$ $t$ $(h'~\%~(h~\%~acc))$ $sx$ |\label{line:revacco15}|
    ))
    \end{lstlisting}
  \end{minipage}
  \end{center}
  \caption{Отношение $revacc^o1 \ xs \ h \ acc \ sx$, полученное подстановкой $acc~\gets~(h:acc)$ в $revacc^o \ x \ y$}
  \label{lst:revacco1}
\end{figure}

Видно, что в строке~\ref{line:revacco15} такая подстановка привела только к большей вложенности конструкторов.
Это означает, что неподвижная точка не будет достигнута никогда.

Альтернативное решение проблемы рекурсивных вызовов на конструкторах состоит из двух шагов:
\begin{itemize}
    \item в дизъюнкт, содержащий рекурсивный вызов на конструкторе, добавим конъюнкт --- унификацию этого конструктора с новой переменной;
    \item в вызове аргумент-конструктор заменим на новую переменную.
\end{itemize}

На рисунке~\ref{lst:revacco2IOOANN} приведён пример применения данного решения и результат аннотирования $revacc^o \ xs \ ? \ ?$.

\begin{figure}[h!]
  \begin{center}
  \begin{minipage}{0.35\textwidth}
  \begin{lstlisting}[language=Haskell, frame=single, numbers=left,numberstyle=\small, firstnumber=81, escapechar=|]
  $revacc^o2$ $xs0$ $acc1$ $sx1$ =
    ($xs0$ $\equiv$ [] $\wedge$
    $sx1$ $\equiv$ $<gen:>$ $\wedge$|\label{line:revacco2IOOANN2}|
    $sx1$ $\equiv$ $acc2$) $\vee$ |\label{line:revacco2IOOANN3}|
    ($fresh$ [$h$, $t$, $hacc$] (
        $xs0$ $\equiv$ $h1$ : $t1$ $\wedge$ |\label{line:revacco2IOOANN5}|
        $hacc2$ $\equiv$ $h1$ : $acc3$ $\wedge$ |\label{line:revacco2IOOANN6}|
        $revacc^o$ $t1$ $hacc2$ $sx2$ |\label{line:revacco2IOOANN7}|
    ))
    \end{lstlisting}
  \end{minipage}
  \end{center}
  \caption{Результат аннотирования отношения $revacc^o \ xs \ ? \ ?$, полученного унификацией аргумента-конструктора с первым входным аргументом из $append^o \ x \ y \ z$}
  \label{lst:revacco2IOOANN}
\end{figure}

Унификация позволит определять к моменту вызова, является ли аргумент, бывший конструктором, входным или выходным.
Недостатком данного подхода является возможность потерять информацию об аннотациях переменных конструктора для аннотирования тела вызова.
Потеря этой информации может привести к завершению аннотирования неудачей.
Применение алгоритма аннотирования с добавлением генерации (см. рисунок~\ref{alg:addGeneration}) способно исправить ситуацию.

Данный подход может работать и для нерекурсивных вызовов на конструкторах, но он с большей вероятностью потребует генерацию, применения которой хочется избежать.

%%%%%%%%%%%%%%%%%%%%%%%%%%%%%%%%%%%%%%%%%%%%%%%%%%%%%%%%%%%%%%%%%%%%%%%%%%%%%%%%

\subsubsection{Вызовы на повторяющихся переменных}

Вызовы отношений могут происходить на одних и тех же переменных.
В этом случае аннотации соответствующих аргументов обязаны совпадать.
Это делает невалидными некоторые направления, которые, судя по количеству аргументов, должны существовать.

Рассмотрим пример: $append^o~x~x~z$.
У $append^o$ три аргумента и, значит, восемь направлений.
Однако, первые два аргумента данного вызова совпадают и направлений остаётся четыре, так как направления с разной аннотацией первых двух аргументов становятся невалидными.

\begin{figure}[h!]
  \begin{center}
  \begin{minipage}{0.35\textwidth}
  \begin{lstlisting}[language=Haskell, frame=single, numbers=left,numberstyle=\small, firstnumber=90, escapechar=|]
  $append^o2$ $x$ $z$ =
    ($x$ $\equiv$ [] $\wedge$ $x$ $\equiv$ $z$) $\vee$ |\label{line:appendo22}|
    ($fresh$ [$h$, $t$, $r$] (
        $x$ $\equiv$ $h$ : $t$ $\wedge$ |\label{line:appendo24}|
        $z$ $\equiv$ $h$ : $r$ $\wedge$ |\label{line:appendo25}|
        $append^o$ $t$ $x$ $r$          |\label{line:appendo26}|
    ))
    \end{lstlisting}
  \end{minipage}
  \end{center}
  \caption{$append^o2 \ x \ z$, полученное подстановкой $y~\gets~x$ в $append^o \ x \ y \ z$}
  \label{lst:appendo2}
\end{figure}

Справиться с проблемой вызовов на одних и тех же переменных помогает тот же подход, что и для нерекурсивных вызовах на конструкторах: создадим новое отношение, подставив аргументы в тело исходного.
В созданном отношении (см. рисунок~\ref{lst:appendo2}) заведомо не может существовать невалидных направлений.

%%%%%%%%%%%%%%%%%%%%%%%%%%%%%%%%%%%%%%%%%%%%%%%%%%%%%%%%%%%%%%%%%%%%%%%%%%%%%%%%

\subsubsection{Доказательство утверждения о нормализации}
\label{lab:normProof}

\begin{theorem}
    Любое отношение на \miniKanren{}, записанном в синтаксисе на рисунке~\ref{lst:miniKanren} можно привести к нормальной форме, представленной на рисунке~\ref{fig:normMiniKanren}.
\end{theorem}

\begin{proof}
    Необходимо показать, что любое отношение ненормализофанной формы~\ref{lst:miniKanren} можно привести к отношению нормализованной формы~\ref{fig:normMiniKanren}.
    Для доказательства воспользуемся идеей структурной индукции: разберём способы приведения каждого из конструктов к нормальной форме.
    
    \textit{Унификация}.
    Запрещены унификации двух конструкторов.
    Пусть существует унификация $C~name~[t_0 \dots t_k]~:=:~C~name~[s_0 \dots s_k]$, где $t_i$ и $s_i$ --- термы.
    Ее можно заменить на конъюнкцию унификаций вида $t_0~:=:~s_0~\wedge~\dots~\wedge~t_k~:=:~s_k$.
    Если среди пар унификаций вида $t_i~:=:~s_i$ окажется унификация двух конструкторов, ее так же необходимо заменить на конъюнкцию унификаций.
    
    \textit{Вызов отношения}.
    Запрещены вызовы отношений на аргументах-конструкторах.
    Случаи нерекурсивного и рекурсивного вызовов на аргументах-конструкторах рассмотрены в разделах~\ref{lab:non-rec} и~\ref{lab:rec} соответственно.
    В данных разделах рассказывается, как из вызовов на аргументах-конструкторах получить вызовы на переменных.
    
    \textit{Объявление новых переменных}.
    $fresh$-цель разрешена только на самом верхнем уровне.
    Если уникально переименовать все $fresh$-переменные отношения, то $fresh$-цель можно оставить только на самом верхнем уровне, избежав перекрытия имён.
    
    \textit{Конъюнкция и дизъюнкция}.
    Нормализованное отношение должно быть в ДНФ.
    Приведение булевого выражения в дизъюнктивную нормальную форму --- тривиальная задача.
\end{proof}

\subsection{Корректность алгоритма аннотирования}

Алгоритм аннотирования, представленный в работе, способен аннотировать только нормализованные программы на \miniKanren{}.
Однако, любую программу на \miniKanren{} можно привести в нормальную форму описанными выше методами~\ref{lab:normProof}.
Доказав корректность аннотирования нормализованных программ, мы докажем и корректность ненормализованных.

Алгоритм представляет собой адаптацию алгоритма анализа времени связывания для \miniKanren{}.
Для доказательства корректности необходимо показать его терминируемость и согласованность, что и сделано в последующих разделах.

%%%%%%%%%%%%%%%%%%%%%%%%%%%%%%%%%%%%%%%%%%%%%%%%%%%%%%%%%%%%%%%%%%%%%%%%%%%%%%%%

\subsubsection{Терминируемость}

\begin{theorem}
Предложенный в разделе~\ref{lab:coreAnn} алгоритм аннотирования нормализованных программ терминируется
\end{theorem}

\begin{proof}
Алгоритм терминируется, так как повторное аннотирование отношений не производится.
Это, в свою очередь, означает, что будет достигнута неподвижная точка и алгоритм завершит свою работу.
Проверим, что каждая из частей алгоритма не производит повторного аннотирования.

\textit{Унификация}.
Аннотирование происходит только в случае существования $Undef$-аннотаций.

\textit{Вызов отношения}.
Имеющиеся в стеке вызовов отношения не аннотируются снова, а в каждом отношении используется конечное количество уникальных переменных.

\textit{Конъюнкция / Дизъюнкт}.
Так как конъюнкт --- либо унификация, либо вызов отношения, то ни один конъюнкт не будет проаннотирован повторно.
Конъюнктов в дизъюнкте конечное количество, значит, неподвижная точка будет достигнута.

\textit{Объявление новых переменных}.
Только на верхнем уровне; аннотирование такой цели равнозначно аннотированию цели внутри $fresh$-цели.

\textit{Дизъюнкция}.
Аннотирование происходит независимо для каждого дизъюнкта.

Из представленых выше рассуждений можно сделать вывод, что каждому отношению можно сопоставить конечное количество уникальных аннотаций и терминируемость этой части алгоритма доказана.
\end{proof}

\begin{theorem} Предложенная в разделе~\ref{lab:disjPerm} модификация алгоритма аннотирования нормализованных программ для разрешения проблемы нескольких вызовов в одном дизъюнкте терминируется.
\end{theorem}

\begin{proof}
Существование нескольких вызовов в одном дизъюнкте приводит к необходимости применять алгоритм аннотирования ко всем возможным версиям дизъюнкта, каждая из которых отличается очередной перестановкой вызовов.
Терминируемость в этом случае следует из двух фактов:
\begin{itemize}
    \item количество перестановок вызовов конечно и, значит, конечно количество версий дизъюнкта;
    \item алгоритм аннотирования нормализованных программ терминируется.
\end{itemize}
\end{proof}

\begin{theorem} 
Предложенная в разделе~\ref{lab:gen} модификация алгоритма аннотирования нормализованных программ для разрешения проблемы зависимости $fresh$-переменных только друг от друга терминируется.
\end{theorem}

\begin{proof}
Добавление генерации не влияет на терминируемость несмотря на итеративность процесса.
Терминируемость следует из двух фактов.
\begin{itemize}
    \item Количество переменных, оставшихся в случае неуспешного аннотирования помеченными $Undef$, конечно для всего стека вызовов, так как конечно количество переменных в любом отношении, а, значит, и в стеке вызовов; на каждой итерации генерации происходит добавление хотя бы одной генерации хотя бы в одно определение со стека вызовов.
    \item Aлгоритм аннотирования нормализованных программ, запускаемый повторно после генерации, терминируется.
\end{itemize}
\end{proof}

%%%%%%%%%%%%%%%%%%%%%%%%%%%%%%%%%%%%%%%%%%%%%%%%%%%%%%%%%%%%%%%%%%%%%%%%%%%%%%%%

\subsubsection{Согласованность}

В анализе времени связывания под согласованностью понимается \emph{зависимость статических данных только от статических}: статические данные не могут определяться динамическими.

\begin{theorem} 
Предложенный в разделе~\ref{lab:coreAnn} алгоритм аннотирования нормализованных программ с модификациями для разрешения проблемы нескольких вызовов в одном дизъюнкте~\ref{lab:disjPerm} и проблемы зависимости $fresh$-переменных только друг от друга~\ref{lab:gen} является согласованным.
\end{theorem}

\begin{proof}
Вычисление дизъюнктов в \miniKanren{} происходит независимо, значит, и аннотировать их можно независимо.
Как следствие, для аннотации тела отношения необходимо проаннотировать входящие в него дизъюнкты.
Показав корректность аннотирования одного дизъюнкта, покажем корректность аннотирования всего тела.

Каждый дизъюнкт --- это конъюнкция вызовов и унификаций.
Вычисление конъюнктов в \miniKanren{} происходит одновременно: значение полученное в одном конъюнкте, мгновенно становится известно в другом.
Для аннотирования это означает, что, если стала известна аннотация целевой переменной в одном конъюнкте, она мгновенно становится известна во всех конъюнктах, в которые эта переменная входит.
Именно так и происходит в алгоритме: дизъюнкты аннотируются независимо, а аннотация переменной, ставшая известной в одном конъюнкте, распространяется на все вхождения этой переменной в другие конъюнкты.

Введём понятие зависимости одной переменной от другой в рамках предложенного алгоритма.
Понятие отношения подразумевает ``равноправие'' переменных, участвующих в нём.
Однако, при выборе конкретного направления вычисления значения переменных множества $X$ становятся известны раньше значений переменных множества $Y$.
В этом случае будем говорить, что переменные Y \emph{зависят} от переменных X.
Проиллюстрируем понятие зависимости переменных друг от друга на примерах.

Пример: зависимость для унификаций.
Пусть есть два конъюнкта: $x \equiv y$ и $y \equiv 7$.
Во втором конъюнкте $7$ --- константа, поэтому мы можем проунифицировать $y$ и сказать, что $y = 7$.
В этот же момент мы узнаем в первом конъюнкте, что $y$ стала известна, и можем превратить унификацию в равенство $x = y$, обозначающее зависимость $x$ от $y$.

Пример: зависимость для вызовов отношений.
Пусть есть вызов отношения $append^o~x~y~z$, где мы уже знаем из других конъюнктов значение $z$.
В этом случае алгоритм посчитает, что этот вызов $append^o$ происходит направлении, при котором переменные $x$ и $y$ являются выходными.
В этом случае можно говорить о зависимости $x$ и $y$ от $z$: $(x,~y)~=~append^o~z$ (в случае недетерминированной семантики $apppend^o$ корректнее говорить о $[(x,~y)]~=~append^o~z$).

Введём инвариант, отражающий идею согласованности.
Доказав его выполнение на любом шаге алгоритма, докажем его корректность.

\emph{Инвариант:
\begin{itemize}
    \item либо переменная не проаннотирована (имеет аннотацию $Undef$);
    \item либо переменная проаннотирована числом; тогда существует хотя бы один конъюнкт, в котором все переменные, от которых она зависит, проаннотированы строго меньшими числами.
\end{itemize}
}

Рассмотрим алгоритм, чтобы убедиться в выполнении инварианта.
В начальный момент времени аннотацию $0$ имеют только входные переменные.
Остальные переменные проаннотированы $Undef$.

Конъюнкты отсортированы: вызовы следуют за унификациями.
К каждой унификации применяется алгоритм аннотирования унификаций, в точности выполняющий инвариант.
$Undef$-аннотация целевой переменной заменяется всегда на строго большее значение, чем значение аннотации любой переменной, от которой целевая переменная зависит.
После аннотирования каждого конъюнкта информация об аннотациях его переменных распространяется на все оставшиеся конъюнкты.
Следующий для аннотирования конъюнкт обладает релевантными аннотациями.

К моменту начала аннотирования вызовов отношения можем быть уверены, что в текущем вызове известны все аннотации переменных, которые можно было получить из унификаций.
Все другие --- только из последующих вызовов отношений.
Тем самым, мы знаем направление первого вызова в текущей перестановке вызовов конкретного дизъюнкта.
При наличии нескольких вызовов их порядок влияет на аннотирование.
Наилучший порядок, позволяющий получить проаннотированное отношение, можно найти только опытным путём --- перебрав все перестановки вызовов.
Поэтому, без ограничения общности можно считать, что первый вызов выбран верно.
Если аннотирование при этом закончится неудачей, запустится аннотирование того же дизъюнкта с другим порядком вызовов.
Важно заметить, что, в случае неуспеха аннотирования стек вызовов будет содержать переменные с $Undef$ аннотациями --- это является частью инварианта.

Вернёмся к аннотированию вызова.
Алгоритм аннотации аргументов вызова в точности соблюдает инвариант.
Каждое вызываемое в конкретном направлении отношение добавляется в стек, если оно там отсутствовало, и инициализируется так, что его входные переменные имеют аннотацию $0$.
Это позволяет рассматривать аннотацию тела вызываемого отношения независимо от причин аннотирования: является ли аннотируемая цель целью программы или телом вызываемого отношения.

Для случая необходимости добавления генерации остаётся заметить, что данная модификация лишь изменяют структуру программы, но не производит аннотирование.
За счёт чего можно утверждать, что инвариант сохраняется.
\end{proof}
