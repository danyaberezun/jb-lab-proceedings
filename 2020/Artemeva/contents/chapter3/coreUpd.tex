\subsection{Проблемы аннотирования нормализованной программы}
\label{lab:problems}

%%%%%%%%%%%%%%%%%%%%%%%%%%%%%%%%%%%%%%%%%%%%%%%%%%%%%%%%%%%%%%%%%%%%%%%%%%%%%%%%%%%%%%%%%%%%%%%

\subsubsection{Несколько вызовов в одном дизъюнкте}
\label{lab:disjPerm}

Существуют отношения, точная аннотация которых невозможна без полного перебора возникающих вариантов или пользовательских аннотаций.
Один из видов таких отношений --- отношения, содержащие несколько вызовов в одном дизъюнкте.
Пример такого отношения приведен на рисунке~\ref{lst:reloDEF}.

\begin{figure}[h!]
  \begin{center}
  \begin{minipage}{0.18\textwidth}
  \begin{lstlisting}[language=Haskell, frame=single, numbers=left,numberstyle=\small, firstnumber=29, escapechar=|]
  $rel^o$ $x$ $y$ $z$ =
    $f^o$ $x$ $y$ $\vee$
    $h^o$ $z$ $y$ $\vee$
    $g^o$ $x$ $z$
    \end{lstlisting}
  \end{minipage}
  \end{center}
  \caption{Пример программы на \miniKanren{} с несколькими вызовами в одном дизъюнкте}
  \label{lst:reloDEF}
\end{figure}

Пусть $y$ --- входная переменная.
В этом случае порядок вычисления вызовов $f^o$ и $h^o$ не зависит друг от друга, но зависит от направления вычисления $g^o$.
Оно не может вычисляться до вычисления $f^o$ и $h^o$ (неизвестны входные переменные), но может вычисляться между ними ($g^o \ x \ ?$ или $g^0 \ ? \ z$) или после (выполнять роль предиката: $g^o \ x \ y$).

Пример такого поведения можно увидеть при аннотировании $revers^o$.
Аннотирование того же определения $revers^o$, что было рассмотрено на рисунке~\ref{lst:reversoOIANN}, в направлении $revers^o \ x \ ?$, приведено на рисунке~\ref{lst:reversoIOANNfail}).

\begin{figure}[h!]
  \begin{center}
  \begin{minipage}{0.35\textwidth}
  \begin{lstlisting}[language=Haskell, frame=single, numbers=left,numberstyle=\small, firstnumber=33, escapechar=|]
  $revers^o$ $x^0$ $y^1$ =
    ($x^0$ $\equiv$ [] $\wedge$ $y^1$ $\equiv$ []) $\vee$ |\label{line:reversoIOANNfail2}|
    ($fresh$ [$h$, $t$, $r$] (
        $x^0$ $\equiv$ $h^1$ : $t^1$ $\wedge$ |\label{line:reversoIOANNfail4}|
        $append^o$ $r^{2}$ $[h^1]$ $y^{2}$ |\label{line:reversoIOANNfail5}|
        $revers^o$ $t^1$ $r^{2}$ $\wedge$ |\label{line:reversoIOANNfail6}|
    ))
    \end{lstlisting}
  \end{minipage}
  \end{center}
  \caption{Результат аннотирования отношения $revers^o \ x \ ?$ с порядком вызовов $append^o$-$revers^o$}
  \label{lst:reversoIOANNfail}
\end{figure}

Во втором дизъюнкте в начале определяются аннотации переменных $h$ и $t$ в строке~\ref{line:reversoIOANNfail4} --- затем они распространяются на последующие вызовы.
Входным является только второй аргумент, поэтому необходимо проаннотировать тело $append^o \ ? \ y \ ?$ (см. рисунок~\ref{lst:appendoOIOANNfail}).
Аннотация $append^o \ ? \ y \ ?$ завершится неудачией: строках~\ref{line:appendoOIOANNfail4} и~\ref{line:appendoOIOANNfail5} остались переменные, проаннотированные $Undef$.
Причина в том, что при таком направлении не существует возможности узнать значения (а, значит, и время связывания) переменных $h$, $x$ и $z$ (данная проблема обсуждается в следующем подразделе).
Аннотирование $revers^o \ x \ ?$ завершится успешно, однако, при трансляции невозможно получить работающую программу, так как неуспешно завершилось аннотирование $append^o \ ? \ y \ ?$.

\begin{figure}[h!]
  \begin{center}
  \begin{minipage}{0.40\textwidth}
  \begin{lstlisting}[language=Haskell, frame=single, numbers=left,numberstyle=\small, firstnumber=40, escapechar=|]
  $append^o$ $x^1$ $y^0$ $z^1$ =
    ($x^1$ $\equiv$ [] $\wedge$ $y^1$ $\equiv$ $z^0$) $\vee$ |\label{line:appendoOIOANNfail2}|
    ($fresh$ [$h$, $t$, $r$] (
        $x^{Undef}$ $\equiv$ $h^{Undef}$ : $t^1$ $\wedge$ |\label{line:appendoOIOANNfail4}|
        $z^{Undef}$ $\equiv$ $h^{Undef}$ : $r^1$ $\wedge$ |\label{line:appendoOIOANNfail5}|
        $append^o$ $t^1$ $y^0$ $r^1$ |\label{line:appendoOIOANNfail6}|
    ))
    \end{lstlisting}
  \end{minipage}
  \end{center}
  \caption{Результат аннотирования отношения $append^o \ ? \ y \ ?$}
  \label{lst:appendoOIOANNfail}
\end{figure}

На рисунке~\ref{lst:reversoIOANN} приведён результат аннотирования в случае другого порядка вызовов: теперь $revers^o$ происходит перед $append^o$.

\begin{figure}[h!]
  \begin{center}
  \begin{minipage}{0.4\textwidth}
  \begin{lstlisting}[language=Haskell, frame=single, numbers=left,numberstyle=\small, firstnumber=47, escapechar=|]
  $revers^o$ $x^0$ $y^1$ =
    ($x^0$ $\equiv$ [] $\wedge$ $y^1$ $\equiv$ []) $\vee$ |\label{line:reversoIOANN2}|
    ($fresh$ [$h$, $t$, $r$] (
        $x^0$ $\equiv$ $h^1$ : $t^1$ $\wedge$ |\label{line:reversoIOANN4}|
        $revers^o$ $t^4$ $r^3$ $\wedge$ |\label{line:reversoIOANN5}|
        $append^o$ $r^3$ $[h^2]$ $y^0$ |\label{line:reversoIOANN6}|
    ))
    \end{lstlisting}
  \end{minipage}
  \end{center}
  \caption{Результат аннотирования отношения $revers^o \ x \ ?$ с порядком вызовов $revers^o$-$append^o$}
  \label{lst:reversoIOANN}
\end{figure}

Строка~\ref{line:reversoIOANN5} содержит рекурсивный вызов того же направления, что и исходное отношение --- так становится известна переменная $r$.
Аннотирование будет успешно завершено аннотированием вызова $append^o \ ? \ ? \ z$ (см. рисунок~\ref{lst:appendoOOIANN}) в строке~\ref{line:reversoIOANN6}.

Для решения проблемы нескольких вызовов в одном дизъюнкте предложено решение с перестановками конъюнктов.
Для каждого дизъюнкта аннотируего отношения создаётся несколько его версий при помощи функции $makeInvokesPermutations$.
Каждая версия отличается очередной перестановкой вызовов.
Далее происходит запуск аннотирования дизъюнкта на каждой из версий до тех пор, пока либо аннотирование закончится успехом (проверка данного факта осуществляется с помощью функции $isSuccessfulAnnotation$), либо будут просмотрены все возможные версии.
В последнем случае считается, что аннотирование не успешно.
Описанный алгоритм находится на рисунке~\ref{alg:annotateDisjWithSomeInvokes}.
Алгоритм использует функцию $annotateDisj$, приведенную на рисунке~\ref{alg:annotateDisj}.

\begin{figure}[h!]
  \begin{center}
  \begin{minipage}{1\textwidth}
\begin{algorithm}[H]
  % \KwIn{$disj$ --- дизъюнкт; $stack$ --- стек вызовов}
  % \KwOut{$disj$ --- проаннотированный дизъюнкт; $stack$ --- стек вызовов}
  % $backUpStack \gets stack$\;
  % \For {$disj~\KwFrom~makeInvokesPermutations(disj)$} {
  %   $(disj,~stack) \gets annotateDisj(disj,~stack)$\;
  %   \eIf {$isSuccessfulAnnotation(stack)$} {
  %     \Return {$(disj,~stack)$}
  %   } {
  %     $stack \gets backUpStack$
  %   }
  %   \Return {($disj,~backUpStack$)}
  % }
\end{algorithm}
  \end{minipage}
  \end{center}
  \caption{Алгоритм $annotateDisjWithSomeInvokes$ для аннотирования дизъюнкта с несколькими вызовами отношений}
  \label{alg:annotateDisjWithSomeInvokes}
\end{figure}

%%%%%%%%%%%%%%%%%%%%%%%%%%%%%%%%%%%%%%%%%%%%%%%%%%%%%%%%%%%%%%%%%%%%%%%%%%%%%%%%%%%%%%%%%%%%%%%

\subsubsection{Зависимость $fresh$-переменных только друг от друга}
\label{lab:gen}

Причина, по которой невозможно проаннотировать $append^o \ ? \ y \ ?$ (см. рисунок~\ref{lst:appendoOIOANNfail}) --- $fresh$-переменные, которые зависят только друг от друга.
В данном примере это переменные $h$, $x$ и $z$.
$h$ встречается в двух унификациях, но проаннотировать её невозможно.
В строке~\ref{line:appendoOIOANNfail4} она унифицируется с переменной $x$, а в строке~\ref{line:appendoOIOANNfail5} --- с переменной $z$.
$x$, и $z$ являются выходными ($fresh$-переменными) и их значения остаются неизвестными, так как они не присутствуют в последнем конъюнкте (строка~\ref{line:appendoOIOANNfail6}).

Таким образом, не существует возможности проаннотировать $fresh$-переменные, зависящие друг от друга, так как они никогда не станут известны и останутся свободными.
В \miniKanren{} такие переменные могут принимать все допустимые значения.
Чтобы решить проблему зависимости $fresh$-переменных только друг от друга, сымитируем подход \miniKanren{}, добавив генерацию оставшихся свободными переменных.
Под \emph{добавлением генерации} понимается добавление в дизъюнкт нового конъюнкта --- унификации целевой переменной со специальным термом-конструктором вида $C~gen~[]$ (в конкретном синтаксисе $<gen:>$).
Здесь подразуменвается, что слово $gen$ зарезервировано и не используется как название какого-либо конструктора в исходной программе.
В этом случае аннотация такой переменной будет являться аннотацией константы и равняться $1$.

\begin{figure}[h!]
  \begin{center}
  \begin{minipage}{1.1\textwidth}
\begin{algorithm}[H]
  % \KwIn{$stack$ --- стек вызовов}
  % \KwOut{$stack$ --- стек вызовов с добавленной генерацией}
  % \For {$(nameDirection,~goal)~\KwFrom~stack$} {
  %   \For {$unif~\KwFrom~goal$} {
  %     \If {$containsUndefAnnotations(unif)$} {
  %       $(left,~right) \gets unif$\;
  %       $subst \gets isSubsetOf(left,~right)$\;
  %       $termWithVarsForGeneration \gets null$\;
  %       \eIf {$isNull(subst)$} {
  %         $termWithVarsForGeneration \gets right$
  %       } {
  %         $termWithVarsForGeneration \gets left$
  %       }
  %       \For {$(var,~ann)~\KwFrom~termWithVarsForGeneration$} {
  %         \If {$ann~=~Undef$} {
  %           $unif \gets (unif~\wedge~(V~var~:=:~C~gen~[])))$
  %         }
  %       }
  %       \KwBreak
  %     }
  %   }
  % }
\end{algorithm}
  \end{minipage}
  \end{center}
  \caption{Алгоритм $addGeneration$ для добавления генерации зависимых друг от друга $fresh$-переменных}
  \label{alg:addGeneration}
\end{figure}

Псевдокод алгоритма добавления генерации находится на рисунке~\ref{alg:addGeneration}.
Он использует функцию $isSubsetOf$, приведенную на рисунке~\ref{alg:isSubsetOf}.

Первый шаг для генерации --- проаннотировать программу без генерации по алгоритму аннотирования, приведённому на рисукнке~\ref{alg:annotate}, чтобы выяснить, какие переменные проаннотировать не удалось.
Для каждого определения из полученного стека вызовов, содержащего $Undef$-аннотации: для каждой унификации, если она содержит непроаннотированные переменные ($containsUndefAnnotations$), добавить генерацию.

Если унификация содержит непроаннотированные переменные, то она содержит их в обеих частях.
Нет смысла генерировать переменные обеих частей: переменные одной части станут известны, если станут известны переменные другой.
Будем генерировать переменные той части унификации, которая является частным случаем другой.

Рассмотрим, как по двум термам определить, является ли первый из них частным случаем второго (см. псевдокод на рисунке~\ref{alg:isSubsetOf}).

\begin{figure}[h!]
  \begin{center}
  \begin{minipage}{1.1\textwidth}
\begin{algorithm}[H]
  % \KwIn{($left,~right$) --- два терма: левая и правая часть унификации}
  % \KwOut{$subst$ --- список пар переменная-терм, являющийся подстановкой подтермов второго терма вместо переменных первого терма (возвращает $null$, если подстановки не существует)}
  % $subst \gets []$\;
  % \Switch{$(left,~right)$} {
  %   \Case{$((var,~\_) \gets isVariable(left),~\_)$} {
  %     $term \gets getTermByVar(subst,~var)$\;
  %     \eIf {$isNull(term)$} {
  %       \If {$term~\neq~right$} {
  %         \Return {$null$}
  %       }
  %     } {
  %       $subst \gets insert(subst,~var,~right)$\;
  %     }
  %     \Return {$subst$}
  %   }
  %   \Other{
  %     \Return {$null$}
  %   }
  % }
\end{algorithm}
  \end{minipage}
  \end{center}
  \caption{Алгоритм $isSubsetOf$ для определения того, является ли первый входной терм частным случаем второго}
  \label{alg:isSubsetOf}
\end{figure}

Заведём пустую подстановку: список пар переменная-терм.
Если первый терм является частным случаем второго, то результатом алгоритма будет подстановка соответствующих подтермов второго терма вместо переменных первого терма.
Возможным следующие варианты.
\begin{itemize}
    \item Первый терм --- переменная $var$, второй --- терм $right$ (может быть и переменной, и конструктором).
    По переменной $var$ попробуем получить терм подстановки $term$.
    Если это удалось сделать, сравним полученный терм с $term$ --- в случае совпадения вернём текущую подстановку, иначе искомой подстановки не существует и первый подтерм не является подтермом второго.
    Если $right$ в подстановке не нашлось, добавим в неё пару $(var,~right)$.
    \item Случая двух конструкторов для нормализованной программы нет.
    \item В оставшихся случаях искомой подстановки заведомо не существует.
\end{itemize}

После добавления генерации необходимо запустить алгоритм аннотирования~\ref{alg:annotate} снова для того, чтобы проаннотировать сгенерированные переменные и распространить их аннотации на всё отношение.
Шаг ``генерация-аннотация'' нужно повторять до достижения неподвижной точки: после генерации при повторном аннотировании может появиться возможность проаннотировать вызов, который до этого был на полностью неопределённых переменных.
Для аннотирования тела этого вызова также может потребоваться генерация.
Результат аннотирования отношения $append^o \ ? \ y \ ?$ с добавлением генерации приведен на рисунке~\ref{lst:appendoOIOANN}.

\begin{figure}[h!]
  \begin{center}
  \begin{minipage}{0.4\textwidth}
  \begin{lstlisting}[language=Haskell, frame=single, numbers=left,numberstyle=\small, firstnumber=54, escapechar=|]
  $append^o$ $x^1$ $y^0$ $z^1$ =
    ($x^1$ $\equiv$ [] $\wedge$ $y^1$ $\equiv$ $z^0$) $\vee$ |\label{line:appendoOIOANN2}|
    ($fresh$ [$h$, $t$, $r$] (
        $h^1$ $\equiv$ $<gen:>$ $\wedge$ |\label{line:appendoOIOANN4}|
        $x^2$ $\equiv$ $h^1$ : $t^1$ $\wedge$ |\label{line:appendoOIOANN5}|
        $z^2$ $\equiv$ $h^1$ : $r^1$ $\wedge$ |\label{line:appendoOIOANN6}|
        $append^o$ $t^1$ $y^0$ $r^1$ |\label{line:appendoOIOANN7}|
    ))
    \end{lstlisting}
  \end{minipage}
  \end{center}
  \caption{Результат аннотирования отношения $append^o \ ? \ y \ ?$ c добавлением генерации}
  \label{lst:appendoOIOANN}
\end{figure}

Генерация позволила проаннотировать $append^o \ ? \ y \ ?$.
В этом случае аннотирование $revers^o$ с последовательностью вызовов $append^o$-$revers^o$ (рисунок~\ref{lst:reversoIOANNfail}) становится успешным.
В итоге сущесвует два способа трансляции $revers^o \ x \ ?$.
Так, для примера на рисунке~\ref{lst:reversoIOANNfail} из вызовов будут сгенерированны две функции: предикат $revers^o$ и $append^o \ ? \ y \ ?$.
Для отношения на рисунке~\ref{lst:reversoIOANN}) --- только $append^o \ ? \ ? \ z$.

Генерация переменных способна влиять на направления вычислений конъюнктов, поэтому её стоит применять только по необходимости.
Чтобы уменьшить это влияние, будем генерировать переменные только в случае не успешного аннотирования.
Для этого запустим алгоритм генерации после запуска алгоритма перебора перестановок конъюнктов.
На примере $revers^o$: из двух вариантов предпочтительной последовательностью конъюнктов обладает вариант на рисунке~\ref{lst:reversoIOANN} --- $revers^o$ с последовательностью вызовов $revers^o$-$append^o$.
Аннотирование второго дизъюнкта на рисунке~\ref{lst:reversoIOANNfail} завершится неудачей, поэтому произойдёт перестановка вызовов и получим последовательность конъюнктов другого варианта.
Это приведёт к успешному аннотированию и генерация не понадобится.
