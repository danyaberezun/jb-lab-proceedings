\subsection{Классификация программ для трансляции}
\label{lab:classif}

В главе~\ref{annotator} об анализе времени связывания было введено понятие нормальной формы программы на \miniKanren{} (см. раздел~\ref{lab:normform}).
Его введение позволило разбить всё множество программ на \miniKanren{} на классы по наличию тех или иных конструкций.
Каждый класс, обладая своим, особенным свойством или конструкцией, требовал особого подхода в процессе трансляции.
При этом одна и та же программа может содержать разные конструкции, и, как следствие, находиться одновременно в нескольких классах.

Классификация:
\begin{itemize}
    \item тело отношения находится в ДНФ;
    \item $fresh$-переменные только на верхнем уровне;
    \item присутствуют унификации двух конструкторов;
    \item несколько вызовов на аргументах-переменных;
    \item унификация $fresh$-переменных только друг с другом;
    \item нерекурсивные вызовы на аргументах-конструкторах;
    \item рекурсивные вызовы на аргументах-конструкторах;
    \item вызовы на повторяющихся переменных.
\end{itemize}
