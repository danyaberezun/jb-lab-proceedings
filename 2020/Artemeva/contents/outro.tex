\section*{Заключение}

В рамках данной работы получены следующие результаты.

\begin{itemize}
\item Разработан алгоритм анализа времени связывания для реляционного языка  \miniKanren{}, позволяющий транслятору определять направления и порядок вычисления конъюнктов за счет использования информации о времени связывания переменных. Доказана его терминируемость и согласованность. 
\item Разработан алгоритм трансляции программ на реляционном языке \miniKanren{} в подмножестве функционального языка \haskell{}.

\item Проведено экспериментальное исследование реализованного транслятора. 
\begin{itemize}
    \item Реализована система для тестирования результатов трансляции, включающая синтаксический анализатор конкретного синтаксиса \miniKanren{}.
    \item Сформирован набор программ для тестирования, демонстрирующий нетривиальные для трансляции особенности языка \miniKanren{}. 
    \item Осуществлена проверка того, что на этом наборе программ транслятор работает корректно. 
    \item На примере трансляции отношения для сортировки и генерации перестановок элементов списка продемонстрирована применимость реализованного транслятора для ускорения программ на \miniKanren{}.
\end{itemize}

\item Исходный код проекта можно найти в репозитории на сайте~\url{https://github.com/Pluralia/uKanren_translator/}, автор принимал участие под учётной записью \emph{Pluralia}.
\item Результаты работы опубликованы в сборнике конференции SEIM'20 и приняты на конференцию TEASE-LP'20.
\end{itemize}

В дальнейшем планируется.
\begin{itemize}
    \item Решить проблему порядка влияния последовательности конструкций на скорость выполнения транслированной программы, реализовав алгоритм поиска такого порядка конструкций, при котором скорость выполнения транслированной программы была бы максимальной.
    \item Сравнение скорости работы транслятора с существующими решениями ускорения вычислений чистых реляционных языков. 
    \item Разрешить проблему необходимости выдачи транслированной программой ответов в том же порядке, что и транслируемое отношение в транслируемом направлении.
    \item Формально доказать сохранение семантики отношения на языке \miniKanren{} при трансляции в конкретном направлении в функцию на языке \haskell{}.
\end{itemize}
