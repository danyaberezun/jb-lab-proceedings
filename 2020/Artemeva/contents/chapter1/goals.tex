\subsection{Цель и задачи работы}
\label{goals}

\paragraph{Целью} данной работы является создание такого транслятора реляционного языка \miniKanren{} в функциональный язык \haskell{}, что транслированная в конкретном направлении функция будет выдавать то же множество ответов, что и исходное отношение в этом направлении.

Для достижения этой цели были поставлены следующие задачи.

\begin{itemize}
    \item Разработать алгоритм анализа времени связывания для языка \miniKanren{}, позволяющий транслятору из \miniKanren{} в язык \haskell{} определять направления и порядок вычислений конъюнктов за счёт использования информации о времени связывания переменных.
    
    \item Разработать алгоритм трансляции реляционного языка \miniKanren{} в функциональный язык \haskell{}.
    
    \item Провести экспериментальное исследование результатов работы разработанного транслятора из языка \miniKanren{} в язык \haskell{}.

\end{itemize}
