\title{Разработка транслятора из реляционного языка программирования в функциональный}
\titlerunning{Разработка транслятора из реляционного языка программирования в функциональный}

\author{Артемьева Ирина}
\authorrunning{Артемьева И.}

\tocauthor{Артемьева И.}
\institute{Национальный исследовательский университет ИТМО\\
	\email{irina-pluralia@rambler.ru}}

\maketitle

\lstdefinestyle{mycode}{
  belowcaptionskip=1\baselineskip,
  breaklines=true,
  xleftmargin=\parindent,
  showstringspaces=false,
  basicstyle=\footnotesize\ttfamily,
  keywordstyle=\bfseries,
  commentstyle=\itshape\color{gray!40!black},
  stringstyle=\color{red},
  numbers=left,
  numbersep=5pt,
  numberstyle=\tiny\color{gray},
}
\lstset{escapechar=@,style=mycode}

\newcommand{\miniKanren}{\textsc{miniKanren}}
\newcommand{\microKanren}{\textsc{microKanren}}
\newcommand{\mercury}{\textsc{Mercury}}
\newcommand{\haskell}{\textsc{Haskell}}
\newcommand{\prolog}{\textsc{Prolog}}
\newcommand{\scheme}{\textsc{Scheme}}
\newcommand{\logen}{\textsc{LOGEN}}
\newcommand{\ocanren}{\textsc{OCanren}}
\newcommand{\curry}{\textsc{Curry}}
\newcommand{\github}{\textsc{GitHub}}
\lstset{mathescape=true}


\graphicspath{{Artemeva/}}

% \SetKwInput{KwIn}{Вход}
% \SetKwInput{KwOut}{Результат}
% \SetKw{KwFrom}{from}
% \SetKw{KwAnd}{and}
% \SetKw{KwBreak}{break}
% \SetAlgorithmName{Алгоритм}{алгоритм}{Список алгоритмов}


% \newtheorem{theorem}{Утверждение}


%
\filltitle{ru}{
    chair              = {Кафедра компьютерных технологий},
    title              = {Разработка транслятора из реляционного языка программирования в функциональный},
    type               = {master},
    position           = {студента},
    group              = M42363,
    author             = {Артемьева Ирина Александровна},
    supervisorPosition = {},
    supervisor         = {Вербицкая Е.\,А.},
    reviewerPosition   = {},
    reviewer           = {Березун Д.\,A.},
    chairHeadPosition  = {д.\,ф.-м.\,н., профессор},
    chairHead          = {Омельченко А.\,В.},
    university         = {УНИВЕРСИТЕТ ИТМО},
    faculty            = {Факультет информационных технологий и программирования},
    city               = {Санкт-Петербург},
    year               = {2020}
}

\maketitle

\phantomsection
\section*{Введение}

Реляционное программирование~--- это чистая форма логического программирования,
в которой программы представляются как наборы математических отношений~\cite{byrdMK}.
Отношения
не различают входные и выходные параметры, из-за чего одно и то же
отношение может решать несколько связанных проблем. К примеру, отношение, задающее
интерпретатор языка, можно использовать не только для вычисления программ по
заданному входу, но и для генерации возможных входных значений по заданному результату
или самих программ по спецификации входных и выходных значений.

miniKanren~--- это семейство встраиваемых предметно-ориентированных языков программирования~\cite{byrdMK}.
miniKanren был специально сконструирован для поддержки реляционной парадигмы,
опираясь на опыт логических языков, таких как языки семейства Prolog~\cite{logicMJ},
Mercury~\cite{mercury} и Curry~\cite{curry}.

Реляционная парадигма довольно сложна, хотя потенциал её весьма велик.
Часто наиболее естественный способ записи отношения не является эффективным. В
частности, при задании функциональных отношений как сопоставления выходов
входам, как это наблюдается в примере с интерпретатором, поиск входов по выходам практически
всегда работает медленно.

Специализация --- это техника автоматической оптимизации программ,
при которой на основе программы и её частично известного входа
порождается новая, более эффективная программа, которая сохраняет семантику
исходной. Для специализации логических языков используются методы частичной дедукции~\cite{advanced},
самый проработанный из которых --- это \cpd\cite{cpd}. ECCE, реализация \forcpd для Prolog, показывает
хорошие результаты~\cite{controlPoly},
% однако специфика реляционного программирования и его отличия от логических языков подразумевает возможность разработать более подходящий
однако, в силу различий между реляционным и логическим программированием, можно предположить возможность разработать более подходящий
метод специализации. Уже существует адаптация \forcpd для miniKanren~\cite{lozov},
однако её результаты нестабильны: несмотря на то, что в некоторых случаях
производительность программ улучшается, в других -- она может существенно ухудшиться.

Другой подход для специализации --- это суперкомпиляция,
техника автоматической трансформации и анализа программ,
при которой программа символьно исполняется с сохранением истории вычислений,
на основе которой строится оптимизированная версия кода.
Суперкомпиляция успешно применяется к функциональным и императивным языкам,
однако для логических языков не сильно развита. Существуют
работы, посвящённые демонстрации сходства процессов частичной дедукции и суперкомпиляции~\cite{pdAndDriving},
а также предназначенный для Prolog суперкомпилятор APROPOS~\cite{apropos}, который, однако, довольно ограничен
в своих возможностях и требует ручного контроля.

В данной работе предлагается способ адаптации и реализации суперкомпилятора для
реляционного языка miniKanren, а также рассматриваются его возможные вариации, приводящие к
дальнейшему повышению производительности реляционных программ, и производится экспериментальное
исследование результата.


\title{Разработка матричного алгоритма поиска путей с контекстно-свободными ограничениями для RedisGraph}
\titlerunning{Поиск путей с КС ограничениями для RedisGraph}

\author{Терехов Арсений Константинович}
\authorrunning{Терехов~А.~К.}

\tocauthor{Терехов~А.~К.}
\institute{Санкт-Петербургский государственный университет\\
	\email{simpletondl@yandex.ru}}

\maketitle

\begin{abstract}
Поиск путей с контекстно-свободными ограничениями подразумевает использование контекстно-свободной грамматики для задания ограничений на множество искомых путей в графе. Данные ограничения используются в таких областях, как статический анализ кода и анализ RDF-данных. Однако на текущий момент ни одна графовая база данных не поддерживает запросы с контекстно-свободными ограничениями, что препятствует развитию прикладных решений. В данной работе представлено решение данной проблемы: реализована поддержка расширенного необходимыми конструкциями языка запросов Cypher для графовой базы данных RedisGraph.
\end{abstract}

\phantomsection
\section*{Введение}

Реляционное программирование~--- это чистая форма логического программирования,
в которой программы представляются как наборы математических отношений~\cite{byrdMK}.
Отношения
не различают входные и выходные параметры, из-за чего одно и то же
отношение может решать несколько связанных проблем. К примеру, отношение, задающее
интерпретатор языка, можно использовать не только для вычисления программ по
заданному входу, но и для генерации возможных входных значений по заданному результату
или самих программ по спецификации входных и выходных значений.

miniKanren~--- это семейство встраиваемых предметно-ориентированных языков программирования~\cite{byrdMK}.
miniKanren был специально сконструирован для поддержки реляционной парадигмы,
опираясь на опыт логических языков, таких как языки семейства Prolog~\cite{logicMJ},
Mercury~\cite{mercury} и Curry~\cite{curry}.

Реляционная парадигма довольно сложна, хотя потенциал её весьма велик.
Часто наиболее естественный способ записи отношения не является эффективным. В
частности, при задании функциональных отношений как сопоставления выходов
входам, как это наблюдается в примере с интерпретатором, поиск входов по выходам практически
всегда работает медленно.

Специализация --- это техника автоматической оптимизации программ,
при которой на основе программы и её частично известного входа
порождается новая, более эффективная программа, которая сохраняет семантику
исходной. Для специализации логических языков используются методы частичной дедукции~\cite{advanced},
самый проработанный из которых --- это \cpd\cite{cpd}. ECCE, реализация \forcpd для Prolog, показывает
хорошие результаты~\cite{controlPoly},
% однако специфика реляционного программирования и его отличия от логических языков подразумевает возможность разработать более подходящий
однако, в силу различий между реляционным и логическим программированием, можно предположить возможность разработать более подходящий
метод специализации. Уже существует адаптация \forcpd для miniKanren~\cite{lozov},
однако её результаты нестабильны: несмотря на то, что в некоторых случаях
производительность программ улучшается, в других -- она может существенно ухудшиться.

Другой подход для специализации --- это суперкомпиляция,
техника автоматической трансформации и анализа программ,
при которой программа символьно исполняется с сохранением истории вычислений,
на основе которой строится оптимизированная версия кода.
Суперкомпиляция успешно применяется к функциональным и императивным языкам,
однако для логических языков не сильно развита. Существуют
работы, посвящённые демонстрации сходства процессов частичной дедукции и суперкомпиляции~\cite{pdAndDriving},
а также предназначенный для Prolog суперкомпилятор APROPOS~\cite{apropos}, который, однако, довольно ограничен
в своих возможностях и требует ручного контроля.

В данной работе предлагается способ адаптации и реализации суперкомпилятора для
реляционного языка miniKanren, а также рассматриваются его возможные вариации, приводящие к
дальнейшему повышению производительности реляционных программ, и производится экспериментальное
исследование результата.

\input{Terekhov/diploma_bib}

\title{Разработка матричного алгоритма поиска путей с контекстно-свободными ограничениями для RedisGraph}
\titlerunning{Поиск путей с КС ограничениями для RedisGraph}

\author{Терехов Арсений Константинович}
\authorrunning{Терехов~А.~К.}

\tocauthor{Терехов~А.~К.}
\institute{Санкт-Петербургский государственный университет\\
	\email{simpletondl@yandex.ru}}

\maketitle

\begin{abstract}
Поиск путей с контекстно-свободными ограничениями подразумевает использование контекстно-свободной грамматики для задания ограничений на множество искомых путей в графе. Данные ограничения используются в таких областях, как статический анализ кода и анализ RDF-данных. Однако на текущий момент ни одна графовая база данных не поддерживает запросы с контекстно-свободными ограничениями, что препятствует развитию прикладных решений. В данной работе представлено решение данной проблемы: реализована поддержка расширенного необходимыми конструкциями языка запросов Cypher для графовой базы данных RedisGraph.
\end{abstract}

\phantomsection
\section*{Введение}

Реляционное программирование~--- это чистая форма логического программирования,
в которой программы представляются как наборы математических отношений~\cite{byrdMK}.
Отношения
не различают входные и выходные параметры, из-за чего одно и то же
отношение может решать несколько связанных проблем. К примеру, отношение, задающее
интерпретатор языка, можно использовать не только для вычисления программ по
заданному входу, но и для генерации возможных входных значений по заданному результату
или самих программ по спецификации входных и выходных значений.

miniKanren~--- это семейство встраиваемых предметно-ориентированных языков программирования~\cite{byrdMK}.
miniKanren был специально сконструирован для поддержки реляционной парадигмы,
опираясь на опыт логических языков, таких как языки семейства Prolog~\cite{logicMJ},
Mercury~\cite{mercury} и Curry~\cite{curry}.

Реляционная парадигма довольно сложна, хотя потенциал её весьма велик.
Часто наиболее естественный способ записи отношения не является эффективным. В
частности, при задании функциональных отношений как сопоставления выходов
входам, как это наблюдается в примере с интерпретатором, поиск входов по выходам практически
всегда работает медленно.

Специализация --- это техника автоматической оптимизации программ,
при которой на основе программы и её частично известного входа
порождается новая, более эффективная программа, которая сохраняет семантику
исходной. Для специализации логических языков используются методы частичной дедукции~\cite{advanced},
самый проработанный из которых --- это \cpd\cite{cpd}. ECCE, реализация \forcpd для Prolog, показывает
хорошие результаты~\cite{controlPoly},
% однако специфика реляционного программирования и его отличия от логических языков подразумевает возможность разработать более подходящий
однако, в силу различий между реляционным и логическим программированием, можно предположить возможность разработать более подходящий
метод специализации. Уже существует адаптация \forcpd для miniKanren~\cite{lozov},
однако её результаты нестабильны: несмотря на то, что в некоторых случаях
производительность программ улучшается, в других -- она может существенно ухудшиться.

Другой подход для специализации --- это суперкомпиляция,
техника автоматической трансформации и анализа программ,
при которой программа символьно исполняется с сохранением истории вычислений,
на основе которой строится оптимизированная версия кода.
Суперкомпиляция успешно применяется к функциональным и императивным языкам,
однако для логических языков не сильно развита. Существуют
работы, посвящённые демонстрации сходства процессов частичной дедукции и суперкомпиляции~\cite{pdAndDriving},
а также предназначенный для Prolog суперкомпилятор APROPOS~\cite{apropos}, который, однако, довольно ограничен
в своих возможностях и требует ручного контроля.

В данной работе предлагается способ адаптации и реализации суперкомпилятора для
реляционного языка miniKanren, а также рассматриваются его возможные вариации, приводящие к
дальнейшему повышению производительности реляционных программ, и производится экспериментальное
исследование результата.

\input{Terekhov/diploma_bib}

\title{Разработка матричного алгоритма поиска путей с контекстно-свободными ограничениями для RedisGraph}
\titlerunning{Поиск путей с КС ограничениями для RedisGraph}

\author{Терехов Арсений Константинович}
\authorrunning{Терехов~А.~К.}

\tocauthor{Терехов~А.~К.}
\institute{Санкт-Петербургский государственный университет\\
	\email{simpletondl@yandex.ru}}

\maketitle

\begin{abstract}
Поиск путей с контекстно-свободными ограничениями подразумевает использование контекстно-свободной грамматики для задания ограничений на множество искомых путей в графе. Данные ограничения используются в таких областях, как статический анализ кода и анализ RDF-данных. Однако на текущий момент ни одна графовая база данных не поддерживает запросы с контекстно-свободными ограничениями, что препятствует развитию прикладных решений. В данной работе представлено решение данной проблемы: реализована поддержка расширенного необходимыми конструкциями языка запросов Cypher для графовой базы данных RedisGraph.
\end{abstract}

\phantomsection
\section*{Введение}

Реляционное программирование~--- это чистая форма логического программирования,
в которой программы представляются как наборы математических отношений~\cite{byrdMK}.
Отношения
не различают входные и выходные параметры, из-за чего одно и то же
отношение может решать несколько связанных проблем. К примеру, отношение, задающее
интерпретатор языка, можно использовать не только для вычисления программ по
заданному входу, но и для генерации возможных входных значений по заданному результату
или самих программ по спецификации входных и выходных значений.

miniKanren~--- это семейство встраиваемых предметно-ориентированных языков программирования~\cite{byrdMK}.
miniKanren был специально сконструирован для поддержки реляционной парадигмы,
опираясь на опыт логических языков, таких как языки семейства Prolog~\cite{logicMJ},
Mercury~\cite{mercury} и Curry~\cite{curry}.

Реляционная парадигма довольно сложна, хотя потенциал её весьма велик.
Часто наиболее естественный способ записи отношения не является эффективным. В
частности, при задании функциональных отношений как сопоставления выходов
входам, как это наблюдается в примере с интерпретатором, поиск входов по выходам практически
всегда работает медленно.

Специализация --- это техника автоматической оптимизации программ,
при которой на основе программы и её частично известного входа
порождается новая, более эффективная программа, которая сохраняет семантику
исходной. Для специализации логических языков используются методы частичной дедукции~\cite{advanced},
самый проработанный из которых --- это \cpd\cite{cpd}. ECCE, реализация \forcpd для Prolog, показывает
хорошие результаты~\cite{controlPoly},
% однако специфика реляционного программирования и его отличия от логических языков подразумевает возможность разработать более подходящий
однако, в силу различий между реляционным и логическим программированием, можно предположить возможность разработать более подходящий
метод специализации. Уже существует адаптация \forcpd для miniKanren~\cite{lozov},
однако её результаты нестабильны: несмотря на то, что в некоторых случаях
производительность программ улучшается, в других -- она может существенно ухудшиться.

Другой подход для специализации --- это суперкомпиляция,
техника автоматической трансформации и анализа программ,
при которой программа символьно исполняется с сохранением истории вычислений,
на основе которой строится оптимизированная версия кода.
Суперкомпиляция успешно применяется к функциональным и императивным языкам,
однако для логических языков не сильно развита. Существуют
работы, посвящённые демонстрации сходства процессов частичной дедукции и суперкомпиляции~\cite{pdAndDriving},
а также предназначенный для Prolog суперкомпилятор APROPOS~\cite{apropos}, который, однако, довольно ограничен
в своих возможностях и требует ручного контроля.

В данной работе предлагается способ адаптации и реализации суперкомпилятора для
реляционного языка miniKanren, а также рассматриваются его возможные вариации, приводящие к
дальнейшему повышению производительности реляционных программ, и производится экспериментальное
исследование результата.

\input{Terekhov/diploma_bib}

\title{Разработка матричного алгоритма поиска путей с контекстно-свободными ограничениями для RedisGraph}
\titlerunning{Поиск путей с КС ограничениями для RedisGraph}

\author{Терехов Арсений Константинович}
\authorrunning{Терехов~А.~К.}

\tocauthor{Терехов~А.~К.}
\institute{Санкт-Петербургский государственный университет\\
	\email{simpletondl@yandex.ru}}

\maketitle

\begin{abstract}
Поиск путей с контекстно-свободными ограничениями подразумевает использование контекстно-свободной грамматики для задания ограничений на множество искомых путей в графе. Данные ограничения используются в таких областях, как статический анализ кода и анализ RDF-данных. Однако на текущий момент ни одна графовая база данных не поддерживает запросы с контекстно-свободными ограничениями, что препятствует развитию прикладных решений. В данной работе представлено решение данной проблемы: реализована поддержка расширенного необходимыми конструкциями языка запросов Cypher для графовой базы данных RedisGraph.
\end{abstract}

\phantomsection
\section*{Введение}

Реляционное программирование~--- это чистая форма логического программирования,
в которой программы представляются как наборы математических отношений~\cite{byrdMK}.
Отношения
не различают входные и выходные параметры, из-за чего одно и то же
отношение может решать несколько связанных проблем. К примеру, отношение, задающее
интерпретатор языка, можно использовать не только для вычисления программ по
заданному входу, но и для генерации возможных входных значений по заданному результату
или самих программ по спецификации входных и выходных значений.

miniKanren~--- это семейство встраиваемых предметно-ориентированных языков программирования~\cite{byrdMK}.
miniKanren был специально сконструирован для поддержки реляционной парадигмы,
опираясь на опыт логических языков, таких как языки семейства Prolog~\cite{logicMJ},
Mercury~\cite{mercury} и Curry~\cite{curry}.

Реляционная парадигма довольно сложна, хотя потенциал её весьма велик.
Часто наиболее естественный способ записи отношения не является эффективным. В
частности, при задании функциональных отношений как сопоставления выходов
входам, как это наблюдается в примере с интерпретатором, поиск входов по выходам практически
всегда работает медленно.

Специализация --- это техника автоматической оптимизации программ,
при которой на основе программы и её частично известного входа
порождается новая, более эффективная программа, которая сохраняет семантику
исходной. Для специализации логических языков используются методы частичной дедукции~\cite{advanced},
самый проработанный из которых --- это \cpd\cite{cpd}. ECCE, реализация \forcpd для Prolog, показывает
хорошие результаты~\cite{controlPoly},
% однако специфика реляционного программирования и его отличия от логических языков подразумевает возможность разработать более подходящий
однако, в силу различий между реляционным и логическим программированием, можно предположить возможность разработать более подходящий
метод специализации. Уже существует адаптация \forcpd для miniKanren~\cite{lozov},
однако её результаты нестабильны: несмотря на то, что в некоторых случаях
производительность программ улучшается, в других -- она может существенно ухудшиться.

Другой подход для специализации --- это суперкомпиляция,
техника автоматической трансформации и анализа программ,
при которой программа символьно исполняется с сохранением истории вычислений,
на основе которой строится оптимизированная версия кода.
Суперкомпиляция успешно применяется к функциональным и императивным языкам,
однако для логических языков не сильно развита. Существуют
работы, посвящённые демонстрации сходства процессов частичной дедукции и суперкомпиляции~\cite{pdAndDriving},
а также предназначенный для Prolog суперкомпилятор APROPOS~\cite{apropos}, который, однако, довольно ограничен
в своих возможностях и требует ручного контроля.

В данной работе предлагается способ адаптации и реализации суперкомпилятора для
реляционного языка miniKanren, а также рассматриваются его возможные вариации, приводящие к
дальнейшему повышению производительности реляционных программ, и производится экспериментальное
исследование результата.

\input{Terekhov/diploma_bib}

\title{Разработка матричного алгоритма поиска путей с контекстно-свободными ограничениями для RedisGraph}
\titlerunning{Поиск путей с КС ограничениями для RedisGraph}

\author{Терехов Арсений Константинович}
\authorrunning{Терехов~А.~К.}

\tocauthor{Терехов~А.~К.}
\institute{Санкт-Петербургский государственный университет\\
	\email{simpletondl@yandex.ru}}

\maketitle

\begin{abstract}
Поиск путей с контекстно-свободными ограничениями подразумевает использование контекстно-свободной грамматики для задания ограничений на множество искомых путей в графе. Данные ограничения используются в таких областях, как статический анализ кода и анализ RDF-данных. Однако на текущий момент ни одна графовая база данных не поддерживает запросы с контекстно-свободными ограничениями, что препятствует развитию прикладных решений. В данной работе представлено решение данной проблемы: реализована поддержка расширенного необходимыми конструкциями языка запросов Cypher для графовой базы данных RedisGraph.
\end{abstract}

\phantomsection
\section*{Введение}

Реляционное программирование~--- это чистая форма логического программирования,
в которой программы представляются как наборы математических отношений~\cite{byrdMK}.
Отношения
не различают входные и выходные параметры, из-за чего одно и то же
отношение может решать несколько связанных проблем. К примеру, отношение, задающее
интерпретатор языка, можно использовать не только для вычисления программ по
заданному входу, но и для генерации возможных входных значений по заданному результату
или самих программ по спецификации входных и выходных значений.

miniKanren~--- это семейство встраиваемых предметно-ориентированных языков программирования~\cite{byrdMK}.
miniKanren был специально сконструирован для поддержки реляционной парадигмы,
опираясь на опыт логических языков, таких как языки семейства Prolog~\cite{logicMJ},
Mercury~\cite{mercury} и Curry~\cite{curry}.

Реляционная парадигма довольно сложна, хотя потенциал её весьма велик.
Часто наиболее естественный способ записи отношения не является эффективным. В
частности, при задании функциональных отношений как сопоставления выходов
входам, как это наблюдается в примере с интерпретатором, поиск входов по выходам практически
всегда работает медленно.

Специализация --- это техника автоматической оптимизации программ,
при которой на основе программы и её частично известного входа
порождается новая, более эффективная программа, которая сохраняет семантику
исходной. Для специализации логических языков используются методы частичной дедукции~\cite{advanced},
самый проработанный из которых --- это \cpd\cite{cpd}. ECCE, реализация \forcpd для Prolog, показывает
хорошие результаты~\cite{controlPoly},
% однако специфика реляционного программирования и его отличия от логических языков подразумевает возможность разработать более подходящий
однако, в силу различий между реляционным и логическим программированием, можно предположить возможность разработать более подходящий
метод специализации. Уже существует адаптация \forcpd для miniKanren~\cite{lozov},
однако её результаты нестабильны: несмотря на то, что в некоторых случаях
производительность программ улучшается, в других -- она может существенно ухудшиться.

Другой подход для специализации --- это суперкомпиляция,
техника автоматической трансформации и анализа программ,
при которой программа символьно исполняется с сохранением истории вычислений,
на основе которой строится оптимизированная версия кода.
Суперкомпиляция успешно применяется к функциональным и императивным языкам,
однако для логических языков не сильно развита. Существуют
работы, посвящённые демонстрации сходства процессов частичной дедукции и суперкомпиляции~\cite{pdAndDriving},
а также предназначенный для Prolog суперкомпилятор APROPOS~\cite{apropos}, который, однако, довольно ограничен
в своих возможностях и требует ручного контроля.

В данной работе предлагается способ адаптации и реализации суперкомпилятора для
реляционного языка miniKanren, а также рассматриваются его возможные вариации, приводящие к
дальнейшему повышению производительности реляционных программ, и производится экспериментальное
исследование результата.

\input{Terekhov/diploma_bib}


\section*{Заключение}

В рамках данной работы получены следующие результаты.

\begin{itemize}
\item Разработан алгоритм анализа времени связывания для реляционного языка  \miniKanren{}, позволяющий транслятору определять направления и порядок вычисления конъюнктов за счет использования информации о времени связывания переменных. Доказана его терминируемость и согласованность. 
\item Разработан алгоритм трансляции программ на реляционном языке \miniKanren{} в подмножестве функционального языка \haskell{}.

\item Проведено экспериментальное исследование реализованного транслятора. 
\begin{itemize}
    \item Реализована система для тестирования результатов трансляции, включающая синтаксический анализатор конкретного синтаксиса \miniKanren{}.
    \item Сформирован набор программ для тестирования, демонстрирующий нетривиальные для трансляции особенности языка \miniKanren{}. 
    \item Осуществлена проверка того, что на этом наборе программ транслятор работает корректно. 
    \item На примере трансляции отношения для сортировки и генерации перестановок элементов списка продемонстрирована применимость реализованного транслятора для ускорения программ на \miniKanren{}.
\end{itemize}

\item Исходный код проекта можно найти в репозитории на сайте~\url{https://github.com/Pluralia/uKanren_translator/}, автор принимал участие под учётной записью \emph{Pluralia}.
\item Результаты работы опубликованы в сборнике конференции SEIM'20 и приняты на конференцию TEASE-LP'20.
\end{itemize}

В дальнейшем планируется.
\begin{itemize}
    \item Решить проблему порядка влияния последовательности конструкций на скорость выполнения транслированной программы, реализовав алгоритм поиска такого порядка конструкций, при котором скорость выполнения транслированной программы была бы максимальной.
    \item Сравнение скорости работы транслятора с существующими решениями ускорения вычислений чистых реляционных языков. 
    \item Разрешить проблему необходимости выдачи транслированной программой ответов в том же порядке, что и транслируемое отношение в транслируемом направлении.
    \item Формально доказать сохранение семантики отношения на языке \miniKanren{} при трансляции в конкретном направлении в функцию на языке \haskell{}.
\end{itemize}


\bibliographystyle{ugost2008ls}
\bibliography{Artemeva/contents/bibfile}

