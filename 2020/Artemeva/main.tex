\title{Разработка транслятора из реляционного языка программирования в функциональный}
\titlerunning{Транслятора реляционного языка в функциональный}

\author{Артемьева Ирина}
\authorrunning{Артемьева И.}

\tocauthor{Артемьева И.}
\institute{Национальный исследовательский университет ИТМО\\
	\email{irina-pluralia@rambler.ru}}

\maketitle

\lstdefinestyle{mycode}{
  belowcaptionskip=1\baselineskip,
  breaklines=true,
  xleftmargin=\parindent,
  showstringspaces=false,
  basicstyle=\footnotesize\ttfamily,
  keywordstyle=\bfseries,
  commentstyle=\itshape\color{gray!40!black},
  stringstyle=\color{red},
  numbers=left,
  numbersep=5pt,
  numberstyle=\tiny\color{gray},
}
\lstset{escapechar=@,style=mycode}

\newcommand{\miniKanren}{\textsc{miniKanren}}
\newcommand{\microKanren}{\textsc{microKanren}}
\newcommand{\mercury}{\textsc{Mercury}}
\newcommand{\haskell}{\textsc{Haskell}}
\newcommand{\prolog}{\textsc{Prolog}}
\newcommand{\scheme}{\textsc{Scheme}}
\newcommand{\logen}{\textsc{LOGEN}}
\newcommand{\ocanren}{\textsc{OCanren}}
\newcommand{\curry}{\textsc{Curry}}
\newcommand{\github}{\textsc{GitHub}}
\lstset{mathescape=true}

\newcommand{\fromkw}{\textbf{\textit{from}}}
\newcommand{\andkw}{\textbf{\textit{and}}}
\newcommand{\returnkw}{\textbf{return}}
\algrenewcommand\algorithmicindent{1.em}

\graphicspath{{Artemeva/}}

\algnewcommand\algorithmicotherwise{\textbf{otherwise}}
\algrenewcommand\algorithmicreturn{\textbf{return}}

\algdef{SE}[RETURN]{Return}{EndReturn}[1]{\algorithmicreturn\ #1}{\algorithmicend| \algorithmicreturn}

\algdef{SE}[OTHERWISE]{Otherwise}{EndOtherwise}[0]{\algorithmicotherwise\ \algorithmicdo}{\algorithmicend \algoorithmicotherwise}
\algtext*{EndOtherwise}%
\algtext*{EndReturn}%
\algtext*{EndSwitch}%
\algtext*{EndCase}%

% \newtheorem{theorem}{Утверждение}

\phantomsection
\section*{Введение}

Реляционное программирование~--- это чистая форма логического программирования,
в которой программы представляются как наборы математических отношений~\cite{byrdMK}.
Отношения
не различают входные и выходные параметры, из-за чего одно и то же
отношение может решать несколько связанных проблем. К примеру, отношение, задающее
интерпретатор языка, можно использовать не только для вычисления программ по
заданному входу, но и для генерации возможных входных значений по заданному результату
или самих программ по спецификации входных и выходных значений.

miniKanren~--- это семейство встраиваемых предметно-ориентированных языков программирования~\cite{byrdMK}.
miniKanren был специально сконструирован для поддержки реляционной парадигмы,
опираясь на опыт логических языков, таких как языки семейства Prolog~\cite{logicMJ},
Mercury~\cite{mercury} и Curry~\cite{curry}.

Реляционная парадигма довольно сложна, хотя потенциал её весьма велик.
Часто наиболее естественный способ записи отношения не является эффективным. В
частности, при задании функциональных отношений как сопоставления выходов
входам, как это наблюдается в примере с интерпретатором, поиск входов по выходам практически
всегда работает медленно.

Специализация --- это техника автоматической оптимизации программ,
при которой на основе программы и её частично известного входа
порождается новая, более эффективная программа, которая сохраняет семантику
исходной. Для специализации логических языков используются методы частичной дедукции~\cite{advanced},
самый проработанный из которых --- это \cpd\cite{cpd}. ECCE, реализация \forcpd для Prolog, показывает
хорошие результаты~\cite{controlPoly},
% однако специфика реляционного программирования и его отличия от логических языков подразумевает возможность разработать более подходящий
однако, в силу различий между реляционным и логическим программированием, можно предположить возможность разработать более подходящий
метод специализации. Уже существует адаптация \forcpd для miniKanren~\cite{lozov},
однако её результаты нестабильны: несмотря на то, что в некоторых случаях
производительность программ улучшается, в других -- она может существенно ухудшиться.

Другой подход для специализации --- это суперкомпиляция,
техника автоматической трансформации и анализа программ,
при которой программа символьно исполняется с сохранением истории вычислений,
на основе которой строится оптимизированная версия кода.
Суперкомпиляция успешно применяется к функциональным и императивным языкам,
однако для логических языков не сильно развита. Существуют
работы, посвящённые демонстрации сходства процессов частичной дедукции и суперкомпиляции~\cite{pdAndDriving},
а также предназначенный для Prolog суперкомпилятор APROPOS~\cite{apropos}, который, однако, довольно ограничен
в своих возможностях и требует ручного контроля.

В данной работе предлагается способ адаптации и реализации суперкомпилятора для
реляционного языка miniKanren, а также рассматриваются его возможные вариации, приводящие к
дальнейшему повышению производительности реляционных программ, и производится экспериментальное
исследование результата.


\title{Разработка матричного алгоритма поиска путей с контекстно-свободными ограничениями для RedisGraph}
\titlerunning{Поиск путей с КС ограничениями для RedisGraph}

\author{Терехов Арсений Константинович}
\authorrunning{Терехов~А.~К.}

\tocauthor{Терехов~А.~К.}
\institute{Санкт-Петербургский государственный университет\\
	\email{simpletondl@yandex.ru}}

\maketitle

\begin{abstract}
Поиск путей с контекстно-свободными ограничениями подразумевает использование контекстно-свободной грамматики для задания ограничений на множество искомых путей в графе. Данные ограничения используются в таких областях, как статический анализ кода и анализ RDF-данных. Однако на текущий момент ни одна графовая база данных не поддерживает запросы с контекстно-свободными ограничениями, что препятствует развитию прикладных решений. В данной работе представлено решение данной проблемы: реализована поддержка расширенного необходимыми конструкциями языка запросов Cypher для графовой базы данных RedisGraph.
\end{abstract}

\phantomsection
\section*{Введение}

Реляционное программирование~--- это чистая форма логического программирования,
в которой программы представляются как наборы математических отношений~\cite{byrdMK}.
Отношения
не различают входные и выходные параметры, из-за чего одно и то же
отношение может решать несколько связанных проблем. К примеру, отношение, задающее
интерпретатор языка, можно использовать не только для вычисления программ по
заданному входу, но и для генерации возможных входных значений по заданному результату
или самих программ по спецификации входных и выходных значений.

miniKanren~--- это семейство встраиваемых предметно-ориентированных языков программирования~\cite{byrdMK}.
miniKanren был специально сконструирован для поддержки реляционной парадигмы,
опираясь на опыт логических языков, таких как языки семейства Prolog~\cite{logicMJ},
Mercury~\cite{mercury} и Curry~\cite{curry}.

Реляционная парадигма довольно сложна, хотя потенциал её весьма велик.
Часто наиболее естественный способ записи отношения не является эффективным. В
частности, при задании функциональных отношений как сопоставления выходов
входам, как это наблюдается в примере с интерпретатором, поиск входов по выходам практически
всегда работает медленно.

Специализация --- это техника автоматической оптимизации программ,
при которой на основе программы и её частично известного входа
порождается новая, более эффективная программа, которая сохраняет семантику
исходной. Для специализации логических языков используются методы частичной дедукции~\cite{advanced},
самый проработанный из которых --- это \cpd\cite{cpd}. ECCE, реализация \forcpd для Prolog, показывает
хорошие результаты~\cite{controlPoly},
% однако специфика реляционного программирования и его отличия от логических языков подразумевает возможность разработать более подходящий
однако, в силу различий между реляционным и логическим программированием, можно предположить возможность разработать более подходящий
метод специализации. Уже существует адаптация \forcpd для miniKanren~\cite{lozov},
однако её результаты нестабильны: несмотря на то, что в некоторых случаях
производительность программ улучшается, в других -- она может существенно ухудшиться.

Другой подход для специализации --- это суперкомпиляция,
техника автоматической трансформации и анализа программ,
при которой программа символьно исполняется с сохранением истории вычислений,
на основе которой строится оптимизированная версия кода.
Суперкомпиляция успешно применяется к функциональным и императивным языкам,
однако для логических языков не сильно развита. Существуют
работы, посвящённые демонстрации сходства процессов частичной дедукции и суперкомпиляции~\cite{pdAndDriving},
а также предназначенный для Prolog суперкомпилятор APROPOS~\cite{apropos}, который, однако, довольно ограничен
в своих возможностях и требует ручного контроля.

В данной работе предлагается способ адаптации и реализации суперкомпилятора для
реляционного языка miniKanren, а также рассматриваются его возможные вариации, приводящие к
дальнейшему повышению производительности реляционных программ, и производится экспериментальное
исследование результата.

\input{Terekhov/diploma_bib}

\title{Разработка матричного алгоритма поиска путей с контекстно-свободными ограничениями для RedisGraph}
\titlerunning{Поиск путей с КС ограничениями для RedisGraph}

\author{Терехов Арсений Константинович}
\authorrunning{Терехов~А.~К.}

\tocauthor{Терехов~А.~К.}
\institute{Санкт-Петербургский государственный университет\\
	\email{simpletondl@yandex.ru}}

\maketitle

\begin{abstract}
Поиск путей с контекстно-свободными ограничениями подразумевает использование контекстно-свободной грамматики для задания ограничений на множество искомых путей в графе. Данные ограничения используются в таких областях, как статический анализ кода и анализ RDF-данных. Однако на текущий момент ни одна графовая база данных не поддерживает запросы с контекстно-свободными ограничениями, что препятствует развитию прикладных решений. В данной работе представлено решение данной проблемы: реализована поддержка расширенного необходимыми конструкциями языка запросов Cypher для графовой базы данных RedisGraph.
\end{abstract}

\phantomsection
\section*{Введение}

Реляционное программирование~--- это чистая форма логического программирования,
в которой программы представляются как наборы математических отношений~\cite{byrdMK}.
Отношения
не различают входные и выходные параметры, из-за чего одно и то же
отношение может решать несколько связанных проблем. К примеру, отношение, задающее
интерпретатор языка, можно использовать не только для вычисления программ по
заданному входу, но и для генерации возможных входных значений по заданному результату
или самих программ по спецификации входных и выходных значений.

miniKanren~--- это семейство встраиваемых предметно-ориентированных языков программирования~\cite{byrdMK}.
miniKanren был специально сконструирован для поддержки реляционной парадигмы,
опираясь на опыт логических языков, таких как языки семейства Prolog~\cite{logicMJ},
Mercury~\cite{mercury} и Curry~\cite{curry}.

Реляционная парадигма довольно сложна, хотя потенциал её весьма велик.
Часто наиболее естественный способ записи отношения не является эффективным. В
частности, при задании функциональных отношений как сопоставления выходов
входам, как это наблюдается в примере с интерпретатором, поиск входов по выходам практически
всегда работает медленно.

Специализация --- это техника автоматической оптимизации программ,
при которой на основе программы и её частично известного входа
порождается новая, более эффективная программа, которая сохраняет семантику
исходной. Для специализации логических языков используются методы частичной дедукции~\cite{advanced},
самый проработанный из которых --- это \cpd\cite{cpd}. ECCE, реализация \forcpd для Prolog, показывает
хорошие результаты~\cite{controlPoly},
% однако специфика реляционного программирования и его отличия от логических языков подразумевает возможность разработать более подходящий
однако, в силу различий между реляционным и логическим программированием, можно предположить возможность разработать более подходящий
метод специализации. Уже существует адаптация \forcpd для miniKanren~\cite{lozov},
однако её результаты нестабильны: несмотря на то, что в некоторых случаях
производительность программ улучшается, в других -- она может существенно ухудшиться.

Другой подход для специализации --- это суперкомпиляция,
техника автоматической трансформации и анализа программ,
при которой программа символьно исполняется с сохранением истории вычислений,
на основе которой строится оптимизированная версия кода.
Суперкомпиляция успешно применяется к функциональным и императивным языкам,
однако для логических языков не сильно развита. Существуют
работы, посвящённые демонстрации сходства процессов частичной дедукции и суперкомпиляции~\cite{pdAndDriving},
а также предназначенный для Prolog суперкомпилятор APROPOS~\cite{apropos}, который, однако, довольно ограничен
в своих возможностях и требует ручного контроля.

В данной работе предлагается способ адаптации и реализации суперкомпилятора для
реляционного языка miniKanren, а также рассматриваются его возможные вариации, приводящие к
дальнейшему повышению производительности реляционных программ, и производится экспериментальное
исследование результата.

\input{Terekhov/diploma_bib}

\title{Разработка матричного алгоритма поиска путей с контекстно-свободными ограничениями для RedisGraph}
\titlerunning{Поиск путей с КС ограничениями для RedisGraph}

\author{Терехов Арсений Константинович}
\authorrunning{Терехов~А.~К.}

\tocauthor{Терехов~А.~К.}
\institute{Санкт-Петербургский государственный университет\\
	\email{simpletondl@yandex.ru}}

\maketitle

\begin{abstract}
Поиск путей с контекстно-свободными ограничениями подразумевает использование контекстно-свободной грамматики для задания ограничений на множество искомых путей в графе. Данные ограничения используются в таких областях, как статический анализ кода и анализ RDF-данных. Однако на текущий момент ни одна графовая база данных не поддерживает запросы с контекстно-свободными ограничениями, что препятствует развитию прикладных решений. В данной работе представлено решение данной проблемы: реализована поддержка расширенного необходимыми конструкциями языка запросов Cypher для графовой базы данных RedisGraph.
\end{abstract}

\phantomsection
\section*{Введение}

Реляционное программирование~--- это чистая форма логического программирования,
в которой программы представляются как наборы математических отношений~\cite{byrdMK}.
Отношения
не различают входные и выходные параметры, из-за чего одно и то же
отношение может решать несколько связанных проблем. К примеру, отношение, задающее
интерпретатор языка, можно использовать не только для вычисления программ по
заданному входу, но и для генерации возможных входных значений по заданному результату
или самих программ по спецификации входных и выходных значений.

miniKanren~--- это семейство встраиваемых предметно-ориентированных языков программирования~\cite{byrdMK}.
miniKanren был специально сконструирован для поддержки реляционной парадигмы,
опираясь на опыт логических языков, таких как языки семейства Prolog~\cite{logicMJ},
Mercury~\cite{mercury} и Curry~\cite{curry}.

Реляционная парадигма довольно сложна, хотя потенциал её весьма велик.
Часто наиболее естественный способ записи отношения не является эффективным. В
частности, при задании функциональных отношений как сопоставления выходов
входам, как это наблюдается в примере с интерпретатором, поиск входов по выходам практически
всегда работает медленно.

Специализация --- это техника автоматической оптимизации программ,
при которой на основе программы и её частично известного входа
порождается новая, более эффективная программа, которая сохраняет семантику
исходной. Для специализации логических языков используются методы частичной дедукции~\cite{advanced},
самый проработанный из которых --- это \cpd\cite{cpd}. ECCE, реализация \forcpd для Prolog, показывает
хорошие результаты~\cite{controlPoly},
% однако специфика реляционного программирования и его отличия от логических языков подразумевает возможность разработать более подходящий
однако, в силу различий между реляционным и логическим программированием, можно предположить возможность разработать более подходящий
метод специализации. Уже существует адаптация \forcpd для miniKanren~\cite{lozov},
однако её результаты нестабильны: несмотря на то, что в некоторых случаях
производительность программ улучшается, в других -- она может существенно ухудшиться.

Другой подход для специализации --- это суперкомпиляция,
техника автоматической трансформации и анализа программ,
при которой программа символьно исполняется с сохранением истории вычислений,
на основе которой строится оптимизированная версия кода.
Суперкомпиляция успешно применяется к функциональным и императивным языкам,
однако для логических языков не сильно развита. Существуют
работы, посвящённые демонстрации сходства процессов частичной дедукции и суперкомпиляции~\cite{pdAndDriving},
а также предназначенный для Prolog суперкомпилятор APROPOS~\cite{apropos}, который, однако, довольно ограничен
в своих возможностях и требует ручного контроля.

В данной работе предлагается способ адаптации и реализации суперкомпилятора для
реляционного языка miniKanren, а также рассматриваются его возможные вариации, приводящие к
дальнейшему повышению производительности реляционных программ, и производится экспериментальное
исследование результата.

\input{Terekhov/diploma_bib}

\title{Разработка матричного алгоритма поиска путей с контекстно-свободными ограничениями для RedisGraph}
\titlerunning{Поиск путей с КС ограничениями для RedisGraph}

\author{Терехов Арсений Константинович}
\authorrunning{Терехов~А.~К.}

\tocauthor{Терехов~А.~К.}
\institute{Санкт-Петербургский государственный университет\\
	\email{simpletondl@yandex.ru}}

\maketitle

\begin{abstract}
Поиск путей с контекстно-свободными ограничениями подразумевает использование контекстно-свободной грамматики для задания ограничений на множество искомых путей в графе. Данные ограничения используются в таких областях, как статический анализ кода и анализ RDF-данных. Однако на текущий момент ни одна графовая база данных не поддерживает запросы с контекстно-свободными ограничениями, что препятствует развитию прикладных решений. В данной работе представлено решение данной проблемы: реализована поддержка расширенного необходимыми конструкциями языка запросов Cypher для графовой базы данных RedisGraph.
\end{abstract}

\phantomsection
\section*{Введение}

Реляционное программирование~--- это чистая форма логического программирования,
в которой программы представляются как наборы математических отношений~\cite{byrdMK}.
Отношения
не различают входные и выходные параметры, из-за чего одно и то же
отношение может решать несколько связанных проблем. К примеру, отношение, задающее
интерпретатор языка, можно использовать не только для вычисления программ по
заданному входу, но и для генерации возможных входных значений по заданному результату
или самих программ по спецификации входных и выходных значений.

miniKanren~--- это семейство встраиваемых предметно-ориентированных языков программирования~\cite{byrdMK}.
miniKanren был специально сконструирован для поддержки реляционной парадигмы,
опираясь на опыт логических языков, таких как языки семейства Prolog~\cite{logicMJ},
Mercury~\cite{mercury} и Curry~\cite{curry}.

Реляционная парадигма довольно сложна, хотя потенциал её весьма велик.
Часто наиболее естественный способ записи отношения не является эффективным. В
частности, при задании функциональных отношений как сопоставления выходов
входам, как это наблюдается в примере с интерпретатором, поиск входов по выходам практически
всегда работает медленно.

Специализация --- это техника автоматической оптимизации программ,
при которой на основе программы и её частично известного входа
порождается новая, более эффективная программа, которая сохраняет семантику
исходной. Для специализации логических языков используются методы частичной дедукции~\cite{advanced},
самый проработанный из которых --- это \cpd\cite{cpd}. ECCE, реализация \forcpd для Prolog, показывает
хорошие результаты~\cite{controlPoly},
% однако специфика реляционного программирования и его отличия от логических языков подразумевает возможность разработать более подходящий
однако, в силу различий между реляционным и логическим программированием, можно предположить возможность разработать более подходящий
метод специализации. Уже существует адаптация \forcpd для miniKanren~\cite{lozov},
однако её результаты нестабильны: несмотря на то, что в некоторых случаях
производительность программ улучшается, в других -- она может существенно ухудшиться.

Другой подход для специализации --- это суперкомпиляция,
техника автоматической трансформации и анализа программ,
при которой программа символьно исполняется с сохранением истории вычислений,
на основе которой строится оптимизированная версия кода.
Суперкомпиляция успешно применяется к функциональным и императивным языкам,
однако для логических языков не сильно развита. Существуют
работы, посвящённые демонстрации сходства процессов частичной дедукции и суперкомпиляции~\cite{pdAndDriving},
а также предназначенный для Prolog суперкомпилятор APROPOS~\cite{apropos}, который, однако, довольно ограничен
в своих возможностях и требует ручного контроля.

В данной работе предлагается способ адаптации и реализации суперкомпилятора для
реляционного языка miniKanren, а также рассматриваются его возможные вариации, приводящие к
дальнейшему повышению производительности реляционных программ, и производится экспериментальное
исследование результата.

\input{Terekhov/diploma_bib}

\title{Разработка матричного алгоритма поиска путей с контекстно-свободными ограничениями для RedisGraph}
\titlerunning{Поиск путей с КС ограничениями для RedisGraph}

\author{Терехов Арсений Константинович}
\authorrunning{Терехов~А.~К.}

\tocauthor{Терехов~А.~К.}
\institute{Санкт-Петербургский государственный университет\\
	\email{simpletondl@yandex.ru}}

\maketitle

\begin{abstract}
Поиск путей с контекстно-свободными ограничениями подразумевает использование контекстно-свободной грамматики для задания ограничений на множество искомых путей в графе. Данные ограничения используются в таких областях, как статический анализ кода и анализ RDF-данных. Однако на текущий момент ни одна графовая база данных не поддерживает запросы с контекстно-свободными ограничениями, что препятствует развитию прикладных решений. В данной работе представлено решение данной проблемы: реализована поддержка расширенного необходимыми конструкциями языка запросов Cypher для графовой базы данных RedisGraph.
\end{abstract}

\phantomsection
\section*{Введение}

Реляционное программирование~--- это чистая форма логического программирования,
в которой программы представляются как наборы математических отношений~\cite{byrdMK}.
Отношения
не различают входные и выходные параметры, из-за чего одно и то же
отношение может решать несколько связанных проблем. К примеру, отношение, задающее
интерпретатор языка, можно использовать не только для вычисления программ по
заданному входу, но и для генерации возможных входных значений по заданному результату
или самих программ по спецификации входных и выходных значений.

miniKanren~--- это семейство встраиваемых предметно-ориентированных языков программирования~\cite{byrdMK}.
miniKanren был специально сконструирован для поддержки реляционной парадигмы,
опираясь на опыт логических языков, таких как языки семейства Prolog~\cite{logicMJ},
Mercury~\cite{mercury} и Curry~\cite{curry}.

Реляционная парадигма довольно сложна, хотя потенциал её весьма велик.
Часто наиболее естественный способ записи отношения не является эффективным. В
частности, при задании функциональных отношений как сопоставления выходов
входам, как это наблюдается в примере с интерпретатором, поиск входов по выходам практически
всегда работает медленно.

Специализация --- это техника автоматической оптимизации программ,
при которой на основе программы и её частично известного входа
порождается новая, более эффективная программа, которая сохраняет семантику
исходной. Для специализации логических языков используются методы частичной дедукции~\cite{advanced},
самый проработанный из которых --- это \cpd\cite{cpd}. ECCE, реализация \forcpd для Prolog, показывает
хорошие результаты~\cite{controlPoly},
% однако специфика реляционного программирования и его отличия от логических языков подразумевает возможность разработать более подходящий
однако, в силу различий между реляционным и логическим программированием, можно предположить возможность разработать более подходящий
метод специализации. Уже существует адаптация \forcpd для miniKanren~\cite{lozov},
однако её результаты нестабильны: несмотря на то, что в некоторых случаях
производительность программ улучшается, в других -- она может существенно ухудшиться.

Другой подход для специализации --- это суперкомпиляция,
техника автоматической трансформации и анализа программ,
при которой программа символьно исполняется с сохранением истории вычислений,
на основе которой строится оптимизированная версия кода.
Суперкомпиляция успешно применяется к функциональным и императивным языкам,
однако для логических языков не сильно развита. Существуют
работы, посвящённые демонстрации сходства процессов частичной дедукции и суперкомпиляции~\cite{pdAndDriving},
а также предназначенный для Prolog суперкомпилятор APROPOS~\cite{apropos}, который, однако, довольно ограничен
в своих возможностях и требует ручного контроля.

В данной работе предлагается способ адаптации и реализации суперкомпилятора для
реляционного языка miniKanren, а также рассматриваются его возможные вариации, приводящие к
дальнейшему повышению производительности реляционных программ, и производится экспериментальное
исследование результата.

\input{Terekhov/diploma_bib}


\section*{Заключение}

В рамках данной работы получены следующие результаты.

\begin{itemize}
\item Разработан алгоритм анализа времени связывания для реляционного языка  \miniKanren{}, позволяющий транслятору определять направления и порядок вычисления конъюнктов за счет использования информации о времени связывания переменных. Доказана его терминируемость и согласованность. 
\item Разработан алгоритм трансляции программ на реляционном языке \miniKanren{} в подмножестве функционального языка \haskell{}.

\item Проведено экспериментальное исследование реализованного транслятора. 
\begin{itemize}
    \item Реализована система для тестирования результатов трансляции, включающая синтаксический анализатор конкретного синтаксиса \miniKanren{}.
    \item Сформирован набор программ для тестирования, демонстрирующий нетривиальные для трансляции особенности языка \miniKanren{}. 
    \item Осуществлена проверка того, что на этом наборе программ транслятор работает корректно. 
    \item На примере трансляции отношения для сортировки и генерации перестановок элементов списка продемонстрирована применимость реализованного транслятора для ускорения программ на \miniKanren{}.
\end{itemize}

\item Исходный код проекта можно найти в репозитории на сайте~\url{https://github.com/Pluralia/uKanren_translator/}, автор принимал участие под учётной записью \emph{Pluralia}.
\item Результаты работы опубликованы в сборнике конференции SEIM'20 и приняты на конференцию TEASE-LP'20.
\end{itemize}

В дальнейшем планируется.
\begin{itemize}
    \item Решить проблему порядка влияния последовательности конструкций на скорость выполнения транслированной программы, реализовав алгоритм поиска такого порядка конструкций, при котором скорость выполнения транслированной программы была бы максимальной.
    \item Сравнение скорости работы транслятора с существующими решениями ускорения вычислений чистых реляционных языков. 
    \item Разрешить проблему необходимости выдачи транслированной программой ответов в том же порядке, что и транслируемое отношение в транслируемом направлении.
    \item Формально доказать сохранение семантики отношения на языке \miniKanren{} при трансляции в конкретном направлении в функцию на языке \haskell{}.
\end{itemize}

\begin{thebibliography}{10}
  \def\selectlanguageifdefined#1{
  \expandafter\ifx\csname date#1\endcsname\relax
  \else\selectlanguage{#1}\fi}
  \providecommand*{\href}[2]{{\small #2}}
  \providecommand*{\url}[1]{{\small #1}}
  \providecommand*{\BibUrl}[1]{\url{#1}}
  \providecommand{\BibAnnote}[1]{}
  \providecommand*{\BibEmph}[1]{#1}
  \ProvideTextCommandDefault{\cyrdash}{\iflanguage{russian}{\hbox
    to.8em{--\hss--}}{\textemdash}}
  \providecommand*{\BibDash}{\ifdim\lastskip>0pt\unskip\nobreak\hskip.2em plus
    0.1em\fi
  \cyrdash\hskip.2em plus 0.1em\ignorespaces}
  \renewcommand{\newblock}{\ignorespaces}

  \bibitem{interleaving}
  \selectlanguageifdefined{english}
  Backtracking, Interleaving, and Terminating Monad Transformers: (Functional
    Pearl)~/ Oleg~Kiselyov, Chung-chieh~Shan, Daniel~P.~Friedman, Amr~Sabry~//
    \href{http://dx.doi.org/10.1145/1090189.1086390}{\BibEmph{SIGPLAN Not.}}
    \BibDash
  \newblock 2005. \BibDash Sep. \BibDash
  \newblock Vol.~40, no.~9. \BibDash
  \newblock P.~192–203. \BibDash
  \newblock Access mode: \BibUrl{https://doi.org/10.1145/1090189.1086390}.

  \bibitem{scHaskell}
  \selectlanguageifdefined{english}
  \BibEmph{Bolingbroke~Maximilian, Peyton~Jones~Simon}. Supercompilation by
    Evaluation~//
    \href{http://dx.doi.org/10.1145/2088456.1863540}{\BibEmph{SIGPLAN Not.}}
    \BibDash
  \newblock 2010. \BibDash Sep. \BibDash
  \newblock Vol.~45, no.~11. \BibDash
  \newblock P.~135–146. \BibDash
  \newblock Access mode: \BibUrl{https://doi.org/10.1145/2088456.1863540}.

  \bibitem{prologInt}
  \selectlanguageifdefined{english}
  \BibEmph{Bratko~Ivan}. Prolog programming for artificial intelligence (4th
    ed.). \BibDash
  \newblock Harlow, England ; New York: Addison Wesley.

  \bibitem{byrdMK}
  \selectlanguageifdefined{english}
  \BibEmph{Byrd~William}. Relational Programming in miniKanren: Techniques,
    Applications, and Implementations. \BibDash
  \newblock 2009. \BibDash 09. \BibDash
  \newblock P.~297.

  \bibitem{cpd}
  \selectlanguageifdefined{english}
  Conjunctive partial deduction: Foundations, control, algorithms, and
    experiments~/ Danny~De~Schreye, Robert~Gl{\"u}ck, Jesper~J{\o}rgensen
    et~al.~// \BibEmph{The Journal of Logic Programming}. \BibDash
  \newblock 1999. \BibDash
  \newblock Vol.~41, no. 2-3. \BibDash
  \newblock P.~231--277.

  \bibitem{apropos}
  \selectlanguageifdefined{english}
  \BibEmph{Diehl~Stephan}. A Prolog Positive Supercompiler. \BibDash
  \newblock 1997.

  \bibitem{reasonedSchemer}
  \selectlanguageifdefined{english}
  \BibEmph{Friedman~Daniel~P., Byrd~William~E., Kiselyov~Oleg}. The Reasoned
    Schemer. \BibDash
  \newblock The MIT Press, 2005. \BibDash
  \newblock
    ISBN:~\href{http://isbndb.com/search-all.html?kw=0262562146}{0262562146}.

  \bibitem{futamura}
  \selectlanguageifdefined{english}
  \BibEmph{Futamura~Yoshihiko}. Partial Evaluation of Computation Process - An
    Approach to a Compiler-Compiler~// \BibEmph{Systems, Computers, Controls}.
    \BibDash
  \newblock 1999. \BibDash
  \newblock Vol.~2. \BibDash
  \newblock P.~45--50.

  \bibitem{pdAndDriving}
  \selectlanguageifdefined{english}
  \BibEmph{Glück~Robert, Sørensen~Morten~Heine}. Partial Deduction and Driving
    are Equivalent~// Lecture Notes in Computer Science. \BibDash
  \newblock Springer, Berlin, Heidelberg, 1994.

  \bibitem{currySearch}
  \selectlanguageifdefined{english}
  \BibEmph{Hanus~M., Peem{\"o}ller~B., Reck~F.} Search Strategies for Functional
    Logic Programming~// Proc. of the 5th Working Conference on Programming
    Languages (ATPS'12). \BibDash
  \newblock Springer LNI 199, 2012. \BibDash
  \newblock P.~61--74.

  \bibitem{curry}
  \selectlanguageifdefined{english}
  \BibEmph{Hanus~(ed.)~M.} Curry: An Integrated Functional Logic Language (Vers.\
    0.9.0). \BibDash
  \newblock Available at \url{http://www.curry-language.org}. \BibDash
  \newblock 2016.

  \bibitem{uKanren}
  \selectlanguageifdefined{english}
  \BibEmph{Hemann~Jason, Friedman~Daniel~P.} $\mu$Kanren: A Minimal Functional
    Core for Relational Programming~// Proceedings of the 2013 Annual Workshop on
    Scheme and Functional Programming. \BibDash
  \newblock 2013.

  \bibitem{prologTheorem}
  \selectlanguageifdefined{english}
  \BibEmph{Hsiang~Jieh, Srivas~Mandayam}. Automatic inductive theorem proving
    using prolog~//
    \href{http://dx.doi.org/https://doi.org/10.1016/0304-3975(87)90016-8}{\BibEmph{Theoretical
    Computer Science}}. \BibDash
  \newblock 1987. \BibDash
  \newblock Vol.~54, no.~1. \BibDash
  \newblock P.~3 -- 28. \BibDash
  \newblock Access mode:
    \BibUrl{http://www.sciencedirect.com/science/article/pii/0304397587900168}.

  \bibitem{zipper}
  \selectlanguageifdefined{english}
  \BibEmph{Huet~G\'{e}rard}. The Zipper~//
    \href{http://dx.doi.org/10.1017/S0956796897002864}{\BibEmph{J. Funct.
    Program.}} \BibDash
  \newblock 1997. \BibDash Sep. \BibDash
  \newblock Vol.~7, no.~5. \BibDash
  \newblock P.~549–554. \BibDash
  \newblock Access mode: \BibUrl{https://doi.org/10.1017/S0956796897002864}.

  \bibitem{jones}
  \selectlanguageifdefined{english}
  \BibEmph{Jones~Neil~D, Gomard~Carsten~K, Sestoft~Peter}. Partial evaluation and
    automatic program generation. \BibDash
  \newblock Peter Sestoft, 1993.

  \bibitem{prologPE}
  \selectlanguageifdefined{english}
  \BibEmph{Jørgensen~Jesper, Leuschel~Michael}. Efficiently generating efficient
    generating extensions in Prolog~// Partial Evaluation, International Seminar,
    LNCS 1110, pages 238–262, Schloß Dagstuhl. \BibDash
  \newblock Springer-Verlag, 1996. \BibDash
  \newblock P.~238--262.

  \bibitem{cpdPract}
  \selectlanguageifdefined{english}
  \BibEmph{Jørgensen~Jesper, Leuschel~Michael, Martens~Bern}. Conjunctive
    partial deduction in practice~// Proceedings of the International Workshop on
    Logic Program Synthesis and Transformation (LOPSTR'96), LNCS 1207. \BibDash
  \newblock Springer-Verlag, 1996. \BibDash
  \newblock P.~59--82.

  \bibitem{scJava}
  \selectlanguageifdefined{english}
  \BibEmph{Klimov~Andrei~V.} An Approach to Supercompilation for Object-oriented
    Languages: the Java Supercompiler Case Study. \BibDash
  \newblock 2008.

  \bibitem{simplesc}
  \selectlanguageifdefined{english}
  \BibEmph{Klyuchnikov~Ilya, Romanenko~Sergei}. SPSC: a Simple Supercompiler in
    Scala. \BibDash
  \newblock 2009.

  \bibitem{ocanren}
  \selectlanguageifdefined{english}
  \BibEmph{Kosarev~Dmitrii, Boulytchev~Dmitri}. Typed Embedding of a Relational
    Language in OCaml~//
    \href{http://dx.doi.org/10.4204/EPTCS.285.1}{\BibEmph{Electronic Proceedings
    in Theoretical Computer Science}}. \BibDash
  \newblock 2018. \BibDash 12. \BibDash
  \newblock Vol. 285. \BibDash
  \newblock P.~1--22.

  \bibitem{unifRev}
  \selectlanguageifdefined{english}
  \BibEmph{Lassez~Jean-Louis, Maher~Michael, Marriott~Kim}.
    \href{http://dx.doi.org/10.1007/3-540-19129-1_4}{Unification Revisited}.
    \BibDash
  \newblock 2006. \BibDash 01. \BibDash
  \newblock Vol.~306. \BibDash
  \newblock P.~67--113.

  \bibitem{unification}
  \selectlanguageifdefined{english}
  \BibEmph{Lassez~Jean-Louis, Maher~Michael, Marriott~Kim}.
    \href{http://dx.doi.org/10.1007/3-540-19129-1_4}{Unification Revisited}.
    \BibDash
  \newblock 2006. \BibDash 01. \BibDash
  \newblock Vol.~306. \BibDash
  \newblock P.~67--113.

  \bibitem{advanced}
  \selectlanguageifdefined{english}
  Advanced Techniques for Logic Program Specialisation~: Rep.~; Executor:
    Michael~Leuschel~: 1997.

  \bibitem{ecce}
  \selectlanguageifdefined{english}
  The ecce Partial Deduction System~: Rep.~/ In Proc. of the ILPS'97 Workshop on
    Tools and Environments for (Constraint) Logic Programming, U.P~; Executor:
    Michael~Leuschel~: 1997.

  \bibitem{homeo}
  \selectlanguageifdefined{english}
  \BibEmph{Leuschel~Michael}. Homeomorphic embedding for online termination of
    symbolic methods~// In The essence of computation, volume 2566 of LNCS.
    \BibDash
  \newblock Springer, 2002. \BibDash
  \newblock P.~379--403.

  \bibitem{controlPoly}
  \selectlanguageifdefined{english}
  \BibEmph{Leuschel~Michael, Martens~Bern, De~Schreye~Danny}. Controlling
    Generalization and Polyvariance in Partial Deduction of Normal Logic
    Programs~// \href{http://dx.doi.org/10.1145/271510.271525}{\BibEmph{ACM
    Trans. Program. Lang. Syst.}} \BibDash
  \newblock 1998. \BibDash Jan. \BibDash
  \newblock Vol.~20, no.~1. \BibDash
  \newblock P.~208–258. \BibDash
  \newblock Access mode: \BibUrl{https://doi.org/10.1145/271510.271525}.

  \bibitem{raf}
  \selectlanguageifdefined{english}
  \BibEmph{Leuschel~Michael, Soerensen~Morten~Heine}. Redundant Argument
    Filtering of Logic Programs. \BibDash
  \newblock Springer-Verlag, 1996. \BibDash
  \newblock P.~83--103.

  \bibitem{lozov}
  \selectlanguageifdefined{english}
  \BibEmph{Lozov~Petr, Verbitskaia~Ekaterina, Boulytchev~Dmitry}. Relational
    Interpreters for Search Problems~// Relational Programming Workshop. \BibDash
  \newblock 2019. \BibDash
  \newblock P.~43.

  \bibitem{trconv}
  \selectlanguageifdefined{english}
  \BibEmph{Lozov~P.~Vyatkin~A.~Boulytchev~D.} Typed Relational Conversion~//
    \BibEmph{Wang M., Owens S. (eds) Trends in Functional Programming. TFP 2017.
    Lecture Notes in Computer Science, vol 10788. Springer, Cham}. \BibDash
  \newblock 2018.

  \bibitem{relML}
  \selectlanguageifdefined{english}
  Introduction to relational programming~: Rep.~; Executor: Bruce~J.~MacLennan~:
    1981.

  \bibitem{prologExSys}
  \selectlanguageifdefined{english}
  \BibEmph{Merritt~Dennis}. Building Expert Systems in Prolog. \BibDash
  \newblock Spring-Verlag, 1989.

  \bibitem{medMK}
  \selectlanguageifdefined{english}
  \BibEmph{Might~Matthew}. \href{http://dx.doi.org/10.1145/3359061.3365208}{The
    Algorithm for Precision Medicine (Invited Talk)}~// Proceedings Companion of
    the 2019 ACM SIGPLAN International Conference on Systems, Programming,
    Languages, and Applications: Software for Humanity. \BibDash
  \newblock SPLASH Companion 2019. \BibDash
  \newblock New York, NY, USA~: Association for Computing Machinery, 2019.
    \BibDash
  \newblock P.~2. \BibDash
  \newblock Access mode: \BibUrl{https://doi.org/10.1145/3359061.3365208}.

  \bibitem{mkProver}
  \selectlanguageifdefined{english}
  \BibEmph{Near~Joseph, Byrd~William, Friedman~Daniel}.
    \href{http://dx.doi.org/10.1007/978-3-540-89982-2_26}{aleanTAP: A Declarative
    Theorem Prover for First-Order Classical Logic}. \BibDash
  \newblock Vol.~5366. \BibDash
  \newblock 2008. \BibDash 12. \BibDash
  \newblock P.~238--252.

  \bibitem{optimus}
  \selectlanguageifdefined{english}
  Optimus Prime: A new tool for interactive transformation and supercompilation
    of functional programs~: Rep.~/ The University of York, Department of
    Computer Science~; Executor: Jason~S.~Reich~: 2009.

  \bibitem{semanticMK}
  \selectlanguageifdefined{english}
  \BibEmph{Rozplokhas~Dmitry, Boulytchev~Dmitry}. Certified Semantics for
    miniKanren~// miniKanren and Relational Programming Workshop. \BibDash
  \newblock 2019.

  \bibitem{scPerf}
  \selectlanguageifdefined{english}
  \BibEmph{Secher~Jens~Peter, Sørensen~Morten~Heine}. On perfect
    supercompilation~// \BibEmph{Journal of Functional Programming}. \BibDash
  \newblock 1996. \BibDash
  \newblock Vol.~6. \BibDash
  \newblock P.~465--479.

  \bibitem{soerensen1996positive}
  \selectlanguageifdefined{english}
  \BibEmph{Soerensen~Morten~Heine, Gl{\"u}ck~Robert, Jones~Neil~D.} A positive
    supercompiler~// \BibEmph{Journal of functional programming}. \BibDash
  \newblock 1996. \BibDash
  \newblock Vol.~6, no.~6. \BibDash
  \newblock P.~811--838.

  \bibitem{mercury}
  \selectlanguageifdefined{english}
  \BibEmph{Somogyi~Zoltan, Henderson~Fergus~James, Conway~Thomas~Charles}. The
    implementation of Mercury, an efficient purely declarative logic programming
    language~// IN PROCEEDINGS OF THE AUSTRALIAN COMPUTER SCIENCE CONFERENCE.
    \BibDash
  \newblock 1995. \BibDash
  \newblock P.~499--512.

  \bibitem{scGen}
  \selectlanguageifdefined{english}
  \BibEmph{S{\o}rensen~Morten~Heine, Gl{\"u}ck~Robert}. An Algorithm of
    Generalization in Positive Supercompilation~// ILPS. \BibDash
  \newblock 1995.

  \bibitem{scPos}
  \selectlanguageifdefined{english}
  \BibEmph{S\o{}rensen~Morten~Heine, Gl\"{u}ck~Robert}. Introduction to
    Supercompilation~// Partial Evaluation - Practice and Theory, DIKU 1998
    International Summer School. \BibDash
  \newblock Berlin, Heidelberg~: Springer-Verlag, 1998. \BibDash
  \newblock P.~246–270.

  \bibitem{offlinePD}
  \selectlanguageifdefined{english}
  \href{http://dx.doi.org/10.1007/978-3-540-25951-0_11}{Specialising Interpreters
    Using Offline Partial Deduction}~/ Michael~Leuschel, Stephen-John~Craig,
    Maurice~Bruynooghe, Wim~Vanhoof. \BibDash
  \newblock Vol.~3049. \BibDash
  \newblock 2004. \BibDash 01. \BibDash
  \newblock P.~340--375.

  \bibitem{scompRevisited}
  \selectlanguageifdefined{english}
  \BibEmph{Sørensen~Morten~Heine}. Turchin's Supercompiler Revisited - An
    operational theory of positive information propagation. \BibDash
  \newblock 1996.

  \bibitem{tupling}
  \selectlanguageifdefined{english}
  Tupling Calculation Eliminates Multiple Data Traversals~/ Zhenjiang~Hu,
    Hideya~Iwasaki, Masato~Takeichi, Akihiko~Takano~//
    \href{http://dx.doi.org/10.1145/258948.258964}{\BibEmph{ACM SIGPLAN
    Notices}}. \BibDash
  \newblock 1997. \BibDash 12. \BibDash
  \newblock Vol.~32.

  \bibitem{turchinSC}
  \selectlanguageifdefined{english}
  \BibEmph{Turchin~Valentin~F.} The Concept of a Supercompiler~// \BibEmph{ACM
    Transactions on Programming Languages and Systems}. \BibDash
  \newblock 1986. \BibDash
  \newblock Vol.~8. \BibDash
  \newblock P.~292--325.

  \bibitem{unifiedMK}
  \selectlanguageifdefined{english}
  A Unified Approach to Solving Seven Programming Problems (Functional Pearl)~/
    William~E.~Byrd, Michael~Ballantyne, Gregory~Rosenblatt, Matthew~Might~//
    \href{http://dx.doi.org/10.1145/3110252}{\BibEmph{Proc. ACM Program. Lang.}}
    \BibDash
  \newblock 2017. \BibDash Aug. \BibDash
  \newblock Vol.~1, no. ICFP. \BibDash
  \newblock Access mode: \BibUrl{https://doi.org/10.1145/3110252}.

  \bibitem{mkLing}
  \selectlanguageifdefined{english}
  \BibEmph{Varju~Zoltan, Littauer~Richard, Erins~Peteris}. Using Clojure in
    Linguistic Computing. \BibDash
  \newblock 2012. \BibDash 02.

  \bibitem{deforest}
  \selectlanguageifdefined{english}
  \BibEmph{Wadler~Philip}. Deforestation: Transforming Programs to Eliminate
    Trees~// Proceedings of the 2nd European Symposium on Programming. \BibDash
  \newblock ESOP ’88. \BibDash
  \newblock Berlin, Heidelberg~: Springer-Verlag, 1988. \BibDash
  \newblock P.~344–358.

  \bibitem{mkConstr}
  \selectlanguageifdefined{english}
  cKanren miniKanren with Constraints~/ Claire~E.~Alvis, Jeremiah~J.~Willcock,
    Kyle~M.~Carter et~al.~// In Proceedings of the 2011 Workshop on Scheme and
    Functional Programming (Scheme '11), Portland, OR. \BibDash
  \newblock 2011.

  \bibitem{scmini}
  \selectlanguageifdefined{russian}
  \BibEmph{Ключников~Илья.} Суперкомпиляция: идеи
    и методы~// \BibEmph{Практика функционального
    программирования, №7.} \BibDash
  \newblock 2011.

  \bibitem{logicMJ}
  \selectlanguageifdefined{russian}
  \BibEmph{Малпас~Дж.} Реляционный язык Пролог и
    его применение. \BibDash
  \newblock M.: Наука, 1990.

  \end{thebibliography}
